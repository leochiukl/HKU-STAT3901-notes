\section{Premiums}
\label{sect:premiums}
\begin{enumerate}
\item As mentioned in \labelcref{it:charge-premiums}, premiums
\faIcon{dollar-sign} are charged for providing a life insurance
\faIcon{file-alt} (same for life annuity).

\item Typically premiums \faIcon{dollar-sign} are \emph{series} of \emph{level}
payments made at equal interval, as long as the policyholder \faIcon{user} is
alive (possibly with a maximum number of payments to be made).  Hence, the
premiums can actually be seen as a \emph{life annuity}
\faIcon{hand-holding-usd} (but payments are given to the insurer
\faIcon{building}, not to the policyholder \faIcon{user}).

\begin{tikzpicture}
\draw[-Latex] (0,0) -- (10,0) node[right]{Time};
\fill[] (0,0) circle [radius=0.05]
node[below] {0};
\fill[] (2,0) circle [radius=0.05]
node[below] {1};
\fill[] (4,0) circle [radius=0.05]
node[below] {2};
\fill[] (6,0) circle [radius=0.05]
node[below] {3};
\fill[] (7,0) circle [radius=0.05]
node[above] {\faIcon{skull}};
\fill[] (8,0) circle [radius=0.05]
node[below] {4};
\draw[->, red, line width=0.3mm] (0,-0.2) -- (0,0.2)
node[pos=1.6] {\faIcon{dollar-sign}};
\draw[->, red, line width=0.3mm] (2,-0.2) -- (2,0.2)
node[pos=1.6] {\faIcon{dollar-sign}};
\draw[->, red, line width=0.3mm] (4,-0.2) -- (4,0.2)
node[pos=1.6] {\faIcon{dollar-sign}};
\draw[->, red, line width=0.3mm] (6,-0.2) -- (6,0.2)
node[pos=1.6] {\faIcon{dollar-sign}};
\draw[pen colour=violet!30!white, very thick, decorate,decoration={calligraphic brace, amplitude=5pt, raise=20pt}] (0,0) -- (10,0)
node[midway, above=1cm, violet!80!white] {insurance};
\draw[->, ForestGreen, line width=0.3mm] (8,0.2) -- (8,-0.2)
node[pos=1.8] {\faIcon{money-bill-wave}};
\end{tikzpicture}

\item Due to the importance of premiums \faIcon{dollar-sign} to the insurer's
profitability \faIcon{piggy-bank}, charging suitable amounts of premiums
\faIcon{dollar-sign} is crucial. If the amounts of premiums
\faIcon{dollar-sign} are too low, \faIcon{building} can go bankrupt easily; But
if the amounts are too high, \faIcon{building} would be not ``competitive
enough'' and perhaps lose a lot of clients \faIcon{users}. Hence, the insurer
\faIcon{building} (more specifically, \emph{actuarial pricing team}) needs to
consider many factors when determining the premiums.

\item A way to determine the premiums (called a \defn{premium principle}) is
the \emph{equivalence principle}, which is quite ``simple'' and makes the
calculations mathematically convenient. Based on the equivalence principle, the
amounts of (level) premiums are set such that the APV of premiums equals the
APV of the benefits (and expenses, if considered). In other words, the expected
\emph{present value of future loss at issue} \faIcon{fire-alt}
\footnote{``Future loss'' here means future ``net'' cash outflows from the
\emph{insurer}'s \faIcon{building} perspective.} is zero.

\begin{note}
But of course, in the actual practice, determining premiums is far from just
merely following a ``premium principle'' --- it is much more complex. To learn
more, consider having an actuarial internship in pricing team
\faIcon[regular]{smile}.
\end{note}


\item The present value of future loss at issue \faIcon{fire-alt} (which is a life
contingent random variable) serves as an important basis for determining
premiums \faIcon{dollar-sign} and assessing the insurer's profitability
\faIcon{piggy-bank}.
\end{enumerate}
\subsection{Present Value of Future Loss at Issue}
\begin{enumerate}
\item \label{it:pv-future-loss-at-issue}
\defn{Present value of future loss at issue}, from the insurer's \faIcon{building}
perspective, can be expressed as
\begin{align*}
&\text{PV of ``future'' benefits \faIcon{money-bill-wave} outgo} \\
&\quad+ \text{PV of ``future'' expenses \faIcon{file-invoice-dollar} (if considered)}\\
&\quad- \text{PV of ``future'' premium incomes \faIcon{dollar-sign}}
\end{align*}
(which is a random variable).

\begin{warning}
One should be careful that, \emph{conventionally}, ``future'' \emph{insurance}
benefits and the related expenses do \underline{not} include the ones at time
0, while ``future'' premiums and premium-related expenses \underline{do}
include the ones at time 0.\footnote{This applies similarly in a more general
situation (see \cref{sect:policy-values}) where ``future'' is taken with
respect a time \(t\) (just replace ``time 0'' by ``time \(t\)'' here in such
case).}

But for \emph{life annuity} benefit payments and the related expenses,
there is not a standard convention, so more specifications are needed in case
they appear in this context.  (Fortunately it is rather uncommon to encounter
such situation. Often we are interested in finding premiums for insurance
policies.)
\end{warning}

\item When the expenses \faIcon{file-invoice-dollar} are considered, the
present value of future loss at issue is known as \defn{gross loss at issue}.
Otherwise it is known as \defn{net loss at issue}.

\item The gross (net) loss at issue is denoted by \(L_0^g\)
(\(L_0^n\) resp.).

\begin{remark}
\item If there is no risk of ambiguity or one would like to refer to either of gross
and net losses at issue simultaneously, one may drop the superscript and just
write \(L_0\).
\item The ``0'' indicates the present value is at time 0.
\end{remark}

\item Three common types of expenses \faIcon{file-invoice-dollar}:
\begin{enumerate}
\item \defn{initial expenses} \faIcon{hourglass-start}: expenses incurred initially (at time of policy
issue, i.e., time 0);
\item \defn{renewal expenses} \faIcon{redo}: expenses incurred each time a premium
\faIcon{dollar-sign} (or a life annuity benefit payment
\faIcon{hand-holding-usd}) is paid (except the first one, if it is paid at time
0);
\item \defn{termination expenses} (or \defn{settlement expenses})
\faIcon{hourglass-end}: expenses incurred when a policy terminates
\faIcon{hourglass-end} (e.g., death benefit/endowment survival benefit is
paid).
\end{enumerate}

\begin{tikzpicture}
\draw[-Latex] (0,0) -- (10,0) node[right]{Time};
\fill[] (0,0) circle [radius=0.05]
node[below] {0};
\fill[] (2,0) circle [radius=0.05]
node[below] {1};
\fill[] (4,0) circle [radius=0.05]
node[below] {2};
\fill[] (6,0) circle [radius=0.05]
node[below] {3};
\fill[] (7,0) circle [radius=0.05]
node[above] {\faIcon{skull}};
\fill[] (8,0) circle [radius=0.05]
node[below] {4};
\draw[->, red, line width=0.3mm] (0,-0.2) -- (0,0.2)
node[pos=1.6] {\faIcon{dollar-sign}}
node[pos=2.6] {\faIcon{hourglass-start}};
\draw[->, red, line width=0.3mm] (2,-0.2) -- (2,0.2)
node[pos=1.6] {\faIcon{dollar-sign}}
node[pos=2.6] {\faIcon{redo}};
\draw[->, red, line width=0.3mm] (4,-0.2) -- (4,0.2)
node[pos=1.6] {\faIcon{dollar-sign}}
node[pos=2.6] {\faIcon{redo}};
\draw[->, red, line width=0.3mm] (6,-0.2) -- (6,0.2)
node[pos=1.6] {\faIcon{dollar-sign}}
node[pos=2.6] {\faIcon{redo}};
\draw[->, ForestGreen, line width=0.3mm] (7.95,0.2) -- (7.95,-0.2)
node[pos=1.8] {\faIcon{money-bill-wave}};
\draw[->, red, line width=0.3mm] (8.05,-0.2) -- (8.05,0.2)
node[pos=1.6] {\faIcon{hourglass-end}};
\end{tikzpicture}

\item Two commonly seen terminologies (for life insurance):
\begin{itemize}
\item \defn{fully discrete}: both death benefit and premiums are made at
discrete time points (and so do the associated expenses), where the timing
follows the convention in \labelcref{it:prem-ben-conventions};

\begin{tikzpicture}
\draw[-Latex] (0,0) -- (10,0) node[right]{Time};
\fill[] (0,0) circle [radius=0.05]
node[below] {0};
\fill[] (2,0) circle [radius=0.05]
node[below] {1};
\fill[] (4,0) circle [radius=0.05]
node[below] {2};
\fill[] (6,0) circle [radius=0.05]
node[below] {3};
\fill[] (7,0) circle [radius=0.05]
node[above] {\faIcon{skull}};
\fill[] (8,0) circle [radius=0.05]
node[below] {4};
\draw[->, red, line width=0.3mm] (0,-0.2) -- (0,0.2)
node[pos=1.6] {\faIcon{dollar-sign}};
\draw[->, red, line width=0.3mm] (2,-0.2) -- (2,0.2)
node[pos=1.6] {\faIcon{dollar-sign}};
\draw[->, red, line width=0.3mm] (4,-0.2) -- (4,0.2)
node[pos=1.6] {\faIcon{dollar-sign}};
\draw[->, red, line width=0.3mm] (6,-0.2) -- (6,0.2)
node[pos=1.6] {\faIcon{dollar-sign}};
\draw[->, ForestGreen, line width=0.3mm] (8,0.2) -- (8,-0.2)
node[pos=1.8] {\faIcon{money-bill-wave}};
\end{tikzpicture}

\item \defn{fully continuous}: death benefit is payable at the moment of death
and premiums are payable continuously (and so do the associated expenses).

\begin{tikzpicture}
\draw[-Latex] (0,0) -- (10,0) node[right]{Time};
\fill[] (0,0) circle [radius=0.05]
node[below] {0};
\fill[] (2,0) circle [radius=0.05]
node[below] {1};
\fill[] (4,0) circle [radius=0.05]
node[below] {2};
\fill[] (6,0) circle [radius=0.05]
node[below] {3};
\fill[] (7,0) circle [radius=0.05]
node[above] {\faIcon{skull}};
\fill[] (8,0) circle [radius=0.05]
node[below] {4};
\draw[line width=1mm, red] (0,0.7) -- (7,0.7)
node[midway, above] {\faIcon{dollar-sign}};
\draw[->, ForestGreen, line width=0.3mm] (7,0.2) -- (7,-0.2)
node[pos=1.8] {\faIcon{money-bill-wave}};
\end{tikzpicture}
\end{itemize}

\item For the \defn{equivalence principle},
\begin{itemize}
\item the \defn{net premium} (i.e., premium set without considering expenses)
is set such that \(\expv{L_0^n}=0\), i.e.,
\[
\text{APV of ``future'' benefits \faIcon{money-bill-wave} outgo} = \text{APV of
``future'' premium incomes \faIcon{dollar-sign}}; 
\]

\item the \defn{gross premium} (i.e., premium set with expenses considered) is
set such that \(\expv{L_0^g}=0\), i.e.,
\[
\text{APV of ``future'' benefits \faIcon{money-bill-wave} outgo}
+ \text{APV of ``future'' expenses \faIcon{file-invoice-dollar}}
=\text{APV of ``future'' premium incomes \faIcon{dollar-sign}}.
\]
\end{itemize}
\begin{note}
Sometimes for \emph{net premium}, its \emph{definition} also states that it is
determined via equivalence principle, e.g., \textcite{dickson2019actuarial}. In
such case, there is no need to specify whether equivalence principle is used or
not for \emph{net premium}. (For gross premium, one still needs to specify
whether equivalence principle is used.)
\end{note}
\end{enumerate}

\subsection{Actuarial Notations}
\begin{enumerate}
\item \label{it:net-annual-prem-notations}
Some actuarial notations are designed for amounts of \emph{net} annual premiums
(determined via equivalence principle) for \emph{fully discrete} ``standard''
policies (i.e., survival/death benefit is of amount 1):
 \begin{enumerate}
\item whole life insurance \faIcon{infinity}\textbar\faIcon{shield-alt}:
\(\Px{x}\);
\item term life insurance \faIcon{clock}\textbar\faIcon{shield-alt}: \(\Px{\:\termxn}\);
\item endowment insurance \faIcon{clock}\textbar\faIcon{shield-alt}\textbar\faIcon{donate}: \(\Px{\endowxn}\);
\item pure endowment \faIcon{donate}: \(\Px{\pureendowxn}\).
\end{enumerate}

\item \label{it:net-prem-fmlas}
By equivalence principle and some results in \cref{sect:life-annuity}, we can
easily get:
\begin{enumerate}
\item \(\Px{x}=\Ax{x}/\ax**{x}=(d\Ax{x})/(1-\Ax{x})=1/\ax**{x}-d\);
\item \(\Px{\:\termxn}=\Ax{\termxn}/\ax**{\endowxn}\);
\item \(\Px{\endowxn}=\Ax{\endowxn}/\ax**{\endowxn}=(d\Ax{\endowxn})/(1-\Ax{\endowxn})=1/\ax**{\endowxn}-d\);
\item \(\Px{\pureendowxn}=\Ax{\pureendowxn}/\ax**{\endowxn}\).
\end{enumerate}

\begin{remark}
\item By changing \(\Ax{}\to\Ax*{}\), \(\ax**{}\to\ax*{}\) and \(d\to\delta\) in the
formulas above, we can obtain the formulas for the respective \emph{fully
continuous} policies. (But there are not specific actuarial notations designed
for fully continuous policies.)
\item To get the net premiums when the survival/death benefit is of amount
\(S\), multiply the corresponding notation by \(S\) (since the APV of ``future''
benefits \faIcon{money-bill-wave} outgo would just be \(S\) times greater).
\end{remark}

\item Apart from the annual net premium case, there are also actuarial
notations for the \(1/m\)thly net premium case (while the insurance benefits
are still in the annual case). To get the \emph{total amount} of premiums in each
year, we just need to add the superscript ``\((m)\)'' on each notation
introduced in \labelcref{it:net-annual-prem-notations}.

\begin{warning}
This means the amount of \emph{each} premium payment (at the beginning of
each \(1/m\)th of a year) is given by the corresponding notation \emph{divided
by} \(m\).
\end{warning}

For obtaining formulas analogous to \labelcref{it:net-prem-fmlas} in this case,
we change \(\ax**{}\to\ax**{}[(m)]\) and \(d\to d^{(m)}\).
\end{enumerate}
\subsection{Probability and Variance Calculations for \(L_0\)}
\begin{enumerate}
\item To calculate probabilities and variances for \(L_0\), generally the
first step is to write down an expression for \(L_0\).

\item \label{it:wl-gross-loss-expr}
For example, for a whole life insurance with sum insured
\(S\), termination expense \(E\), initial and renewal expenses \(e\) (each),
and annual (level) gross premiums \(G\) (each), the gross loss at issue is
\[
L_0^g=(S+E)v^{K_x+1}-(G-e)\ax**{\angl{K_x+1}},
\]
or
\[
L_0^g=(S+E)Z-(G-e)Y,
\]
where \(Z\) and \(Y\) are the p.v.r.v's for annual whole life insurance and
annual whole life annuity-due (with benefits being 1 [each]) respectively.

\item With the expression of \(L_0^g\), we can easily derive a formula for
\(\vari{L_0^g}\): 
\begin{proposition}
\label{prp:wl-gross-loss-var-fmla}
For the whole life insurance with setting described in
\labelcref{it:wl-gross-loss-expr}, we have
\[
\vari{L_0^g}=\qty(S+E+\frac{G-e}{d})^{2}(\Ax[][2]{x}-\Ax{x}^2)
\]
\end{proposition}
\begin{pf}
Since \(Y=(1-Z)/d\), we can actually express \(L_0^g\) as:
\[
L_0^g=\qty(S+E+\frac{G-e}{d})Z-\frac{G-e}{d}.
\]
The result then follows.
\end{pf}

\item By removing all expenses in the setting in \labelcref{it:wl-gross-loss-expr}
and changing the annual gross premiums \(G\) (each) to the annual net premiums
\(SP_x\) (each) (determined via equivalence principle), we can derive the
following expression for net loss at issue:
\[
L_0^n=SZ-S\Px{x}Y.
\]
We can then derive a formula for \(\vari{L_0^n}\):
\begin{proposition}
\label{prp:wl-net-loss-var-fmla}
For the whole insurance with the setting here, we have
\[
\vari{L_0^n}=\qty(\frac{S}{1-\Ax{x}})^{2}(\Ax[][2]{x}-\Ax{x}^2).
\]
\end{proposition}
\begin{pf}
Note that
\[
L_0^n=\qty(S+\frac{S\Px{x}}{d})Z-\frac{S\Px{x}}{d}
=S\qty(1+\frac{\Ax{x}}{1-\Ax{x}})Z-\frac{S\Px{x}}{d}
=\frac{S}{1-\Ax{x}}Z-\frac{S\Px{x}}{d}.
\]
\end{pf}
\item Using similar arguments, one can adapt
\cref{prp:wl-gross-loss-var-fmla,prp:wl-net-loss-var-fmla} for
\begin{itemize}
\item endowment insurance: change \(\Ax{x}\to\Ax{\endowxn}\) and \(\Ax[][2]{x}\to\Ax[][2]{\endowxn}\);
\item \(1/m\)thly case (for both insurance and premiums): change
\(\Ax{}\to\Ax{}[(m)]\) and \(d\to d^{(m)}\);
\item fully continuous case: change \(\Ax{}\to\Ax*{}\) and \(d\to\delta\).
\end{itemize}
\begin{note}
This is possible mainly because under each of these cases, we have a similar
relationship between the p.v.r.v.'s \(Z\) and \(Y\), and a similar expression
for ``\(\Px{}\;\)''. So, the proofs can go through with only a little change.
\end{note}
\item \label{it:first-principle-cal-var}
An alternative method for calculating variance is the so-called \emph{first
principle} approach, which is sometimes useful for \emph{short-term} fully
discrete insurance policies. The steps are as follows:
\begin{enumerate}
\item Identify all possible values of \(L_0\) with associated probabilities.
\begin{note}
A typical approach to find the associated probabilities is to ``convert'' each
event \(\qty{L_0=j}\) to another event \(\qty{K_x=k}\) which has \emph{the same
probability}. (Finding the probability for the latter event is simple.)
\end{note}
\item Calculate the first and second (raw) moments of \(L_0\):
\begin{itemize}
\item \(\displaystyle
\expv{L_0}=\sum_{}^{}\text{value}\times\text{probability}\) (which is always 0
if equivalence principle is used);
\item \(\displaystyle \expv{L_0^2}=\sum_{}^{}\text{value}^2\times\text{probability}\).
\end{itemize}
\item Calculate the variance
\[
\vari{L_0}=\expv{L_0^2}-\expv{L_0}^2.
\]
\end{enumerate}
\item \label{it:cal-loss-prob}
Now for calculating probabilities for \(L_0\), we often also use the
``conversion'' approach suggested above: ``converting'' the event of interest
in terms of \(L_0\) to another event in terms of \(K_x\) or \(T_x\) which has
the same probability.

\item \label{it:approx-loss-prob}
Another approach is to \emph{approximate} the probabilities for \(L_0\)
through \emph{normal approximation}. If \(L_0\) is loss at issue for a ``large
block'' of independent policies (i.e., a sum of independent losses at issue for
``many'' policies), then \(L_0\overset{\text{approx}}{\sim} N(\expv{L_0},\vari{L_0})\).

\item Then for the normal approximation, we \emph{treat} \(L_0\) as such a
normally distributed random variable, and finding the approximated
probabilities by standardization, i.e., ``converting'' to the event of interest
involving \(L_0\) to another event involving \(\displaystyle
\frac{L_0-\expv{L_0}}{\sqrt{\vari{L_0}}}\overset{\text{this situation}}{\sim} N(0,1)\)
which has the same probability.
\end{enumerate}

