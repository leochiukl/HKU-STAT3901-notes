\section{Introduction to Life Contingencies}
\subsection{Introduction to Life Insurance Business}
\begin{enumerate}
\item Suppose an individual \faIcon{user} (the \defn{policyholder}) wants to buy
a life insurance policy for himself.  Should the insurance company
\faIcon{building} accept his application and sells the policy \faIcon{file-alt}
to him? Often this is decided by performing \emph{underwriting} \faIcon{search}.

\item During the \defn{underwriting} \faIcon{search} process, \faIcon{user} provides some of his
information to \faIcon{building}, like age, gender, occupation, etc.

\item After \faIcon{search} is finished, there are typically three outcomes:
\begin{enumerate}

\item accept with standard premium charged (most common)

\item accept with ``higher than standard'' premium charged (this indicates that
\faIcon{user} has higher risk than usual based on the information provided)

\item reject (the risk level suggested by the information is so high that the
policyholder is \emph{uninsurable} [e.g., it is ``almost'' sure that he will
die very soon])
\end{enumerate}

\item Now suppose \faIcon{building} accepts his application with standard premium
charged. Then, \faIcon{user} will periodically pay \defn{premiums} \faDollarSign{}
to \faIcon{building}, in exchange for the life insurance coverage.

\item Note that the \emph{life insured} for this policy is the policyholder
\faIcon{user} himself. But it is possible that the life insured is someone else
\faIcon{walking}.

\item If the life insured \faIcon{walking} dies within the term covered by the
policy, then \faIcon{building} will pay the \defn{sum insured}/\defn{death
benefit} \faIcon{money-bill-wave} (as specified in the policy
\faIcon{file-alt}) to the policyholder \faIcon{user}.

\begin{note}
From this we can observe a potential issue: if the death of \faIcon{walking}
causes no harm to \faIcon{user} at all, then the policy may \emph{incentivize}
\faIcon{user} to \textcolor{red}{\emph{kill}} \faIcon{walking}! To deal with this
problem, the underwriting process would ensure the policy \faIcon{file-alt} is only
sold to \faIcon{user} that has \defn{insurable interest}, i.e., the death of the
life insured \faIcon{walking} should make the policyholder \faIcon{user} quite worse
off.\footnote{Expressing differently, it is of interest to \faIcon{user} when
\faIcon{walking} is alive --- survival of \faIcon{walking} gives some benefits to
\faIcon{user}.}
\end{note}

\item Typically, the life insurance policy would expire immediately after
the sum insured is paid (if such payment exists), or when such payment is
impossible to be made (e.g., see \labelcref{it:deferral}).

\item The timing of the life insured's death \faIcon{skull} is very important to
\faIcon{building}, since:
\begin{enumerate}
\item it affects the timing of the death benefit payment
\faIcon{money-bill-wave}, hence its \emph{present value};

\item it also influences how much premiums \faIcon{dollar-sign} would be paid
(as the premium payment ceases immediately after the death).
\end{enumerate}
Both of them are crucial to the \emph{profitability} \faIcon{piggy-bank} of the
life insurance business!

\item Both \faIcon{money-bill-wave} and \faIcon{dollar-sign} depend on the
survival status of the life insured \faIcon{walking}, so we say that they are
\defn{life contingent} (contingent on the death/survival of \faIcon{walking}).
Their importance to the profitability of \faIcon{building} motivates the study
of \emph{life contingencies}.
\end{enumerate}

\subsection{Traditional Contracts for Life Insurance Business}
\label{subsect:traditional-contracts}
\begin{enumerate}
\item We introduce some traditional contracts here:
\begin{enumerate}
\item term life insurance \faIcon{clock}\textbar\faIcon{shield-alt}
\item whole life insurance \faIcon{infinity}\textbar\faIcon{shield-alt}
\item endowment insurance \faIcon{clock}\textbar\faIcon{shield-alt}\textbar\faIcon{piggy-bank} and pure endowment \faIcon{piggy-bank}
\item term life annuity \faIcon{clock}\textbar\faIcon{hand-holding-usd}
\item whole life annuity \faIcon{infinity}\textbar\faIcon{hand-holding-usd}
\end{enumerate}
\begin{note}
The last two contracts are \emph{life annuity} contracts, and are
not ``insurance'' contracts. Although \faIcon{building} is called ``life
insurance'' business, \faIcon{building} offers not only insurance policies, but
also policies ``related to life'', e.g.\ life annuity.
\end{note}

\item \defn{Term life insurance} is a life insurance contract
\faIcon{shield-alt} with a specified term \faIcon{clock}--- death benefit is
only payable when the life insured dies \emph{within} the term.

\item \defn{Whole life insurance} is a life insurance contract
\faIcon{shield-alt} with \emph{indefinite} term \faIcon{infinity}, i.e., death
benefit is always payable when the life insured dies.

\item \defn{Endowment insurance}\footnote{Endowment means ``funds invested for
the support and benefit of a person''. In this context, it refers to the
``savings element'' \faIcon{piggy-bank} (to benefit the policyholder
\faIcon{user}).} is a term life insurance
\faIcon{clock}\textbar\faIcon{shield-alt} with an extra survival benefit
\faIcon{money-bill-wave} (equal to the death benefit) payable if the life
insured is still alive at the end of the term (this can be seen as a ``savings
element'' \faIcon{piggy-bank}).

\begin{note}
\defn{Pure endowment} is an endowment insurance with the term life insurance
element taken away --- it only contains a survival benefit payable when the
life insured is still alive at the end of the term.
\end{note}

\item \defn{Term life annuity} is a term \faIcon{clock} annuity with periodic
payments \faIcon{hand-holding-usd} to \faIcon{user} (usually called
\defn{annuitant} in the context of life annuity), while \faIcon{user} is alive.

\item \defn{Whole life annuity} is a perpetuity with periodic payments
\faIcon{hand-holding-usd} to the annuitant \faIcon{user} while \faIcon{user} is
alive.

\item For all the contracts here, premiums \faIcon{dollar-sign} are payable.
Conventionally (for ``discrete'' case), they are paid \emph{at the
beginning} of each period, during a time interval specified in the contract.

\item On the other hand, the death/survival benefit is conventionally paid
\emph{at the end} of the period of death/last period of the term  (for
``discrete'' case).

\item \label{it:deferral} The contracts can be \defn{deferred}, in the sense
that the ``coverage'' only starts a specified time length (called the
\defn{deferral period}) after the purchasing time. For example, for a deferred
whole life insurance, if the death of life insured \faIcon{skull} occurs during
the deferral period, then no death benefit is payable (and the death benefit is
impossible to be made anymore, so the policy would expire).

\item Sometimes it possible for \faIcon{user} to \defn{lapse} or
\defn{surrender} the policy, i.e., \faIcon{user} can choose to stop paying
premiums starting at a certain time, and simultaneously the policy would expire
at that time.  In case of policy lapse/surrender, \faIcon{user} may receive
some refund (called \defn{cash value} or \defn{surrender value}).

\begin{note}
Sometimes the term ``lapse'' is used only when there is no such refund, and the
term ``surrender'' is used only when there is such refund.
\end{note}

\item For some term contracts, it is \defn{renewable}: the policyholder
\faIcon{user} can choose to \emph{renew} the contract at the end of the term,
for a time length specified in the contract, by continuing to pay premiums
(with possibly different amount from before) to receive a prolonged coverage.
\end{enumerate}
