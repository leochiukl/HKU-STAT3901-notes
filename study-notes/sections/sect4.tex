\section{Life Annuity}
\label{sect:life-annuity}
\begin{enumerate}
\item A life annuity can be seen as a combination of multiple pure endowment,
by considering each potential payment to the annuitant \faIcon{user} at time
\(t\) as a \(t\)-year pure endowment with survival benefit equal to that
payment amount issued to the annuitant \faIcon{user}. This provides one
approach for developments of life annuity, and forms the basis for the
general APV formula in \labelcref{it:gen-apv-fmla} \emph{for life annuity}.

\begin{tikzpicture}
\draw[-Latex] (0,0) -- (10,0) node[right]{Time};
\fill[] (0,0) circle [radius=0.05]
node[below] {0}
node[above] {\faIcon{hand-holding-usd}};
\fill[] (2,0) circle [radius=0.05] node[below] {1}
node[above] {\faIcon{hand-holding-usd}};
\fill[] (4,0) circle [radius=0.05]
node[below] {2}
node[above] {\faIcon{hand-holding-usd}};
\fill[] (6,0) circle [radius=0.05]
node[below] {3}
node[above] {\faIcon{hand-holding-usd}};
\fill[] (7,0) circle [radius=0.05]
node[red, above] {\faIcon{skull}};
\fill[] (8,0) circle [radius=0.05]
node[below] {4}
node[above, cross out, draw] {\faIcon{hand-holding-usd}};
\draw[-Latex, violet] (5,-0.7) -- (4.3,0.1)
node[pos=-0.3] {2-year pure endowment};
\draw[-Latex, violet] (9,-0.7) -- (8.3,0.1)
node[pos=-0.3] {4-year pure endowment};
\end{tikzpicture}

\item Another approach for the developments is based on STAT2902.
Recall that we have learned various \emph{annuities-certain} in STAT2902. By
making their terms to depend on the future lifetime (payments
\faIcon{hand-holding-usd} are only made when the annuitant \faIcon{user} is
alive), they become \emph{life annuities}.
\end{enumerate}
\subsection{Whole Life Annuity}
\begin{enumerate}
\item Like STAT2902, for \emph{discrete} life annuity, there is a
distinction between annuity-due and annuity-immediate (where payments are made
at the \emph{beginning} and \emph{end} of each period covered, respectively).
\begin{rmk}
\item It turns out that life annuity-\emph{due} is more frequently considered.
\item For life annuity-immediate, the end-of-period payment is \emph{not} made
for the period in which the life dies (since \emph{at the time of payment} the
life is \emph{not} alive).
\end{rmk}
\item Like \cref{sect:life-insurance}, we also have three ``frequencies'' for
life annuities: (i) continuous, (ii) annual, and (iii) \(1/m\)thly. (They
correspond to ``\(\ax*{}\)'', ``\(\ax{}\)'', \text{ and } ``\(\ax{}[(m)]\)'' in
STAT2902 respectively.)
\item \label{it:cts-wl-annuity-fmlas}
Continuous case (annual payment rate: 1):

\begin{tabular}{cccc}
\toprule
&p.v.r.v.&APV&variance\\
\midrule
expression&\(\ax*{\angl{T_x}}\) or \(\displaystyle\int_{0}^{\infty}e^{-\delta t}\indicset{T_x>t}\,dt\)
&\(\displaystyle \frac{1-\Ax*{x}}{\delta}\) or \(\displaystyle\int_{0}^{\infty}e^{-\delta t}\px[t]{x}\,dt\)
&\(\displaystyle \frac{\Ax*[][2]{x}-\qty(\Ax*{x})^2}{\delta^2}\)\\
notation&\(Y\)&\defn{\(\ax*{x}\)}&\(\vari{Y}\)\\
\bottomrule
\end{tabular}

\begin{warning}
For life annuities, the 2nd moment is \underline{not} the APV at double force
of interest.
\end{warning}

\begin{intuition}
For the p.v.r.v.\ expression \(\displaystyle\int_{0}^{\infty}e^{-\delta
t}\indicset{T_x>t}\,dt\), it can be understood intuitively as ``sum'' of
pure endowment with ``infinitesimal'' survival benefit
\faIcon{hand-holding-usd} ``\(dt\)''. Based on this
understanding, the APV formula \(\displaystyle\int_{0}^{\infty}e^{-\delta
t}\indicset{T_x>t}\,dt\) is ``summing up'' the expected present value of
\faIcon{hand-holding-usd} every \(t\)-year ``infinitesimal'' pure endowment:
\[
\underbrace{dte^{-\delta t}}_{\text{PV of payment}}\times\underbrace{\px[t]{x}}_{\text{prob.\ of surviving at time \(t\)}}.
\]
\end{intuition}

\begin{note}
Note that \(\displaystyle \int_{0}^{\infty}e^{-\delta t}\indicset{T_x>t}\,dt
=\int_{0}^{T_x}e^{-\delta t}\,dt
=\ax*{\angl{T_x}}\), so both p.v.r.v.\ expressions are indeed equivalent.
\end{note}


\begin{pf}
To get the first APV formula, note that
\[
\expv{\ax*{\angl{T_x}}}=\expv{\frac{1-e^{-\delta T_x}}{\delta}}
=\frac{1-\expv{e^{-\delta T_x}}}{\delta}
=\frac{1-\Ax*{x}}{\delta}.
\]
For the second APV formula, note that
\[
\expv{\int_{0}^{\infty}e^{-\delta t}\indicset{T_x>t}\,dt}
=\int_{0}^{\infty}\expv{e^{-\delta t}\indicset{T_x>t}}\,dt
=\int_{0}^{\infty}e^{-\delta t}\px[t]{x}\,dt
\]
where the first equality follows from Fubini's theorem.

Lastly, for the variance formula, we have
\[
\vari{\frac{1-e^{-\delta T_x}}{\delta}}=\frac{1}{\delta^2}
\underbrace{\vari{e^{-\delta T_x}}}_{\Ax*[][2]{x}-\qty(\Ax*{x})^2}
\]
\end{pf}

\item \label{it:ann-wl-annuity-fmlas}
Annual case (amount of each payment: 1):

\begin{tabular}{cccc}
\toprule
&p.v.r.v.&APV&variance\\
\midrule
expression&
\makecell{
due: \(\ax**{\angl{K_x+1}}\) or \(\displaystyle\sum_{k=0}^{\infty}v^k\indicset{T_x>k}\)\\
immediate: \(\ax{\angl{K_x}}\) or \(\displaystyle\sum_{k=1}^{\infty}v^k\indicset{T_x>k}\)
}
&\makecell{
due: \(\displaystyle \frac{1-\Ax{x}}{d}\) or \(\displaystyle\sum_{k=0}^{\infty}v^k\px[k]{x}\)\\
immediate: \(\displaystyle\sum_{k=1}^{\infty}v^k\px[k]{x}\)
}
&\makecell{due: \(\displaystyle \frac{\Ax[][2]{x}-\qty(\Ax{x})^2}{d^2}\)\\
immediate: omitted
}\\
notation&\(Y\)&
\makecell{due: \defn{\(\ax**{x}\)}\\
immediate: \defn{\(\ax{x}\)}}&\(\vari{Y}\)\\
\bottomrule
\end{tabular}

\item \label{it:1m-wl-annuity-fmlas}
\(1/m\)thly case (amount of each payment: \(1/m\); total amount of payments in
each year:  1):

\begin{tabular}{cccc}
\toprule
&p.v.r.v.&APV&variance\\
\midrule
expression&
\makecell{
due: \(\ax**{\angl{K_x^{(m)}+\frac{1}{m}}}[(m)]\) 
or \(\displaystyle\sum_{k=0}^{\infty}{\color{brown}\frac{1}{m}}v^{\frac{k}{m}}\indicset{T_x>\frac{k}{m}}\)\\
immediate: \(\ax{\angl{K_x^{(m)}}}[(m)]\) or \(\displaystyle\sum_{k=1}^{\infty}{\color{brown}\frac{1}{m}}v^{\frac{k}{m}}\indicset{T_x>\frac{k}{m}}\)
}
&\makecell{
due: \(\displaystyle \frac{1-\Ax{x}[(m)]}{d^{(m)}}\) 
or \(\displaystyle\sum_{k=0}^{\infty}{\color{brown}\frac{1}{m}}v^{\frac{k}{m}}\px[ \frac{k}{m}]{x}\)\\
immediate: \(\displaystyle\sum_{k=1}^{\infty}{\color{brown}\frac{1}{m}}v^{\frac{k}{m}}\px[ \frac{k}{m}]{x}\)
}
&\makecell{due: \(\displaystyle \frac{\Ax[][2]{x}[(m)]-\qty(\Ax{x}[(m)])^2}{(d^{(m)})^{2}}\)\\
immediate: omitted
}\\
notation&\(Y\)&
\makecell{due: \defn{\(\ax**{x}[(m)]\)}\\
immediate: \defn{\(\ax{x}[(m)]\)}}&\(\vari{Y}\)\\
\bottomrule
\end{tabular}
\item For discrete life annuity-immediate, we often use the following alternative
formula to compute APV instead:
\begin{proposition}
\label{prp:wl-due-immediate-fmla}
For any age \(x\),
\begin{enumerate}
\item \(\ax{x}=\ax**{x}-1\);
\item \(\ax{x}[(m)]=\ax**{x}[(m)]-1/m\).
\end{enumerate}
\end{proposition}
\begin{pf}
The result follows easily from considering the summation APV formulas.
\end{pf}
\item The APVs of whole life annuities of different frequencies can be ordered
as follows:
\[
\ax{x}\le\ax{x}[(m)]\le\ax*{x}\le\ax**{x}[(m)]\le\ax**{x},
\]
for any age \(x\) and \(m\in\N\).

\begin{intuition}
As we go from the life annuity on LHS to RHS one by one, all potential payments
(where some may get ``split'') ``shift'' \emph{earlier} (or do not ``move''),
so the present value gets higher (or is the same) \emph{always}, regardless of
when the life dies.
\end{intuition}

Also, \(\ax**{x}[(m)]\) decreases in \(m\) while \(\ax{x}[(m)]\) increases in
\(m\). This can be understood via similar intuition --- as \(m\) increases, all
potential payments ``shift'' later (for former) or earlier (for latter).
\end{enumerate}
\subsection{Temporary Life Annuity}
\begin{enumerate}
\item Consider an \(n\)-year temporary life annuity.
\item \label{it:cts-tmp-annuity-fmlas}
Continuous case (annual payment rate: 1):

\begin{tabular}{cccc}
\toprule
&p.v.r.v.&APV&variance\\
\midrule
expression&\(\ax*{\angl{T_x\wedge n}}\) or \(\displaystyle\int_{0}^{n}e^{-\delta t}\indicset{T_x>t}\,dt\)
&\(\displaystyle \frac{1-\Ax*{\endowxn}}{\delta}\) or \(\displaystyle\int_{0}^{n}e^{-\delta t}\px[t]{x}\,dt\)
&\(\displaystyle \frac{\Ax*[][2]{\endowxn}-\qty(\Ax*{\endowxn})^2}{\delta^2}\)\\
notation&\(Y\)&\defn{\(\ax*{\endowxn}\)}&\(\vari{Y}\)\\
\bottomrule
\end{tabular}
\item \label{it:ann-tmp-annuity-fmlas}
Annual case (amount of each payment: 1):

\begin{tabular}{cccc}
\toprule
&p.v.r.v.&APV&variance\\
\midrule
expression&
\makecell{
due: \(\ax**{\angl{\qty(K_x+1)\wedge n}}\) or \(\displaystyle\sum_{k=0}^{n-1}v^k\indicset{T_x>k}\)\\
immediate: \(\ax{\angl{K_x\wedge n}}\) or \(\displaystyle\sum_{k=1}^{n}v^k\indicset{T_x>k}\)
}
&\makecell{
due: \(\displaystyle \frac{1-\Ax{\endowxn}}{d}\) or \(\displaystyle\sum_{k=0}^{n}v^k\px[k]{x}\)\\
immediate: \(\displaystyle\sum_{k=1}^{n}v^k\px[k]{x}\)
}
&\makecell{due: \(\displaystyle \frac{\Ax[][2]{\endowxn}-\qty(\Ax{\endowxn})^2}{d^2}\)\\
immediate: omitted
}\\
notation&\(Y\)&
\makecell{due: \defn{\(\ax**{\endowxn}\)}\\
immediate: \defn{\(\ax{\endowxn}\)}}&\(\vari{Y}\)\\
\bottomrule
\end{tabular}

\item \label{it:1m-tmp-annuity-fmlas}
\(1/m\)thly case (amount of each payment: \(1/m\); total amount of payments in
each year: 1):

\begin{tabular}{cccc}
\toprule
&p.v.r.v.&APV&variance\\
\midrule
expression&
\makecell{
due: \(\ax**{\angl{\qty(K_x^{(m)} + \frac{1}{m})\wedge n}}\) or \\
\(\displaystyle\sum_{k=0}^{mn-1}{\color{brown}\frac{1}{m}}v^{\frac{k}{m}}\indicset{T_x> \frac{k}{m}}\)\\
immediate: \(\ax{\angl{K_x^{(m)}\wedge n}}\) or \\
\(\displaystyle\sum_{k=1}^{n}{\color{brown}\frac{1}{m}}v^{\frac{k}{m}}\indicset{T_x> \frac{k}{m}}\)
}
&\makecell{
due: \(\displaystyle \frac{1-\Ax{\endowxn}[(m)]}{d^{(m)}}\) 
or \(\displaystyle\sum_{k=0}^{mn-1}{\color{brown}\frac{1}{m}}v^{\frac{k}{m}}\px[\frac{k}{m}]{x}\)\\
immediate: \(\displaystyle\sum_{k=1}^{mn}{\color{brown}\frac{1}{m}}v^{\frac{k}{m}}\px[\frac{k}{m}]{x}\)
}
&\makecell{due: \(\displaystyle \frac{\Ax[][2]{\endowxn}[(m)]-\qty(\Ax{\endowxn}[(m)])^2}{\qty(d^{(m)})^{2}}\)\\
immediate: omitted
}\\
notation&\(Y\)&
\makecell{due: \defn{\(\ax**{\endowxn}^{(m)}\)}\\
immediate: \defn{\(\ax{\endowxn}^{(m)}\)}}&\(\vari{Y}\)\\
\bottomrule
\end{tabular}
\item Again for discrete life annuity-immediate, we usually use an alternative
formula for calculating APV:
\begin{proposition}
\label{prp:tmp-due-immediate-fmla}
For any age \(x\),
\begin{enumerate}
\item \(\ax{\endowxn}=\ax**{\endowxn}-1+\Ex[n]{x}\);
\item \(\ax{\endowxn}[(m)]=\ax**{\endowxn}[(m)]-1/m+(1/m)\Ex[n]{x}\).
\end{enumerate}
\end{proposition}
\begin{pf}
Similar to the proof for \cref{prp:wl-due-immediate-fmla}.
\end{pf}
\end{enumerate}
\subsection{Deferred Life Annuity}
\begin{enumerate}
\item Like \cref{subsect:defer-var-insurance}, we have different types of deferred life annuities:
\begin{itemize}
\item deferred whole life annuity
\item deferred temporary life annuity
\end{itemize}
\item Again like \cref{subsect:defer-var-insurance}, the APV formulas can be
developed using the ``actuarial discount factor'' intuition:

\begin{tikzpicture}
\draw[-Latex] (0,0) -- (10,0) node[right]{Time};
\fill[] (0,0) circle [radius=0.05]
node[below] {0}
node[above] {\((x)\)};
\fill[] (3,0) circle [radius=0.05]
node[below=0.05cm, font=\large] {\(u\)}
node[above] {\((x+u)\)};
\draw[very thick, decorate,decoration={calligraphic brace, amplitude=5pt, raise=15pt}] (3,0) -- (10,0)
node[midway, above=0.7cm]{\(\ax*{x+u}\)};
\draw[-Latex, color=ForestGreen] (3,0.6) to[bend right] (0,0.6);
\draw[-Latex, color=ForestGreen] (10,0.6) to[bend right] (0,0.9);
\node[] () at (2.5,1.3) {\(\times\Ex[u]{x}\)};
\draw[-Latex] (3,-0.7) -- (6,0.3);
\node[] () at (3,-1) {deferred coverage};
\end{tikzpicture}

One can also use the general APV formula in \labelcref{it:gen-apv-fmla} to
develop them.
\item Here we shall focus only on deferred \emph{whole life} annuity ---
similar developments can be done for deferred temporary life annuity.
\item \label{it:cts-defer-wl-annuity-fmlas}
Continuous case (annual payment rate: 1):

\begin{tabular}{ccc}
\toprule
&p.v.r.v.&APV \\
\midrule
expression&\(\ax*[u|]{\angl{T_x-u}}\indicset{T_x>u}\) or \(\displaystyle\int_{u}^{\infty}e^{-\delta t}\indicset{T_x>t}\,dt\)
&\makecell{\(\ax*{x}-\ax*{x:\angl{u}}\) or 
\(\Ex[u]{x}\ax*{x+u}\)\\
or \(\displaystyle\int_{u}^{\infty}e^{-\delta t}\px[t]{x}\,dt\)} \\
notation&\(Y\)&\defn{\(\ax*[u|]{x}\)}\\
\bottomrule
\end{tabular}

\begin{note}
We have \(\ax*[u|]{\angl{n}}=\ax*{\angl{u+n}}-\ax*{\angl{u}}\). (Similar for
``\(\ax{}\)'' and ``\(\ax{}[(m)]\)''.)
\end{note}

\item \label{it:ann-defer-wl-annuity-fmlas}
Annual case (amount of each payment: 1):

\begin{tabular}{ccc} \toprule
&p.v.r.v.&APV \\
\midrule
expression&
\makecell{
due: \(\ax**[u|]{\angl{K_x+1-u}}\indicset{T_x>u}\) or
\(\displaystyle\sum_{k=u}^{\infty}v^k\indicset{T_x>k}\) \\
immediate: \(\ax[u|]{\angl{K_x-u}}\indicset{T_x>u}\) or
\(\displaystyle\sum_{k=u+1}^{\infty}v^k\indicset{T_x>k}\)
}
&\makecell{due: \(\ax**{x}-\ax**{x:\angl{u}}\) or
\(\Ex[u]{x}\ax**{x+u}\)\\
or
\(\displaystyle\sum_{k=u}^{\infty}v^k\px[k]{x}\)\\
immediate: \(\ax{x}-\ax{x:\angl{u}}\) or
\(\Ex[u]{x}\ax{x+u}\) \\
or \(\displaystyle\sum_{k=u+1}^{\infty}v^k\px[k]{x}\)
}\\
notation&\(Y\)&
\makecell{
due: \defn{\(\ax**[u|]{x}\)}\\
immediate: \defn{\(\ax[u|]{x}\)}\\
}
\\
\bottomrule
\end{tabular}

\item \label{it:1m-defer-wl-annuity-fmlas}
\(1/m\)thly case (amount of each payment: \(1/m\); total amount of
payments in each year: 1):

\begin{tabular}{ccc}
\toprule
&p.v.r.v.&APV \\
\midrule
expression&
\makecell{
due: \(\ax**[u|]{\angl{K_x^{(m)}+\frac{1}{m}-u}}[(m)]\indicset{T_x>u}\) or
\(\displaystyle\sum_{k=u}^{\infty}\frac{1}{m}v^{\frac{k}{m}}\indicset{T_x>\frac{k}{m}}\) \\
immediate: \(\ax[u|]{\angl{K_x^{(m)}-u}}[(m)]\indicset{T_x>u}\) or
\(\displaystyle\sum_{k=u+1}^{\infty}\frac{1}{m}v^{\frac{k}{m}}\indicset{T_x>\frac{k}{m}}\)
}
&\makecell{due: \(\ax**{x}[(m)]-\ax**{x:\angl{u}}[(m)]\) or
\(\Ex[u]{x}\ax**{x+u}[(m)]\)\\
or
\(\displaystyle\sum_{k=u}^{\infty}\frac{1}{m}v^{\frac{k}{m}}\px[\frac{k}{m}]{x}\)\\
immediate: \(\ax{x}[(m)]-\ax{x:\angl{u}}[(m)]\) or
\(\Ex[u]{x}\ax{x+u}[(m)]\) \\
or \(\displaystyle\sum_{k=u+1}^{\infty}\frac{1}{m}v^{\frac{k}{m}}\px[\frac{k}{m}]{x}\)
}\\
notation&\(Y\)&
\makecell{
due: \defn{\(\ax**[u|]{x}[(m)]\)}\\
immediate: \defn{\(\ax[u|]{x}[(m)]\)}\\
}
\\
\bottomrule
\end{tabular}
\end{enumerate}
\subsection{Certain-And-Life/Guaranteed Annuity}
\begin{enumerate}
\item A key observation is that an \(n\)-year certain-and-life annuity is
indeed just a combination of an \(n\)-year annuity-certain and an \(n\)-year
deferred whole life annuity:

\begin{tikzpicture}
\draw[-Latex] (0,0) -- (10,0) node[right]{Time};
\fill[] (0,0) circle [radius=0.05]
node[below] {0}
node[above] {\((x)\)};
\fill[] (3,0) circle [radius=0.05]
node[below=0.05cm, font=\large] {\(n\)}
node[above, font=\small] {\((x+n)\) or \faIcon{skull}};
\draw[pen colour = violet, color=violet, very thick,
decorate,decoration={calligraphic brace, amplitude=5pt, raise=15pt}] 
(3,0) -- (10,0)
node[midway, above=0.7cm]{payable if alive};
\draw[-Latex, color=ForestGreen] (3,0.6) to[bend right] (0,0.6);
\draw[-Latex, color=ForestGreen] (10,0.6) to[bend right] (0,0.9);
\node[] () at (2.5,1.3) {\(\times\Ex[n]{x}\)};
\draw[very thick, decorate,decoration={mirror, calligraphic brace, amplitude=5pt, raise=15pt}] (0,0) -- (3,0)
node[midway, below=0.7cm]{\(\ax*{\angl{n}}\) (certain payments)};
\draw[color=violet, thick, decorate,decoration={brace, amplitude=5pt, raise=7pt}] (0,0.5) -- (0,1)
node[violet, midway, left=0.5cm] {\(\ax*[n|]{x}\)};
\end{tikzpicture}
\item Once we are aware of this, the developments for certain-and-life annuity become quite simple.
\item \label{it:cts-guar-annuity-fmlas}
Continuous case (annual payment rate: 1):

\begin{tabular}{ccc}
\toprule
&p.v.r.v.&APV \\
\midrule
expression&\(Y\) in \labelcref{it:cts-defer-wl-annuity-fmlas} + \(\ax*{\angl{n}}\)
&\(\ax*{\angl{n}}+\ax*[n|]{x}\)\\
notation&\(Y\)&\defn{\(\ax*{\overline{x:\angl{n}}}\)}\\
\bottomrule
\end{tabular}

\begin{note}
``\(\overline{x:\angl{n}}\)'' suggests that the payments continue until the
\emph{last} of \((x)\) and \(n\)-year term is ``gone''. More details will be
discussed in STAT3909.
\end{note}

\item \label{it:ann-guar-annuity-fmlas}
Annual case (amount of each payment: 1):

\begin{tabular}{ccc}
\toprule
&p.v.r.v.&APV \\
\midrule
expression&
\makecell{
due: \(Y\) in \labelcref{it:ann-defer-wl-annuity-fmlas} (due) + \(\ax**{\angl{n}}\)\\
immediate: \(Y\) in \labelcref{it:ann-defer-wl-annuity-fmlas} (immediate) + \(\ax{\angl{n}}\)
}
&\makecell{
due: \(\ax**{\angl{n}}+\ax**[n|]{x}\)\\
immediate: \(\ax{\angl{n}}+\ax[n|]{x}\)}\\
notation&\(Y\)&
\makecell{
due: \defn{\(\ax**{\overline{x:\angl{n}}}\)}\\
immediate \defn{\(\ax{\overline{x:\angl{n}}}\)}
}\\
\bottomrule
\end{tabular}

\item \label{it:1m-guar-annuity-fmlas}
\(1/m\)thly case (amount of each payment: \(1/m\); total amount of payments in each
year: 1):

\begin{tabular}{ccc}
\toprule
&p.v.r.v.&APV \\
\midrule
expression&
\makecell{
due: \(Y\) in \labelcref{it:1m-defer-wl-annuity-fmlas} (due) + \(\ax**{\angl{n}}[(m)]\)\\
immediate: \(Y\) in \labelcref{it:1m-defer-wl-annuity-fmlas} (immediate) + \(\ax{\angl{n}}[(m)]\)
}
&\makecell{
due: \(\ax**{\angl{n}}[(m)]+\ax**[n|]{x}[(m)]\)\\
immediate: \(\ax{\angl{n}}[(m)]+\ax[n|]{x}[(m)]\)}\\
notation&\(Y\)&
\makecell{
due: \defn{\(\ax**{\overline{x:\angl{n}}}[(m)]\)}\\
immediate \defn{\(\ax{\overline{x:\angl{n}}}[(m)]\)}
}\\
\bottomrule
\end{tabular}
\end{enumerate}

\subsection{Life Annuity With Variable Payments}
\begin{enumerate}
\item Like \cref{subsect:defer-var-insurance}, the general APV formula in
\labelcref{it:gen-apv-fmla} is applicable in the case where the life annuity
has varying payment amounts.
\end{enumerate}

