\section{Life Annuity}
\label{sect:life-annuity}
\begin{enumerate}
\item A life annuity can be seen as a combination of multiple pure endowment,
by considering each potential payment to the annuitant \faIcon{user} at time
\(t\) as a \(t\)-year pure endowment with survival benefit equal to that
payment amount issued to the annuitant \faIcon{user}. This provides one
approach for developments of life annuity, and forms the basis for the
general APV formula in \labelcref{it:gen-apv-fmla} \emph{for life annuity}.

\begin{tikzpicture}
\draw[-Latex] (0,0) -- (10,0) node[right]{Time};
\fill[] (0,0) circle [radius=0.05]
node[below] {0}
node[above] {\faIcon{hand-holding-usd}};
\fill[] (2,0) circle [radius=0.05] node[below] {1}
node[above] {\faIcon{hand-holding-usd}};
\fill[] (4,0) circle [radius=0.05]
node[below] {2}
node[above] {\faIcon{hand-holding-usd}};
\fill[] (6,0) circle [radius=0.05]
node[below] {3}
node[above] {\faIcon{hand-holding-usd}};
\fill[] (7,0) circle [radius=0.05]
node[red, above] {\faIcon{skull}};
\fill[] (8,0) circle [radius=0.05]
node[below] {4}
node[above, cross out, draw] {\faIcon{hand-holding-usd}};
\draw[-Latex, violet] (5,-0.7) -- (4.3,0.1)
node[pos=-0.3] {2-year pure endowment};
\draw[-Latex, violet] (9,-0.7) -- (8.3,0.1)
node[pos=-0.3] {4-year pure endowment};
\end{tikzpicture}

\item Another approach for the developments is based on STAT2902.
Recall that we have learned various \emph{annuities-certain} in STAT2902. By
making their terms to depend on the future lifetime (payments
\faIcon{hand-holding-usd} are only made when the annuitant \faIcon{user} is
alive), they become \emph{life annuities}.

\item Like \labelcref{it:insurance-relationship}, the following shows the
relationship between different kinds of life annuities and pure endowment (in a
``construction'' approach):

\begin{tikzpicture}
\node[draw, brown] (wl) at (0,0) {whole life};
\node[draw, brown] (pe) at (5,0) {pure endowment};
\node[draw, violet] (dl) at (4,-2) {deferred contracts};
\node[draw, violet] (cal) at (-3,-3) {certain-and-life};
\node[draw, gray] (cer) at (-3,-2) {annuity-certain};
\node[gray] () at (-3,-1.5) {(STAT2902)};
\node[draw, violet] (tl) at (1,-2) {term life};
\node[draw, purple] (cmp) at (1.5,-4) {complete};
\node[draw, purple] (app) at (4.5,-4) {apportionable};
\draw[thick, decorate,decoration={calligraphic brace, amplitude=10pt, raise=10pt, mirror}] (-1,-3) -- (7,-3)
node[purple, midway, below=1.5cm, draw] (vb) {variable benefits (different kinds)};
\draw[-Latex, blue] (wl) -- (tl);
\draw[-Latex, blue] (pe) -- (tl);
\draw[-Latex, orange] (wl) -- (dl);
\draw[-Latex, orange] (tl) -- (dl);
\draw[-Latex, orange] (pe) to[bend right] (dl);
\draw[-Latex, magenta] (dl) -- (cal);
\draw[-Latex, magenta] (cer) -- (cal);
\node[brown] () at (9,0) {basic building blocks};
\node[violet] () at (9,-2.5) {``constructed'' contracts};
\node[purple] () at (9,-4.5) {variants};
\end{tikzpicture}

\begin{note}
The complete and apportionable variants are not inside SOA exam FAM syllabus
currently.
\end{note}

\end{enumerate}
\subsection{Whole Life Annuity}
\begin{enumerate}
\item Like STAT2902, for \emph{discrete} life annuity, there is a
distinction between annuity-due and annuity-immediate (where payments are made
at the \emph{beginning} and \emph{end} of each period covered, respectively).
\begin{remark}
\item It turns out that life annuity-\emph{due} is more frequently considered.
\item For life annuity-immediate, the end-of-period payment is \emph{not} made
for the period in which the life dies (since \emph{at the time of payment} the
life is \emph{not} alive).
\end{remark}
\item Like \cref{sect:life-insurance}, we also have three ``frequencies'' for
life annuities: (i) continuous, (ii) annual, and (iii) \(1/m\)thly. (They
correspond to ``\(\ax*{}\)'', ``\(\ax{}\)'', \text{ and } ``\(\ax{}[(m)]\)'' in
STAT2902 respectively.)
\item \label{it:cts-wl-annuity-fmlas}
Continuous case (annual payment rate: 1):

\begin{tabular}{cccc}
\toprule
&p.v.r.v.&APV&variance\\
\midrule
expression&\(\ax*{\angl{T_x}}\) or \(\displaystyle\int_{0}^{\infty}e^{-\delta t}\indicset{T_x>t}\,dt\)
&\(\displaystyle \frac{1-\Ax*{x}}{\delta}\) or \(\displaystyle\int_{0}^{\infty}e^{-\delta t}\px[t]{x}\,dt\)
&\(\displaystyle \frac{\Ax*[][2]{x}-\qty(\Ax*{x})^2}{\delta^2}\)\\
notation&\(Y\)&\defn{\(\ax*{x}\)}&\(\vari{Y}\)\\
\bottomrule
\end{tabular}

\begin{warning}
For life annuities, the 2nd moment is \underline{not} the APV at double force
of interest.
\end{warning}

\begin{intuition}
For the p.v.r.v.\ expression \(\displaystyle\int_{0}^{\infty}e^{-\delta
t}\indicset{T_x>t}\,dt\), it can be understood intuitively as ``sum'' of
pure endowment with ``infinitesimal'' survival benefit
\faIcon{hand-holding-usd} ``\(dt\)''. Based on this
understanding, the APV formula \(\displaystyle\int_{0}^{\infty}e^{-\delta
t}\indicset{T_x>t}\,dt\) is ``summing up'' the expected present value of
\faIcon{hand-holding-usd} every \(t\)-year ``infinitesimal'' pure endowment:
\[
\underbrace{dte^{-\delta t}}_{\text{PV of payment}}\times\underbrace{\px[t]{x}}_{\text{prob.\ of surviving at time \(t\)}}.
\]
\end{intuition}

\begin{note}
Note that \(\displaystyle \int_{0}^{\infty}e^{-\delta t}\indicset{T_x>t}\,dt
=\int_{0}^{T_x}e^{-\delta t}\,dt
=\ax*{\angl{T_x}}\), so both p.v.r.v.\ expressions are indeed equivalent.
\end{note}


\begin{pf}
To get the first APV formula, note that
\[
\expv{\ax*{\angl{T_x}}}=\expv{\frac{1-e^{-\delta T_x}}{\delta}}
=\frac{1-\expv{e^{-\delta T_x}}}{\delta}
=\frac{1-\Ax*{x}}{\delta}.
\]
For the second APV formula, note that
\[
\expv{\int_{0}^{\infty}e^{-\delta t}\indicset{T_x>t}\,dt}
=\int_{0}^{\infty}\expv{e^{-\delta t}\indicset{T_x>t}}\,dt
=\int_{0}^{\infty}e^{-\delta t}\px[t]{x}\,dt
\]
where the first equality follows from Fubini's theorem.

Lastly, for the variance formula, we have
\[
\vari{\frac{1-e^{-\delta T_x}}{\delta}}=\frac{1}{\delta^2}
\underbrace{\vari{e^{-\delta T_x}}}_{\Ax*[][2]{x}-\qty(\Ax*{x})^2}
\]
\end{pf}

\item \label{it:ann-wl-annuity-fmlas}
Annual case (amount of each payment: 1):

\begin{tabular}{cccc}
\toprule
&p.v.r.v.&APV&variance\\
\midrule
expression&
\makecell{
due: \(\ax**{\angl{K_x+1}}\) or \(\displaystyle\sum_{k=0}^{\infty}v^k\indicset{T_x>k}\)\\
immediate: \(\ax{\angl{K_x}}\) or \(\displaystyle\sum_{k=1}^{\infty}v^k\indicset{T_x>k}\)
}
&\makecell{
due: \(\displaystyle \frac{1-\Ax{x}}{d}\) or \(\displaystyle\sum_{k=0}^{\infty}v^k\px[k]{x}\)\\
immediate: \(\displaystyle\sum_{k=1}^{\infty}v^k\px[k]{x}\)
}
&\makecell{due: \(\displaystyle \frac{\Ax[][2]{x}-\qty(\Ax{x})^2}{d^2}\)\\
immediate: same as due
}\\
notation&\(Y\)&
\makecell{due: \defn{\(\ax**{x}\)}\\
immediate: \defn{\(\ax{x}\)}}&\(\vari{Y}\)\\
\bottomrule
\end{tabular}
\begin{note}
The variance for the immediate p.v.r.v.\ equals the one for the due p.v.r.v.\
since the p.v.r.v.'s just differ by a constant (\(v^{0}=1\)).
\end{note}


\item \label{it:1m-wl-annuity-fmlas}
\(1/m\)thly case (amount of each payment: \(1/m\); total amount of payments in
each year:  1):

\begin{tabular}{cccc}
\toprule
&p.v.r.v.&APV&variance\\
\midrule
expression&
\makecell{
due: \(\ax**{\angl{K_x^{(m)}+\frac{1}{m}}}[(m)]\) or \\
\(\displaystyle\sum_{k=0}^{\infty}{\color{brown}\frac{1}{m}}v^{\frac{k}{m}}\indicset{T_x>\frac{k}{m}}\)\\
immediate: \(\ax{\angl{K_x^{(m)}}}[(m)]\) or \\
\(\displaystyle\sum_{k=1}^{\infty}{\color{brown}\frac{1}{m}}v^{\frac{k}{m}}\indicset{T_x>\frac{k}{m}}\)
}
&\makecell{
due: \(\displaystyle \frac{1-\Ax{x}[(m)]}{d^{(m)}}\) 
or \(\displaystyle\sum_{k=0}^{\infty}{\color{brown}\frac{1}{m}}v^{\frac{k}{m}}\px[ \frac{k}{m}]{x}\)\\
immediate: \(\displaystyle\sum_{k=1}^{\infty}{\color{brown}\frac{1}{m}}v^{\frac{k}{m}}\px[ \frac{k}{m}]{x}\)
}
&\makecell{due: \(\displaystyle \frac{\Ax[][2]{x}[(m)]-\qty(\Ax{x}[(m)])^2}{(d^{(m)})^{2}}\)\\
immediate: same as due
}\\
notation&\(Y\)&
\makecell{due: \defn{\(\ax**{x}[(m)]\)}\\
immediate: \defn{\(\ax{x}[(m)]\)}}&\(\vari{Y}\)\\
\bottomrule
\end{tabular}
\item For discrete life annuity-immediate, we often use the following alternative
formulas to compute APV instead:
\begin{proposition}
\label{prp:wl-due-immediate-fmlas}
For any age \(x\),
\begin{enumerate}
\item \(\ax{x}=\ax**{x}-1\);
\item \(\ax{x}[(m)]=\ax**{x}[(m)]-1/m\).
\end{enumerate}
\end{proposition}
\begin{pf}
The result follows easily from considering the summation APV formulas.
\end{pf}
\item The APVs of whole life annuities of different frequencies can be ordered
as follows:
\[
\ax{x}\le\ax{x}[(m)]\le\ax*{x}\le\ax**{x}[(m)]\le\ax**{x},
\]
for any age \(x\) and \(m\in\N\).

\begin{intuition}
As we go from the life annuity on LHS to RHS one by one, all potential payments
(where some may get ``split'') ``shift'' \emph{earlier} (or do not ``move''),
so the present value gets higher (or at least does not decrease) \emph{always},
regardless of when the life dies.
\end{intuition}

Also, \(\ax**{x}[(m)]\) decreases in \(m\) while \(\ax{x}[(m)]\) increases in
\(m\). This can be understood via similar intuition --- as \(m\) increases, all
potential payments ``shift'' later (for former) or earlier (for latter).
\end{enumerate}
\subsection{Temporary Life Annuity}
\begin{enumerate}
\item Consider an \(n\)-year temporary life annuity.
\item \label{it:cts-tmp-annuity-fmlas}
Continuous case (annual payment rate: 1):

\begin{tabular}{cccc}
\toprule
&p.v.r.v.&APV&variance\\
\midrule
expression&\(\ax*{\angl{T_x\wedge n}}\) or \(\displaystyle\int_{0}^{n}e^{-\delta t}\indicset{T_x>t}\,dt\)
&\makecell{\(\displaystyle \frac{1-\Ax*{\endowxn}}{\delta}\) or 
\(\displaystyle \ax*{x}-\Ex[n]{x}\ax*{x+n}\)\\
or \(\displaystyle\int_{0}^{n}e^{-\delta t}\px[t]{x}\,dt\)}
&\(\displaystyle \frac{\Ax*[][2]{\endowxn}-\qty(\Ax*{\endowxn})^2}{\delta^2}\)\\
notation&\(Y\)&\defn{\(\ax*{\endowxn}\)}&\(\vari{Y}\)\\
\bottomrule
\end{tabular}
\item \label{it:ann-tmp-annuity-fmlas}
Annual case (amount of each payment: 1):

\begin{tabular}{cccc}
\toprule
&p.v.r.v.&APV&variance\\
\midrule
expression&
\makecell{
due: \(\ax**{\angl{\qty(K_x+1)\wedge n}}\) or \(\displaystyle\sum_{k=0}^{n-1}v^k\indicset{T_x>k}\)\\
immediate: \(\ax{\angl{K_x\wedge n}}\) or \(\displaystyle\sum_{k=1}^{n}v^k\indicset{T_x>k}\)
}
&\makecell{
due: \(\displaystyle \frac{1-\Ax{\endowxn}}{d}\) or \(\ax**{x}-\Ex[n]{x}\ax**{x+n}\) \\
or \(\displaystyle\sum_{k=0}^{n}v^k\px[k]{x}\)\\
immediate: \(\displaystyle\sum_{k=1}^{n}v^k\px[k]{x}\) \\
or \(\ax{x}-\Ex[n]{x}\ax{x+n}\) \\
}
&\makecell{due: \(\displaystyle \frac{\Ax[][2]{\endowxn}-\qty(\Ax{\endowxn})^2}{d^2}\)\\
immediate: omitted
}\\
notation&\(Y\)&
\makecell{due: \defn{\(\ax**{\endowxn}\)}\\
immediate: \defn{\(\ax{\endowxn}\)}}&\(\vari{Y}\)\\
\bottomrule
\end{tabular}

\item \label{it:1m-tmp-annuity-fmlas}
\(1/m\)thly case (amount of each payment: \(1/m\); total amount of payments in
each year: 1):

\begin{tabular}{cccc}
\toprule
&p.v.r.v.&APV&variance\\
\midrule
expression&
\makecell{
due: \(\ax**{\angl{\qty(K_x^{(m)} + \frac{1}{m})\wedge n}}\) or \\
\(\displaystyle\sum_{k=0}^{mn-1}{\color{brown}\frac{1}{m}}v^{\frac{k}{m}}\indicset{T_x> \frac{k}{m}}\)\\
immediate: \(\ax{\angl{K_x^{(m)}\wedge n}}\) or \\
\(\displaystyle\sum_{k=1}^{mn}{\color{brown}\frac{1}{m}}v^{\frac{k}{m}}\indicset{T_x> \frac{k}{m}}\)
}
&\makecell{
due: \(\displaystyle \frac{1-\Ax{\endowxn}[(m)]}{d^{(m)}}\)
or \(\ax**{x}[(m)]-\Ex[n]{x}\ax**{x+n}[(m)]\) \\
or \(\displaystyle\sum_{k=0}^{mn-1}{\color{brown}\frac{1}{m}}v^{\frac{k}{m}}\px[\frac{k}{m}]{x}\)\\
immediate: \(\displaystyle\sum_{k=1}^{mn}{\color{brown}\frac{1}{m}}v^{\frac{k}{m}}\px[\frac{k}{m}]{x}\) \\
or \(\ax**{x}[(m)]-\Ex[n]{x}\ax**{x+n}[(m)]\) \\
}
&\makecell{due: \(\displaystyle \frac{\Ax[][2]{\endowxn}[(m)]-\qty(\Ax{\endowxn}[(m)])^2}{\qty(d^{(m)})^{2}}\)\\
immediate: omitted
}\\
notation&\(Y\)&
\makecell{due: \defn{\(\ax**{\endowxn}^{(m)}\)}\\
immediate: \defn{\(\ax{\endowxn}^{(m)}\)}}&\(\vari{Y}\)\\
\bottomrule
\end{tabular}
\item We can similarly use the ``actuarial discounting'' intuition to develop
the following formula (analogous to the term life insurance case):

\begin{tikzpicture}
\draw[-Latex] (0,0) -- (10,0) node[right]{Time};
\fill[] (0,0) circle [radius=0.05]
node[below] {0}
node[above] {\((x)\)};
\fill[] (3,0) circle [radius=0.05]
node[below] {\(n\)};
\draw[very thick, decorate,decoration={mirror, calligraphic brace, amplitude=5pt, raise=15pt}] (0,0) -- (3,0)
node[midway, below=0.7cm]{\(\Ax*{\termxn}\)};
\draw[pen colour=red, very thick, decorate,decoration={calligraphic brace, amplitude=5pt, raise=25pt}] (3,0) -- (10,0)
node[red, midway, above=1cm]{\(\Ax*{x+n}\)};
\draw[pen colour=violet, very thick, decorate,decoration={calligraphic brace, amplitude=5pt, raise=15pt}] (0,0) -- (10,0)
node[violet, pos=0.2, above=0.7cm]{\(\Ax*{x}\)};
\draw[pen colour=red, red, thick, decorate,decoration={brace, amplitude=5pt, raise=5pt}] (0,1.5) -- (0,2)
node[midway, left=0.5cm] {take away};
\draw[-Latex, color=ForestGreen] (3,1) to[bend right] (0,1.6);
\draw[-Latex, color=ForestGreen] (10,1) to[bend right] (0,1.9);
\node[] () at (2.5,2) {\(\times\Ex[n]{x}\)};
\end{tikzpicture}

\item Again for discrete life annuity-immediate, we usually use an alternative
formula for calculating APV:
\begin{proposition}
\label{prp:tmp-due-immediate-fmla}
For any age \(x\),
\begin{enumerate}
\item \(\ax{\endowxn}=\ax**{\endowxn}-1+\Ex[n]{x}\);
\item \(\ax{\endowxn}[(m)]=\ax**{\endowxn}[(m)]-1/m+(1/m)\Ex[n]{x}\).
\end{enumerate}
\end{proposition}
\begin{pf}
Similar to the proof for \cref{prp:wl-due-immediate-fmlas}.
\end{pf}
\end{enumerate}
\subsection{Deferred Life Annuity}
\begin{enumerate}
\item Like \cref{subsect:defer-insurance}, we have different types of deferred life annuities:
\begin{itemize}
\item deferred whole life annuity
\item deferred temporary life annuity
\end{itemize}
\item Again like \cref{subsect:defer-insurance}, the APV formulas can be
developed using the ``actuarial discount factor'' intuition:

\begin{tikzpicture}
\draw[-Latex] (0,0) -- (10,0) node[right]{Time};
\fill[] (0,0) circle [radius=0.05]
node[below] {0}
node[above] {\((x)\)};
\fill[] (3,0) circle [radius=0.05]
node[below=0.05cm, font=\large] {\(u\)}
node[above] {\((x+u)\)};
\draw[very thick, decorate,decoration={calligraphic brace, amplitude=5pt, raise=15pt}] (3,0) -- (10,0)
node[midway, above=0.7cm]{\(\ax*{x+u}\)};
\draw[-Latex, color=ForestGreen] (3,0.6) to[bend right] (0,0.6);
\draw[-Latex, color=ForestGreen] (10,0.6) to[bend right] (0,0.9);
\node[] () at (2.5,1.3) {\(\times\Ex[u]{x}\)};
\draw[-Latex] (3,-0.7) -- (6,0.3);
\node[] () at (3,-1) {deferred coverage};
\end{tikzpicture}

One can also use the general APV formula in \labelcref{it:gen-apv-fmla} to
develop them.
\item Here we shall focus only on deferred \emph{whole life} annuity ---
similar developments can be done for deferred temporary life annuity.
\item \label{it:cts-defer-wl-annuity-fmlas}
Continuous case (annual payment rate: 1):

\begin{tabular}{ccc}
\toprule
&p.v.r.v.&APV \\
\midrule
expression&\(\ax*[u|]{\angl{T_x-u}}\indicset{T_x>u}\) or \(\displaystyle\int_{u}^{\infty}e^{-\delta t}\indicset{T_x>t}\,dt\)
&\makecell{\(\ax*{x}-\ax*{x:\angl{u}}\) or 
\(\Ex[u]{x}\ax*{x+u}\)\\
or \(\displaystyle\int_{u}^{\infty}e^{-\delta t}\px[t]{x}\,dt\)} \\
notation&\(Y\)&\defn{\(\ax*[u|]{x}\)}\\
\bottomrule
\end{tabular}

\begin{note}
We have \(\ax*[u|]{\angl{n}}=\ax*{\angl{u+n}}-\ax*{\angl{u}}\). (Similar for
``\(\ax{}\)'' and ``\(\ax{}[(m)]\)''.)
\end{note}

\item \label{it:ann-defer-wl-annuity-fmlas}
Annual case (amount of each payment: 1):

\begin{tabular}{ccc} \toprule
&p.v.r.v.&APV \\
\midrule
expression&
\makecell{
due: \(\ax**[u|]{\angl{K_x+1-u}}\indicset{T_x>u}\) or
\(\displaystyle\sum_{k=u}^{\infty}v^k\indicset{T_x>k}\) \\
immediate: \(\ax[u|]{\angl{K_x-u}}\indicset{T_x>u}\) or
\(\displaystyle\sum_{k=u+1}^{\infty}v^k\indicset{T_x>k}\)
}
&\makecell{due: \(\ax**{x}-\ax**{x:\angl{u}}\) or
\(\Ex[u]{x}\ax**{x+u}\)\\
or
\(\displaystyle\sum_{k=u}^{\infty}v^k\px[k]{x}\)\\
immediate: \(\ax{x}-\ax{x:\angl{u}}\) or
\(\Ex[u]{x}\ax{x+u}\) \\
or \(\displaystyle\sum_{k=u+1}^{\infty}v^k\px[k]{x}\)
}\\
notation&\(Y\)&
\makecell{
due: \defn{\(\ax**[u|]{x}\)}\\
immediate: \defn{\(\ax[u|]{x}\)}\\
}
\\
\bottomrule
\end{tabular}

\item \label{it:1m-defer-wl-annuity-fmlas}
\(1/m\)thly case (amount of each payment: \(1/m\); total amount of
payments in each year: 1):

\begin{tabular}{ccc}
\toprule
&p.v.r.v.&APV \\
\midrule
expression&
\makecell{
due: \(\ax**[u|]{\angl{K_x^{(m)}+\frac{1}{m}-u}}[(m)]\indicset{T_x>u}\) or
\(\displaystyle\sum_{k=u}^{\infty}\frac{1}{m}v^{\frac{k}{m}}\indicset{T_x>\frac{k}{m}}\) \\
immediate: \(\ax[u|]{\angl{K_x^{(m)}-u}}[(m)]\indicset{T_x>u}\) or
\(\displaystyle\sum_{k=u+1}^{\infty}\frac{1}{m}v^{\frac{k}{m}}\indicset{T_x>\frac{k}{m}}\)
}
&\makecell{due: \(\ax**{x}[(m)]-\ax**{x:\angl{u}}[(m)]\) or
\(\Ex[u]{x}\ax**{x+u}[(m)]\)\\
or
\(\displaystyle\sum_{k=u}^{\infty}\frac{1}{m}v^{\frac{k}{m}}\px[\frac{k}{m}]{x}\)\\
immediate: \(\ax{x}[(m)]-\ax{x:\angl{u}}[(m)]\) or
\(\Ex[u]{x}\ax{x+u}[(m)]\) \\
or \(\displaystyle\sum_{k=u+1}^{\infty}\frac{1}{m}v^{\frac{k}{m}}\px[\frac{k}{m}]{x}\)
}\\
notation&\(Y\)&
\makecell{
due: \defn{\(\ax**[u|]{x}[(m)]\)}\\
immediate: \defn{\(\ax[u|]{x}[(m)]\)}\\
}
\\
\bottomrule
\end{tabular}
\end{enumerate}
\subsection{Certain-And-Life/Guaranteed Annuity}
\begin{enumerate}
\item A key observation is that an \(n\)-year certain-and-life annuity is
indeed just a combination of an \(n\)-year annuity-certain and an \(n\)-year
deferred whole life annuity:

\begin{tikzpicture}
\draw[-Latex] (0,0) -- (10,0) node[right]{Time};
\fill[] (0,0) circle [radius=0.05]
node[below] {0}
node[above] {\((x)\)};
\fill[] (3,0) circle [radius=0.05]
node[below=0.05cm, font=\large] {\(n\)}
node[above, font=\small] {\((x+n)\) or \faIcon{skull}};
\draw[pen colour = violet, color=violet, very thick,
decorate,decoration={calligraphic brace, amplitude=5pt, raise=15pt}] 
(3,0) -- (10,0)
node[midway, above=0.7cm]{payable if alive};
\draw[-Latex, color=ForestGreen] (3,0.6) to[bend right] (0,0.6);
\draw[-Latex, color=ForestGreen] (10,0.6) to[bend right] (0,0.9);
\node[] () at (2.5,1.3) {\(\times\Ex[n]{x}\)};
\draw[very thick, decorate,decoration={mirror, calligraphic brace, amplitude=5pt, raise=15pt}] (0,0) -- (3,0)
node[midway, below=0.7cm]{\(\ax*{\angl{n}}\) (certain payments)};
\draw[color=violet, thick, decorate,decoration={brace, amplitude=5pt, raise=7pt}] (0,0.5) -- (0,1)
node[violet, midway, left=0.5cm] {\(\ax*[n|]{x}\)};
\end{tikzpicture}
\item Once we are aware of this, the developments for certain-and-life annuity become quite simple.
\item \label{it:cts-guar-annuity-fmlas}
Continuous case (annual payment rate: 1):

\begin{tabular}{ccc}
\toprule
&p.v.r.v.&APV \\
\midrule
expression&\(Y\) in \labelcref{it:cts-defer-wl-annuity-fmlas} + \(\ax*{\angl{n}}\)
&\(\ax*{\angl{n}}+\ax*[n|]{x}\)\\
notation&\(Y\)&\defn{\(\ax*{\overline{x:\angl{n}}}\)}\\
\bottomrule
\end{tabular}

\begin{note}
``\(\overline{x:\angl{n}}\)'' suggests that the payments continue until the
\emph{last} of \((x)\) and \(n\)-year term is ``gone''. More details will be
discussed in STAT3909.
\end{note}

\item \label{it:ann-guar-annuity-fmlas}
Annual case (amount of each payment: 1):

\begin{tabular}{ccc}
\toprule
&p.v.r.v.&APV \\
\midrule
expression&
\makecell{
due: \(Y\) in \labelcref{it:ann-defer-wl-annuity-fmlas} (due) + \(\ax**{\angl{n}}\)\\
immediate: \(Y\) in \labelcref{it:ann-defer-wl-annuity-fmlas} (immediate) + \(\ax{\angl{n}}\)
}
&\makecell{
due: \(\ax**{\angl{n}}+\ax**[n|]{x}\)\\
immediate: \(\ax{\angl{n}}+\ax[n|]{x}\)}\\
notation&\(Y\)&
\makecell{
due: \defn{\(\ax**{\overline{x:\angl{n}}}\)}\\
immediate \defn{\(\ax{\overline{x:\angl{n}}}\)}
}\\
\bottomrule
\end{tabular}

\item \label{it:1m-guar-annuity-fmlas}
\(1/m\)thly case (amount of each payment: \(1/m\); total amount of payments in each
year: 1):

\begin{tabular}{ccc}
\toprule
&p.v.r.v.&APV \\
\midrule
expression&
\makecell{
due: \(Y\) in \labelcref{it:1m-defer-wl-annuity-fmlas} (due) + \(\ax**{\angl{n}}[(m)]\)\\
immediate: \(Y\) in \labelcref{it:1m-defer-wl-annuity-fmlas} (immediate) + \(\ax{\angl{n}}[(m)]\)
}
&\makecell{
due: \(\ax**{\angl{n}}[(m)]+\ax**[n|]{x}[(m)]\)\\
immediate: \(\ax{\angl{n}}[(m)]+\ax[n|]{x}[(m)]\)}\\
notation&\(Y\)&
\makecell{
due: \defn{\(\ax**{\overline{x:\angl{n}}}[(m)]\)}\\
immediate \defn{\(\ax{\overline{x:\angl{n}}}[(m)]\)}
}\\
\bottomrule
\end{tabular}
\end{enumerate}

\subsection{Complete and Apportionable Life Annuities}
\begin{note}
This topic is not currently inside SOA exam FAM syllabus.
\end{note}
\begin{enumerate}
\item Consider the following situation:

\begin{tikzpicture}
\node[brown] () at (-1.5,0) {Life A \faIcon{user}};
\draw[-Latex] (0,0) -- (10,0) node[right]{Time};
\fill[] (0,0) circle [radius=0.05]
node[below] {0}
node[above] {\faIcon{hand-holding-usd}};
\fill[] (2,0) circle [radius=0.05] node[below] {1}
node[above] {\faIcon{hand-holding-usd}};
\fill[] (4,0) circle [radius=0.05]
node[below] {2}
node[above] {\faIcon{hand-holding-usd}};
\fill[] (6,0) circle [radius=0.05]
node[below] {3}
node[above] {\faIcon{hand-holding-usd}};
\fill[] (6.3,0) circle [radius=0.05]
node[red, above] {\faIcon{skull}};
\fill[] (8,0) circle [radius=0.05]
node[below] {4}
node[above, cross out, draw] {\faIcon{hand-holding-usd}};
\end{tikzpicture}

\begin{tikzpicture}
\node[violet] () at (-1.5,0) {Life B \faIcon{user}};
\draw[-Latex] (0,0) -- (10,0) node[right]{Time};
\fill[] (0,0) circle [radius=0.05]
node[below] {0}
node[above] {\faIcon{hand-holding-usd}};
\fill[] (2,0) circle [radius=0.05] 
node[below] {1}
node[above] {\faIcon{hand-holding-usd}};
\fill[] (4,0) circle [radius=0.05]
node[below] {2}
node[above] {\faIcon{hand-holding-usd}};
\fill[] (6,0) circle [radius=0.05]
node[below] {3}
node[above] {\faIcon{hand-holding-usd}};
\fill[] (7.6,0) circle [radius=0.05]
node[red, above] {\faIcon{skull}};
\fill[] (8,0) circle [radius=0.05]
node[below] {4}
node[above, cross out, draw] {\faIcon{hand-holding-usd}};
\end{tikzpicture}

For both lives A {\color{brown}\faIcon{user}} and B
{\color{violet}\faIcon{user}}, they do not receive the benefit payment
\faIcon{hand-holding-usd} at time 4. But life B {\color{violet}\faIcon{user}}
survives most of the last period, while life A  {\color{brown}\faIcon{user}}
only survives a little of the last period! So this may cause some
``unfairness''.
\begin{note}
This ``unfairness'' only arises for discrete cases.
\end{note}

\item To address this issue, the ``complete'' and ``apportionable'' variants
are developed, for life annuity-immediate and life annuity-due respectively.

\item \label{it:convert-to-stream}
The common idea in the two variants is to convert the potential
``discrete'' payments \faIcon{hand-holding-usd} in life annuity-due/immediate
into \emph{continuous} streams of payments --- the streams would cease upon
death immediately.

\begin{tikzpicture}
\node[brown] () at (-1.25,0) {Life A \faIcon{user}};
\draw[-Latex] (0,0) -- (10,0) node[right]{Time};
\fill[] (0,0) circle [radius=0.05]
node[below] {0}
node[above] {\faIcon{hand-holding-usd}};
\fill[] (2,0) circle [radius=0.05] 
node[below] {1}
node[above] {\faIcon{hand-holding-usd}};
\fill[] (4,0) circle [radius=0.05]
node[below] {2}
node[above] {\faIcon{hand-holding-usd}};
\fill[] (6,0) circle [radius=0.05]
node[below] {3}
node[above] {\faIcon{hand-holding-usd}};
\fill[] (6.3,0) circle [radius=0.05]
node[red, above] {\faIcon{skull}};
\fill[] (8,0) circle [radius=0.05]
node[below] {4}
node[above, cross out, draw] {\faIcon{hand-holding-usd}};
\end{tikzpicture}

\begin{tikzpicture}
\draw[-Stealth, line width=1mm, ForestGreen] (-2,0) -- (-0.5,0);
\draw[-Latex] (0,0) -- (10,0) node[right]{Time};
\fill[] (0,0) circle [radius=0.05]
node[below] {0};
\fill[] (2,0) circle [radius=0.05]
node[below] {1};
\fill[] (4,0) circle [radius=0.05]
node[below] {2};
\fill[] (6,0) circle [radius=0.05]
node[below] {3};
\fill[] (6.3,0) circle [radius=0.05]
node[red, above] {\faIcon{skull}};
\fill[] (8,0) circle [radius=0.05]
node[below] {4};
\draw[line width=1mm, yellow] (0,0.7) -- (6.3,0.7)
node[midway, below, yellow!70!black] {stream of \faIcon{hand-holding-usd}};
\end{tikzpicture}


\begin{tikzpicture}
\node[violet] () at (-1.25,0) {Life B \faIcon{user}};
\draw[-Latex] (0,0) -- (10,0) node[right]{Time};
\fill[] (0,0) circle [radius=0.05]
node[below] {0}
node[above] {\faIcon{hand-holding-usd}};
\fill[] (2,0) circle [radius=0.05] 
node[below] {1}
node[above] {\faIcon{hand-holding-usd}};
\fill[] (4,0) circle [radius=0.05]
node[below] {2}
node[above] {\faIcon{hand-holding-usd}};
\fill[] (6,0) circle [radius=0.05]
node[below] {3}
node[above] {\faIcon{hand-holding-usd}};
\fill[] (7.6,0) circle [radius=0.05]
node[red, above] {\faIcon{skull}};
\fill[] (8,0) circle [radius=0.05]
node[below] {4}
node[above, cross out, draw] {\faIcon{hand-holding-usd}};
\end{tikzpicture}

\begin{tikzpicture}
\draw[-Stealth, line width=1mm, ForestGreen] (-2,0) -- (-0.5,0);
\draw[-Latex] (0,0) -- (10,0) node[right]{Time};
\fill[] (0,0) circle [radius=0.05]
node[below] {0};
\fill[] (2,0) circle [radius=0.05]
node[below] {1};
\fill[] (4,0) circle [radius=0.05]
node[below] {2};
\fill[] (6,0) circle [radius=0.05]
node[below] {3};
\fill[] (7.6,0) circle [radius=0.05]
node[red, above] {\faIcon{skull}};
\fill[] (8,0) circle [radius=0.05]
node[below] {4};
\draw[line width=1mm, yellow] (0,0.7) -- (7.6,0.7)
node[midway, below, yellow!70!black] {stream of \faIcon{hand-holding-usd}};
\end{tikzpicture}

\item \labelcref{it:convert-to-stream} describes the conceptual framework for
developing the complete and apportionable variants. In the actual
implementation, the amount and frequency of the payments payable originally
(those before the death {\color{red}\faIcon{skull}}) remain unchanged. 

\item The twist is that in the ``last'' time interval (the time interval
between the original final payment and the scheduled next payment), the
annuitant's account may receive (pay) an additional amount of money from (to)
the insurer at the time of death {\color{red}\faIcon{skull}}, for the
``complete'' (``apportionable'') variant, where the amount is the value of the
``residual payment stream'' \tikz \draw[line width=1mm, yellow] (0,0) --
(0.5,0); accumulated to the time of death {\color{red}\faIcon{skull}}:

\begin{tikzpicture}
\node[] () at (5,2) {``complete'' (immediate)};
\draw[-Latex] (0,0) -- (10,0) node[right]{Time};
\fill[] (0,0) circle [radius=0.05]
node[below] {0};
\fill[] (2,0) circle [radius=0.05]
node[above] {\faIcon{hand-holding-usd}}
node[below] {1};
\fill[] (4,0) circle [radius=0.05]
node[above] {\faIcon{hand-holding-usd}}
node[below] {2};
\fill[] (6,0) circle [radius=0.05]
node[above] {\faIcon{hand-holding-usd}}
node[below] {3};
\fill[] (7,0) circle [radius=0.05]
node[red, above] {\faIcon{skull}};
\fill[] (8,0) circle [radius=0.05]
node[above, cross out, draw] {\faIcon{hand-holding-usd}}
node[below] {4};
\draw[line width=1mm, yellow] (6,0.7) -- (7,0.7);
\draw[->, ForestGreen] (6,0.7) to[bend left] (7,0.7);
\node[orange] () at (7,1.1) {\faIcon{hand-holding-usd}};
\end{tikzpicture}

\begin{tikzpicture}
\node[] () at (5,2) {``apportionable'' (due)};
\draw[-Latex] (0,0) -- (10,0) node[right]{Time};
\fill[] (0,0) circle [radius=0.05]
node[above] {\faIcon{hand-holding-usd}}
node[below] {0};
\fill[] (2,0) circle [radius=0.05]
node[above] {\faIcon{hand-holding-usd}}
node[below] {1};
\fill[] (4,0) circle [radius=0.05]
node[above] {\faIcon{hand-holding-usd}}
node[below] {2};
\fill[] (6,0) circle [radius=0.05]
node[above] {\faIcon{hand-holding-usd}}
node[below] {3};
\fill[] (7,0) circle [radius=0.05]
node[red, above] {\faIcon{skull}};
\fill[] (8,0) circle [radius=0.05]
node[below] {4};
\draw[line width=1mm, yellow] (7,0.7) -- (8,0.7);
\draw[->, ForestGreen] (8,0.7) to[bend right] (7,0.7);
\draw[->, red, line width=0.3mm] (7,0.9) -- (7,1.3)
node[pos=1.6] {\faIcon{dollar-sign}};
\draw[-Latex] (8.5,1.5) -- (7.2,1)
node[pos=-0.7, font=\small] {amount ``not entitled''};
\draw[-Latex] (7,-0.7) -- (6.2,0.2)
node[pos=-0.3, font=\small] {for \emph{whole} period};
\end{tikzpicture}

\item To preserve the ``worth'' of the original payments
\faIcon{hand-holding-usd} (both monetary and time values), in each (\(1/m\)th
of) period we require the ``present value'' (at the beginning time) of the
(single) ``discrete'' payment \faIcon{hand-holding-usd} to be the same as the
(whole) payment stream (in the framework) in that period:

\begin{tikzpicture}
\draw[-Latex] (0,0) -- (10,0) node[right]{Time};
\fill[] (0,0) circle [radius=0.05]
node[below] {0};
\fill[] (2,0) circle [radius=0.05]
node[below] {1}
node[above] {\faIcon{hand-holding-usd}};
\fill[] (4,0) circle [radius=0.05]
node[below] {2}
node[above, gray!50!white] {\faIcon{hand-holding-usd}};
\fill[] (6,0) circle [radius=0.05]
node[below] {3}
node[above] {\faIcon{hand-holding-usd}};
\fill[] (8,0) circle [radius=0.05]
node[below] {4};
\draw[line width=1mm, yellow] (0,0.7) -- (2,0.7)
node[midway, above, yellow!70!black] {stream};
\draw[line width=1mm, yellow!20!white] (2,0.7) -- (4,0.7);
\draw[line width=1mm, yellow] (4,0.7) -- (6,0.7)
node[midway, above, yellow!70!black] {stream};
\draw[-Latex, ForestGreen] (1.2,0.6) to[bend left] (0,0.3);
\draw[-Latex, ForestGreen] (2,0.2) to[bend left] (0,0.2);
\node[font=\small, ForestGreen] () at (-0.3,0.2) {\faIcon{equals}};
\draw[-Latex, orange] (5.2,0.6) to[bend left] (4,0.3);
\draw[-Latex, orange] (6,0.2) to[bend left] (4,0.2);
\node[font=\small, orange] () at (3.7,0.2) {\faIcon{equals}};
\end{tikzpicture}

Then, the original payments \faIcon{hand-holding-usd} together with the
additional amount at time of death would have the same present value as the
payment stream, regardless of when the annuitant dies. So we can (conveniently)
carry out the calculations of APV and other related quantities in the
continuous framework.

\item Now we consider the general setting (\(1/m\)thly). In each \(1/m\)th of
period (year), there is a single payment \faIcon{hand-holding-usd} of \(1/m\)
(at the beginning or the end of the interval, depending on whether the life
annuity is due or immediate). We then want to find the amount \(k\) such that
the ``present value'' of the stream with annual rate \(k\) equals the ``present
value'' of the single payment \faIcon{hand-holding-usd}.

\item \label{it:complete-1m-annuity-fmlas}
For \defn{complete \(1/m\)thly life annuity-immediate}, we have
\[
k\ax*{\angl{1/m}}=(1/m)(1+i)^{-1/m}
\implies
k=\frac{1/m}{\sx*{\angl{1/m}}}
=\frac{1}{m}\frac{\delta}{(1+i)^{1/m}-1}
=\frac{\delta}{i^{(m)}}.
\]
After getting this amount, we can obtain the following\footnote{Here we just
include whole life and \(n\)-year temporary life as they are more often
considered in this context. But in principle similar developments can be done
for other kinds of life annuities.}:
\begin{enumerate}
\item whole life:

\begin{tabular}{cccc}
\toprule
&p.v.r.v.&APV&variance\\
\midrule
expression&\(\displaystyle \frac{\delta}{i^{(m)}}\ax*{\angl{T_x}}\)
&\(\displaystyle \frac{\delta}{i^{(m)}}\ax*{x}\)
&\(\displaystyle \qty(\frac{\delta}{i^{(m)}})^2\times\)(\(\vari{Y}\) in \labelcref{it:cts-wl-annuity-fmlas})\\
notation&\(Y\)&\defn{\(\aringx{x}[(m)]\)}&\(\vari{Y}\)\\
\bottomrule
\end{tabular}

\item \(n\)-year temporary life:

\begin{tabular}{cccc}
\toprule
&p.v.r.v.&APV&variance\\
\midrule
expression&\(\displaystyle \frac{\delta}{i^{(m)}}\ax*{\angl{T_x\wedge n}}\)
&\(\displaystyle \frac{\delta}{i^{(m)}}\ax*{\endowxn}\)
&\(\displaystyle \qty(\frac{\delta}{i^{(m)}})^2\times\)(\(\vari{Y}\) in \labelcref{it:cts-tmp-annuity-fmlas})\\
notation&\(Y\)&\defn{\(\aringx{\endowxn}[(m)]\)}&\(\vari{Y}\)\\
\bottomrule
\end{tabular}
\end{enumerate}
\begin{note}
Since the APV (whole life/temporary life) is just a constant times a previously
discussed ``standard'' APV notation, many previous results hold analogously for
the ``complete'' APV notation.  (It is similar for the apportionable case; see
below.)
\end{note}

\item \label{it:apportionable-1m-annuity-fmlas}
For \defn{apportionable \(1/m\)thly life annuity-due}, we have
\[
k\ax*{\angl{1/m}}=1/m
\implies
k=\frac{1/m}{\ax*{\angl{1/m}}}
=\frac{1}{m}\frac{\delta}{1-(1+i)^{-1/m}}
=\frac{\delta}{d^{(m)}}.
\]
After getting this amount, we can obtain the following:
\begin{enumerate}
\item whole life:

\begin{tabular}{cccc}
\toprule
&p.v.r.v.&APV&variance\\
\midrule
expression&\(\displaystyle \frac{\delta}{d^{(m)}}\ax*{\angl{T_x}}\)
&\(\displaystyle \frac{\delta}{d^{(m)}}\ax*{x}\)
&\(\displaystyle \qty(\frac{\delta}{d^{(m)}})^2\times\)(\(\vari{Y}\) in \labelcref{it:cts-wl-annuity-fmlas})\\
notation&\(Y\)&\defn{\(\ax**{x}[\{m\}]\)}&\(\vari{Y}\)\\
\bottomrule
\end{tabular}

\item \(n\)-year temporary life:

\begin{tabular}{cccc}
\toprule
&p.v.r.v.&APV&variance\\
\midrule
expression&\(\displaystyle \frac{\delta}{d^{(m)}}\ax*{\angl{T_x\wedge n}}\)
&\(\displaystyle \frac{\delta}{d^{(m)}}\ax*{\endowxn}\)
&\(\displaystyle \qty(\frac{\delta}{d^{(m)}})^2\times\)(\(\vari{Y}\) in \labelcref{it:cts-tmp-annuity-fmlas})\\
notation&\(Y\)&\defn{\(\ax**{\endowxn}[\{m\}]\)}&\(\vari{Y}\)\\
\bottomrule
\end{tabular}
\end{enumerate}

\begin{mnemonic}
``\(i^{(m)}\)'' for complete life annuity-\underline{i}mmediate, and
``\(d^{(m)}\)'' for apportionable life annuity-\underline{d}ue.
\end{mnemonic}
\end{enumerate}

\subsection{Life Annuity With Variable Payments}
\begin{enumerate}
\item Like \cref{subsect:vb-insurance}, the general APV formula in
\labelcref{it:gen-apv-fmla} is applicable in the case where the life annuity
has varying payment amounts. \begin{note}
For the continuous case, the ``benefit amount'' part in the general formula
would involve ``\(dt\)'' --- the benefit paid in an ``infinitesimal'' time interval
is ``infinitesimal'' for the continuous life annuity case.
\end{note}
\item Again we shall discuss two special cases:
\begin{itemize}
\item arithmetically increasing/decreasing life annuities
\item geometrically increasing/decreasing life annuities
\end{itemize}

\item Likewise, for arithmetically increasing/decreasing life annuities, there
are some actuarial notations designed for them. The kinds are similar to
\labelcref{it:vb-insurance-kinds} (with the same definition, but for benefit
payment \faIcon{hand-holding-usd} rather than death benefit
\faIcon{money-bill-wave}):
\begin{enumerate}
\item increasing annually
\item increasing \(1/m\)thly
\item increasing continuously
\item decreasing annually
\item decreasing \(1/m\)thly
\item decreasing continuously
\end{enumerate}

\begin{note}
For life annuity-immediate (-due), the payment amount ``for'' each period
refers to the amount for the payment at the end (beginning) of that period.
E.g.,
\begin{itemize}
\item annually increasing annual life annuity-immediate
(payment amounts for policy years 1,2,3,... are 1,2,3,...):

\begin{tikzpicture}
\draw[-Latex] (0,0) -- (10,0) node[right]{Time};
\fill[] (0,0) circle [radius=0.05]
node[below] {0};
\fill[] (1,0) circle [radius=0.05];
\fill[] (2,0) circle [radius=0.05]
node[above, brown] {\faIcon{hand-holding-usd}: 1}
node[below] {1};
\fill[] (3,0) circle [radius=0.05];
\fill[] (4,0) circle [radius=0.05]
node[above, brown] {\faIcon{hand-holding-usd}: 2}
node[below] {2};
\fill[] (5,0) circle [radius=0.05];
\fill[] (6,0) circle [radius=0.05]
node[above, brown] {\faIcon{hand-holding-usd}: 3}
node[below] {3};
\fill[] (7,0) circle [radius=0.05]
node[above=0.2cm, brown, font=\large] {...};
\fill[] (8,0) circle [radius=0.05]
node[below] {4};
\end{tikzpicture}

\item annually increasing annual life annuity-due
(payment amounts for policy years 1,2,3,... are 1,2,3,...):

\begin{tikzpicture}
\draw[-Latex] (0,0) -- (10,0) node[right]{Time};
\fill[] (0,0) circle [radius=0.05]
node[above, brown] {\faIcon{hand-holding-usd}: 1}
node[below] {0};
\fill[] (1,0) circle [radius=0.05];
\fill[] (2,0) circle [radius=0.05]
node[above, brown] {\faIcon{hand-holding-usd}: 2}
node[below] {1};
\fill[] (3,0) circle [radius=0.05]
;
\fill[] (4,0) circle [radius=0.05]
node[above, brown] {\faIcon{hand-holding-usd}: 3}
node[below] {2};
\fill[] (5,0) circle [radius=0.05]
node[above=0.2cm, brown, font=\large] {...};
\fill[] (6,0) circle [radius=0.05]
node[below] {3};
\fill[] (7,0) circle [radius=0.05];
\fill[] (8,0) circle [radius=0.05]
node[below] {4};
\end{tikzpicture}
\end{itemize}
\end{note}

\item The designed actuarial notations for life annuity are analogous to
\labelcref{it:arith-vb-insurance-notations}: Continuous: \(\Ax*{}\to\ax*{}\);
annual: \(\Ax{}\to\ax**{}\text{ or }\ax{}\) (due/immediate resp.); \(1/m\)thly:
\(\Ax{}[(m)]\to\ax**{}[(m)]\text{ or }\ax{}[(m)]\) (due/immediate resp.).

\item \label{it:as-tmp-annuity-fmlas}
We can develop APV formulas for arithmetically increasing/decreasing
\(n\)-year temporary life annuities like \labelcref{it:as-term-life-fmlas},
using the general APV formula in \labelcref{it:gen-apv-fmla}:
\begin{itemize}
\item continuous life annuity: \(\px[t]{x}\mu_{x+t}\text{ \faIcon{arrow-alt-circle-right} }\px[t]{x}\);
\item annual life annuity: \((k+1)v^{k+1}\px[k]{x}\qx{x+k}\text{
\faIcon{arrow-alt-circle-right} }(k+1)v^{k}\px[k]{x}\text{ or
}(k+1)v^{k+1}\px[k+1]{x}\) (due/immediate resp.) (similar for others);
\item \(1/m\)thly life annuity: \(\displaystyle
\frac{k+1}{m}v^{\frac{k+1}{m}}\px[\frac{k}{m}]{x}\qx{x+\frac{k}{m}} \text{
\faIcon{arrow-alt-circle-right} }
\frac{k+1}{m}v^{\frac{k}{m}}\px[\frac{k}{m}]{x}\text{ or }
\frac{k+1}{m}v^{\frac{k+1}{m}}\px[\frac{k+1}{m}]{x}\) (due/immediate resp.)
(similar for others).
\end{itemize}
Example: We have
\[
(\bar{I}\ax*{})_{\endowxn}=\int_{0}^{n}{\color{brown}t}{\color{violet}e^{-\delta
t}}{\color{magenta}\px[t]{x}}\,{\color{brown}dt}
\]
(\({\color{brown}t\,dt}\): ``benefit payment''; \({\color{violet}e^{-\delta
t}}\): ``discount factor''; \({\color{magenta}\px[t]{x}}\): ``prob.\ of
triggering payment'', for every ``infinitesimal'' time interval \([t,t+dt]\))

\item Like \labelcref{it:geo-vb-insurance-intro}, for geometrically
increasing/decreasing life annuities, we shall focus on the annual case (where
the benefit varying and life annuity ``frequencies'' are the same) here:
benefit payments for policy years \(1,2,3,\dotsc\) are
\(1,(1+j),(1+j)^2,\dotsc\) respectively, where \(-1<j<1\). Again one can use
the general APV formula in \labelcref{it:gen-apv-fmla} in general.

\item The following is a shortcut formula of computing the APV for an
\(n\)-year temporary life annuity:
\begin{proposition}
\label{prp:gs-tmp-annuity-fmla}
The APV of an \(n\)-year temporary life annuity (annual case) with such
geometric sequence in the benefit payments is given by:
\[
\begin{cases}
\ax**{x:\angl{n}i^{*}}&\text{(due)}; \\
\displaystyle 
\frac{1}{1+j}\ax{x:\angl{n}i^{*}}&\text{(immediate)}
\end{cases}
\]
where \(i^*=(i-j)/(1+j)\).
\end{proposition}
\begin{pf}
For the ``due'' case, note that the APV is
\[
\sum_{k=0}^{n-1}(1+j)^{k}v^k\px[k]{x}
=\sum_{k=0}^{n-1}\qty(\frac{1+i}{1+j})^{-k}\px[k]{x}.
\]
For the ``immediate'' case, note that the APV is
\[
\sum_{k=0}^{n-1}(1+j)^{k}v^{k+1}\px[k+1]{x}
=\frac{1}{1+j}\sum_{k=0}^{n-1}\qty(\frac{1+i}{1+j})^{-(k+1)}\px[k+1]{x}.
\]
Also recall that
\[
1+i^*=\frac{1+j+i-j}{1+j}=\frac{1+i}{1+j}.
\]
\end{pf}

\begin{note}
Special case: If \(i=j\) (then \(i^{*}=0\)), for the ``due'' case the APV is
\[
\sum_{k=0}^{n-1}\px[k]{x}
=1+e_{x:\angl{n-1}}
\]
by \cref{prp:exn-fmla}. If we further replace \(n\)-year temporary life
annuity by whole life annuity (i.e., let \(n\to\infty\)), then the APV
becomes \(1+e_x\).
\end{note}
\end{enumerate}
\subsection{Recursions for APVs}
\begin{enumerate}
\item Again, most recursion formulas here can be intuitively understood through
``actuarial discounting''.
\item Due to the ``conceptual similarity'' of insurance and life annuity
(general APV formula in \labelcref{it:gen-apv-fmla} is applicable to both), the
recursion formulas for life annuity have a similar ``form'' as the ones for
life insurance.
\item Recursions for whole life annuity:
\begin{proposition}
\label{prp:wl-annuity-recursions}
For any age \(x\) and any \(n\in\N\),
\begin{enumerate}
\item \(\ax*{x}=\ax*{\endowxn}+\Ex[n]{x}\ax*{x+n}\);
\item \(\ax**{x}=\ax**{\endowxn}+\Ex[n]{x}\ax**{x+n}\) (due) and
\(\ax{x}=\ax{\endowxn}+\Ex[n]{x}\ax{x+n}\) (immediate);
\item \(\ax**{x}[(m)]=\ax**{x:\angl{n/m}}[(m)]+\Ex[n/m]{x}\ax**{x+n/m}[(m)]\) (due) and
\(\ax{x}[(m)]=\ax{x:\angl{n/m}}[(m)]+\Ex[n/m]{x}\ax{x+n/m}[(m)]\) (immediate).
\end{enumerate}
\end{proposition}
\begin{intuition}

\begin{tikzpicture}
\draw[-Latex] (0,0) -- (10,0) node[right]{Time};
\fill[] (0,0) circle [radius=0.05]
node[below] {0}
node[above] {\((x)\)};
\fill[] (3,0) circle [radius=0.05]
node[below] {\(n\)}; \draw[pen colour=brown, very thick, decorate,decoration={calligraphic brace, amplitude=5pt, raise=15pt}] (0,0) -- (3,0) node[brown, midway, above=0.7cm]{\(\ax*{\endowxn}\)};
\draw[pen colour=brown, very thick, decorate,decoration={calligraphic brace, amplitude=5pt, raise=25pt}] (3,0) -- (10,0)
node[brown, midway, above=1cm]{\(\ax*{x+n}\)};
\draw[pen colour=violet, very thick, decorate,decoration={mirror, calligraphic brace, amplitude=5pt, raise=15pt}] (0,0) -- (10,0)
node[violet, midway, below=0.7cm]{\(\ax*{x}\)};
\draw[-Latex, color=ForestGreen] (3,1) to[bend right] (0,1.6);
\draw[-Latex, color=ForestGreen] (10,1) to[bend right] (0,1.9);
\node[] () at (2.5,2) {\(\times\Ex[n]{x}\)};
\end{tikzpicture}
\end{intuition}

\begin{pf}
Similar to the proofs in
\labelcref{it:cts-term-life-fmlas,it:annual-term-life-fmlas,it:1m-term-life-fmlas}.
\end{pf}

\begin{note}
Special case: when \(n=1\), we have:
\begin{itemize}
\item \(\displaystyle \ax**{x}=1+v\px{x}\ax**{x+1}\) (due) and 
\(\displaystyle \ax{x}=v\px{x}+v\px{x}\ax{x+1}\) (immediate);
\item \(\displaystyle \ax**{x}[(m)]=\frac{1}{m}+v^{\frac{1}{m}}\px[\frac{1}{m}]{x}\ax**{x+\frac{1}{m}}[(m)]\) (due)
and \(\displaystyle
\ax{x}[(m)]=\frac{1}{m}v^{\frac{1}{m}}\px[\frac{1}{m}]{x}+v^{\frac{1}{m}}\px[\frac{1}{m}]{x}\ax{x+\frac{1}{m}}[(m)]\) 
(immediate).
\end{itemize}
\end{note}

\item Recursions for temporary life annuity:
\begin{proposition}
\label{prp:tmp-annuity-recursions}
For any age \(x\) and any \(n\in\N\),
\begin{enumerate}
\item \(\ax*{\endowxn}=\ax*{x:\angl{u}}+\Ex[u]{x}\ax*{x+u:\angl{n-u}}\) (for any \(u\in\N\) with \(u\le n\));
\item \(\ax**{\endowxn}=\ax**{x:\angl{u}}+\Ex[u]{x}\ax**{x+u:\angl{n-u}}\) (due) and
\(\ax{\endowxn}=\ax{x:\angl{u}}+\Ex[u]{x}\ax{x+u:\angl{n-u}}\) (for any \(u\in\N\) with \(u\le n\)) (immediate);
\item \(\ax**{\endowxn}[(m)]=\ax**{x:\angl{u/m}}[(m)]+\Ex[u/m]{x}\ax**{(x+u/m):\angl{n-u/m}}[(m)]\) (due) and \\
\(\ax{\endowxn}[(m)]=\ax{x:\angl{u/m}}[(m)]+\Ex[u/m]{x}\ax{(x+u/m):\angl{n-u/m}}[(m)]\) (immediate)
(for any \(u\in\N\) with \(u\le mn\)).
\end{enumerate}
\end{proposition}
\begin{intuition}

\begin{tikzpicture}
\draw[-Latex] (0,0) -- (10,0) node[right]{Time};
\fill[] (0,0) circle [radius=0.05]
node[below] {0}
node[above] {\((x)\)};
\fill[] (3,0) circle [radius=0.05]
node[below] {\(u\)};
\fill[] (8,0) circle [radius=0.05]
node[below] {\(n\)};
\draw[pen colour=brown, very thick, decorate,decoration={calligraphic brace, amplitude=5pt, raise=15pt}] (0,0) -- (3,0)
node[brown, midway, above=0.7cm]{\(\ax*{x:\angl{u}}\)};
\draw[pen colour=brown, very thick, decorate,decoration={calligraphic brace, amplitude=5pt, raise=25pt}] (3,0) -- (8,0)
node[brown, midway, above=1cm]{\(\ax*{x+u:\angl{n-u}}\)};
\draw[pen colour=violet, very thick, decorate,decoration={mirror, calligraphic brace, amplitude=5pt, raise=15pt}] (0,0) -- (8,0)
node[violet, midway, below=0.7cm]{\(\ax*{\endowxn}\)};
\draw[-Latex, color=ForestGreen] (3,1) to[bend right] (0,1.6);
\draw[-Latex, color=ForestGreen] (8,1) to[bend right] (0,1.9);
\node[] () at (2.5,2) {\(\times\Ex[u]{x}\)};
\end{tikzpicture}
\end{intuition}

\begin{pf}
Similar to the proof for \cref{prp:wl-annuity-recursions}.
\end{pf}

\item Recursions for arithmetically increasing/decreasing life annuities
\(\Ia**\)/\(\Da**\):
\begin{proposition}
\label{prp:annual-arith-annuity-recursions}
We have
\begin{enumerate}
\item\label{it:incr-split-first-annuity}
\(\Ia**_{\endowxn}=1+v\px{x}\qty[\Ia**_{x+1:\angl{n-1}}+\ax**{x+1:\angl{n-1}}]\)
and
\(\Ia**_{x}=1+v\px{x}\qty[\Ia**_{x+1}+\ax**{x+1}]\);
\item\label{it:decr-split-first-annuity} \(\Da**_{\endowxn}=n+v\px{x}\Da**_{x+1:\angl{n-1}}\);
\item\label{it:incr-split-horizon-annuity}
\(\Ia**_{\endowxn}=\ax**{\endowxn}+v\px{x}\Ia**_{x+1:\angl{n-1}}\) and
\(\Ia**_{x}=\ax**{x}+v\px{x}\Ia**_{x+1}\).
\end{enumerate}
\end{proposition}

\begin{intuition}

\labelcref{it:incr-split-first-annuity}:

\begin{tikzpicture}
\draw[-Latex] (0,0) -- (10,0) node[right]{Time};
\fill[] (0,0) circle [radius=0.05]
node[below] {0}
node[above, violet] {\faIcon{hand-holding-usd}};
\fill[] (2,0) circle [radius=0.05]
node[below] {1}
node[above, magenta] {\faIcon{hand-holding-usd}}
node[above=0.3cm, brown] {\faIcon{hand-holding-usd}};
\fill[] (4,0) circle [radius=0.05]
node[below] {2}
node[above, magenta] {\faIcon{hand-holding-usd}}
node[above=0.3cm, brown] {\faIcon{hand-holding-usd}}
node[above=0.6cm, brown] {\faIcon{hand-holding-usd}};
\fill[] (6,0) circle [radius=0.05]
node[below] {3}
node[above, magenta] {\faIcon{hand-holding-usd}}
node[above=0.3cm, brown] {\faIcon{hand-holding-usd}}
node[above=0.6cm, brown] {\faIcon{hand-holding-usd}}
node[above=0.9cm, brown] {\faIcon{hand-holding-usd}};
\fill[] (8,0) circle [radius=0.05]
node[below] {4};
\node[thick, brown] () at (7,1) {\(\cdots\)};
\node[thick, magenta] () at (7,0.3) {\(\cdots\)};
\draw[-Latex, violet] (1,-0.7) -- (0.3,0)
node[pos=-0.3] {1};
\draw[-Latex, magenta] (9,0.3) -- (7.5,0.3)
node[pos=-0.6] {\(\ax**{x+1:\angl{n-1}}\)};
\draw[brown, thick, decorate,decoration={mirror, brace, amplitude=5pt, raise=15pt}] (7.5,0.5) -- (7.5,1.5)
node[midway, right=1cm] {\(\Ia**_{x+1:\angl{n-1}}\)};
\draw[-Latex, ForestGreen] (1.7,0.2) to[bend right] (0,0.1);
\draw[-Latex, ForestGreen] (6.2,1.4) to[bend right] (0,0.3);
\node[] () at (1.5,1.8) {\(\times v\px{x}\)};
\end{tikzpicture}

\labelcref{it:decr-split-first-annuity}:

\begin{tikzpicture}
\draw[-Latex] (0,0) -- (10,0) node[right]{Time};
\fill[] (0,0) circle [radius=0.05]
node[below] {0}
node[above, violet] {\faIcon{money-bill-wave}}
node[above=0.4cm, violet] {\(\vdots\)}
node[above=0.8cm, violet] {\faIcon{money-bill-wave}}
node[above=1.2cm, violet] {\faIcon{money-bill-wave}}
node[above=1.6cm, violet] {\faIcon{money-bill-wave}}
node[above=2cm, violet] {\faIcon{money-bill-wave}}
;
\fill[] (2,0) circle [radius=0.05]
node[below] {1}
node[above, brown] {\faIcon{money-bill-wave}}
node[above=0.4cm, brown] {\(\vdots\)}
node[above=0.8cm, brown] {\faIcon{money-bill-wave}}
node[above=1.2cm, brown] {\faIcon{money-bill-wave}}
node[above=1.6cm, brown] {\faIcon{money-bill-wave}}
;
\fill[] (4,0) circle [radius=0.05]
node[below] {2}
node[above, brown] {\faIcon{money-bill-wave}}
node[above=0.4cm, brown] {\(\vdots\)}
node[above=0.8cm, brown] {\faIcon{money-bill-wave}}
node[above=1.2cm, brown] {\faIcon{money-bill-wave}};
;
\fill[] (6,0) circle [radius=0.05]
node[below] {3}
node[above, brown] {\faIcon{money-bill-wave}}
node[above=0.4cm, brown] {\(\vdots\)}
node[above=0.8cm, brown] {\faIcon{money-bill-wave}}
;

\fill[] (8,0) circle [radius=0.05]
node[below] {4}
;
\node[thick, brown] () at (6.7,1) {\(\cdots\)};
\draw[-Latex, violet] (1,-0.7) -- (0.3,0)
node[pos=-0.3] {\(n\)};
\draw[brown, thick, decorate,decoration={mirror, brace, amplitude=5pt, raise=15pt}] (6.5,0.2) -- (6.5,2)
node[midway, right=1cm] {\(\DA_{x+1:\angl{n-1}}\)};
\draw[-Latex, ForestGreen] (1.7,0.2) to[bend right] (0,0.1);
\draw[-Latex, ForestGreen] (6.2,1.4) to[bend right] (0,0.3);
\node[brown] () at (1,1.6) {\(\times v\px{x}\)};
\end{tikzpicture}

\labelcref{it:incr-split-horizon-annuity}:

\begin{tikzpicture}
\draw[-Latex] (0,0) -- (10,0) node[right]{Time};
\fill[] (0,0) circle [radius=0.05]
node[below] {0}
node[above, violet] {\faIcon{hand-holding-usd}};
;
\fill[] (2,0) circle [radius=0.05]
node[below] {1}
node[above, violet] {\faIcon{hand-holding-usd}}
node[above=0.3cm, brown] {\faIcon{hand-holding-usd}};
\fill[] (4,0) circle [radius=0.05]
node[below] {2}
node[above, violet] {\faIcon{hand-holding-usd}}
node[above=0.3cm, brown] {\faIcon{hand-holding-usd}}
node[above=0.6cm, brown] {\faIcon{hand-holding-usd}};
\fill[] (6,0) circle [radius=0.05]
node[below] {3}
node[above, violet] {\faIcon{hand-holding-usd}}
node[above=0.3cm, brown] {\faIcon{hand-holding-usd}}
node[above=0.6cm, brown] {\faIcon{hand-holding-usd}}
node[above=0.9cm, brown] {\faIcon{hand-holding-usd}};
\fill[] (8,0) circle [radius=0.05]
node[below] {4};
\node[thick, brown] () at (7,1) {\(\cdots\)};
\node[thick, violet] () at (7,0.3) {\(\cdots\)};
\draw[-Latex, violet] (9,0.3) -- (7.5,0.3)
node[pos=-0.3] {\(\ax**{\endowxn}\)};
\draw[brown, thick, decorate,decoration={mirror, brace, amplitude=5pt, raise=15pt}] (7.5,0.5) -- (7.5,1.5)
node[midway, right=1cm] {\(\Ia**_{x+1:\angl{n-1}}\)};
\draw[-Latex, ForestGreen] (1.7,0.5) to[bend right] (0,0.1);
\draw[-Latex, ForestGreen] (6.2,1.4) to[bend right] (0,0.3);
\node[brown] () at (1.5,1.8) {\(\times v\px{x}\)};
\end{tikzpicture}
\end{intuition}

\begin{pf}
Exercise. (The intuition already illustrates the key idea in the proof ---
one just needs to ``split'' the terms appropriately.)
\end{pf}
\end{enumerate}
\subsection{Relating \(\ax*{}\), \(\ax**{}\) and \(\ax**{}[(m)]\)}
\begin{enumerate}
\item In a life table, we often only have the values for
``\(\ax**{}\)''\footnote{Usually, ``due'' quantities, instead of ``immediate''
quantities, are given in a life table as they are more frequently used.} but
not for ``\(\ax*{}\)'' and ``\(\ax**{}[(m)]\)''. So, again we are interested in
the relationship between them to see how we can get ``\(\ax*{}\)'' and
``\(\ax**{}[(m)]\)'' from ``\(\ax**{}\)''.
\item To get a nice relationship, again we need to impose UDD assumption (so
that \cref{prp:udd-relate-insurance-apvs} can be utilized):
\begin{proposition}
\label{prp:udd-relate-annuity-apvs}
Under UDD assumption, we have
\begin{enumerate}
\item \(\ax**{x}[(m)]=\alpha(m)\ax**{x}{\color{purple}-}\beta(m)\)
where \(\displaystyle \alpha(m)=\frac{id}{i^{(m)}d^{(m)}}\) and \(\displaystyle \beta(m)=\frac{i-i^{(m)}}{i^{(m)}d^{(m)}}\);
\item \(\displaystyle \ax*{x}=\frac{id}{\delta^2}\ax**{x}{\color{purple}-}\frac{i-\delta}{\delta^2}\).
\end{enumerate}
\end{proposition}
\begin{pf}
It suffices to prove the \(1/m\)thly case since the continuous case can simply
be obtained from the \(1/m\)thly case by letting \(m\to\infty\) (note that
\(\lim_{m\to \infty}i^{(m)}=\lim_{m\to \infty}d^{(m)}=\delta\)).

Now consider:
\[
\ax**{x}[(m)]
=\frac{1-\Ax{x}[(m)]}{d^{(m)}}
=\frac{1-\qty(i/i^{(m)})\Ax{x}}{d^{(m)}}
=\frac{i^{(m)}-i(1-d\ax**{x})}{i^{(m)}d^{(m)}}
=\alpha(m)\ax**{x}-\beta(m).
\]
\end{pf}

\begin{remark}
\item The result for \(1/m\)thly case is more often used.
\item Sometimes values of \(\alpha(m)\) and \(\beta(m)\) for different \(m\) (at some
interest rate \(i\)) are given also in a life table, for convenience.
\item This result is only for \emph{whole life} annuity, unlike
\cref{prp:udd-relate-insurance-apvs}. But we can still get APVs for temporary
life and deferred life annuities easily, by expressing them in terms of APVs of
whole life annuities. For example:
\begin{itemize}
\item \(n\)-year temporary life:
\[\ax**{\endowxn}[(m)]=\ax**{x}-\Ex[n]{x}\ax**{x+n}
=\alpha(m)\ax**{x}-\beta(m)-\Ex[n]{x}(\alpha(m)\ax**{x+n}-\beta(m))
=\alpha(m)\ax**{\endowxn}{\color{purple}-}\beta(m)(1-\Ex[n]{x}).
\]
\item deferred whole life:
\[\ax**[u|]{x}[(m)]=\Ex[u]{x}\ax**{x+u}
=\Ex[u]{x}\qty[\alpha(m)\ax**{x+u}-\beta(m)]
=\alpha(m)\ax**[u|]{x}{\color{purple}-}\beta(m)\Ex[u]{x}.
\]
\end{itemize}
\end{remark}
\end{enumerate}
\subsection{Incorporating Selection}
\begin{enumerate}
\item Again, all previous developments also apply to ``select'' lives (just change
``\(x\)'' to ``\([x]\)'' in the notations).
\end{enumerate}
