\section{Policy Values}
\label{sect:policy-values}
\begin{enumerate}
\item After a life insurance policy \faIcon{file-alt} is sold, the insurance
company \faIcon{building} is not going to completely ignore it. Indeed, to
ensure that \faIcon{building} ``puts aside'' (or \emph{reserves}) sufficient
money \faIcon{coins} for paying the potential future benefits and expenses,
\faIcon{building} should monitor \faIcon{stethoscope} the sold policies
\faIcon{file-alt} periodically. (In practice, this is often required by
regulators \faIcon{balance-scale} to make sure \faIcon{building} has a good
``financial health'' \faIcon{notes-medical}.)

\item To monitor the ``status''  of the policies sold, the insurer
\faIcon{building} (more specifically, \emph{actuarial valuation team}) would
periodically carry out \emph{valuation} on those policies (based on the most
``updated'' information \faIcon{info}). A fundamental quantity of interest is
how much money \faIcon{coins} the insurer \faIcon{building} needs to reserve
for a given policy, at a specified time point.

\item Here we shall discuss a relatively ``simple'' framework for determining
such reserve, which revolves around the concept of \emph{policy value}.

\begin{note}
Again, the process is much more sophisticated in the actual practice, and
readers may consider having an actuarial internship in valuation team to learn
more about it \faIcon[regular]{smile}.
\end{note}
\end{enumerate}

\subsection{Present Value of Future Loss at Time \(t\)}
\begin{enumerate}
\item To define policy value, we need to first introduce \defn{present value of
future loss at time \(t\)}, which has a definition analogous to the present
value of future loss at issue (indeed this is a generalization to it):
\begin{align*}
&\text{``PV'' of ``future'' benefits \faIcon{money-bill-wave} outgo @ time \(t\)}  \\
&\quad+ \text{``PV'' of ``future'' expenses \faIcon{file-invoice-dollar} (if considered) @ time \(t\)}\\
&\quad- \text{``PV'' of ``future'' premium incomes \faIcon{dollar-sign} @ time \(t\)},
\end{align*}
and it is only defined when the policy is \emph{in force} (i.e., is not
terminated) at time \(t\).

\begin{remark}
\item Here the ``in force'' condition is added since it does not make much
sense to talk about ``future'' benefits/expenses/premium incomes when they
\emph{do not even exist} (``no future'' for a terminated policy)!
\item We shall assume here that no policy lapse is possible. So the ``in
force'' condition is equivalent to the life insured \faIcon{user} is still
alive at time \(t\) (\(T_x>t\)).

\item ``PV'' @ time \(t\) means present value with time \(t\) (instead of time
0) treated as ``present''. More specifically, we convert ``\(T_x\)'' in the
expression for loss at issue (in \labelcref{it:pv-future-loss-at-issue}) to
``\(T_x-t\)'' (the ``future lifetime'' is taken with respect to time \(t\)) to
get the expression here.

\item Since the expression here is only defined when \(T_x>t\), it is natural
to identify its distribution as the \emph{conditional distribution} of the
expression involving \(T_x-t\) \emph{given} \(T_x>t\) (which is assumed to be
the same as the \emph{unconditional} distribution of \(T_{x+t}\)). So, we can
replace ``\(T_x-t\)'' in the expression by \(T_{x+t}\) (and drop the extra
condition) without affecting the distribution.

\item As we are almost always only interested in studying the
\emph{distribution} of the expression, writing the expression in terms of
either ``\(T_x-t\)'' (with the condition) or ``\(T_{x+t}\)'' (with the
condition dropped) is fine. \ystar Here, we shall write the expression in terms of
\(T_{x+t}\) (with the condition dropped) for ``simplicity''. (Note that this
means the curtate lifetime random variables (``\(K\)'') would also have
subscript ``\(x+t\)''.)
\end{remark}

\item Likewise, when the expenses \faIcon{file-invoice-dollar} are considered, the
present value of future loss at time \(t\) is known as \defn{gross loss at time
\(t\)}.  Otherwise it is known as \defn{net loss at time \(t\)}.

\item The gross (net) loss at time \(t\) is denoted by \defn{\(L_t^g\)}
(\defn{\(L_t^n\)} resp.).

\begin{remark}
\item Again, if there is no risk of ambiguity or one would like to refer to
either of gross and net losses at time \(t\) simultaneously, one may drop the
superscript and just write \defn{\(L_t\)}.

\item The ``\(t\)'' indicates the present value is at time \(t\).
\end{remark}

\item The \defn{net premium policy value at time \(t\)} and \defn{gross premium
policy value at time \(t\)}, denoted by \defn{\(\Vx[t]{}[n]\)}
and \defn{\(\Vx[t]{}[g]\)} respectively, are given by (when the policy is in
force at time \(t\))
\[
\Vx[t]{}[n]=\expv{L_t^n}\quad\text{and}\quad\Vx[t]{}[g]=\expv{L_t^g},
\]
that is,
\begin{align*}
\Vx[t]{}[n]
&=\text{``APV'' of ``future'' benefits \faIcon{money-bill-wave} outgo @ time \(t\)}\\
&\quad-\text{``APV'' of ``future'' premium incomes \faIcon{dollar-sign} @ time \(t\)},
\end{align*}
and
\begin{align*}
\Vx[t]{}[g]
&=\text{``APV'' of ``future'' benefits \faIcon{money-bill-wave} outgo @ time \(t\)} \\
&\quad+ \text{``APV'' of ``future'' expenses \faIcon{file-invoice-dollar} @ time \(t\)} \\
&\quad-\text{``APV'' of ``future'' premium incomes \faIcon{dollar-sign} @ time \(t\)}.
\end{align*}

\begin{intuition}
The definition is ``similar'' to the definition for ``ordinary'' value. Value
of a ``nonrandom'' project at time \(t\) may be defined as the present
value\footnote{The interest rate used is ``risk-free rate''.} of present/future
(nonrandom) net cash (in)flows at time \(t\) . This is also the amount of
\faIcon{coins} received when the project is ``fairly sold'' at time \(t\)
(intuitively, this is its ``worth'' at time \(t\)).

On the other hand, the policy value at time \(t\) is the \emph{actuarial}
present value of ``future'' cash \emph{outflows} (from the insurer's
\faIcon{building} perspective). It may be understood intuitively (and
informally) as ``amount of \faIcon{coins} \emph{paid} (by \faIcon{building})
when the policy \faIcon{file-alt} is `fairly transferred' to another party
\faIcon[regular]{handshake}''. As a ``simple'' framework for reserving, this
amount \faIcon{coins} may be treated as the amount of reserve held at time
\(t\).
\end{intuition}

\begin{remark}
\item If there is no risk of ambiguity or one would like to refer to either of gross
and net premium policy values at time \(t\) simultaneously, one may drop the
superscript and just write \defn{\(\Vx[t]{}\)}.

\item For an \(n\)-year term policy, technically  \(\Vx[n]{}\) is
\emph{undefined} by definition (as the policy terminates precisely at time
\(n\)). But still it is common to use the notation \(\Vx[n]{}\) to refer to the
\emph{limit} \(\lim_{t\to n^{-}}\Vx[t]{}\) (the value approached as \(t<n\)
gets closer and closer to \(n\)).
\end{remark}

\item \label{it:pv-boundary-values}
Boundary values of \(\Vx[t]{}\):
\begin{itemize}
\item \(\Vx[0]{}[n]=0\) if net premium is determined via equivalence principle;
\item \(\Vx[0]{}[g]=0\) if (i) gross premium is determined via equivalence
principle, and (ii) policy value basis and premium basis are the same;
\begin{note}
\defn{Basis}
means a set of assumptions. This means the sets of assumptions used for
calculating policy value and premium are the same.
\end{note}
\item \[
\Vx[n]{}
=\begin{cases}
0&\text{for \(n\)-year term life insurance}; \\
S&\text{for \(n\)-year endowment insurance},
\end{cases}
\]
where \(S\) is the sum insured (amount of death/survival benefit).
\end{itemize}
\begin{note}
The boundary values are useful for \emph{recursions} (see
\cref{subsect:pv-annual-recursion}).
\end{note}
\end{enumerate}
\subsection{Calculations of Policy Values and Variances of \(L_t\)}
\begin{enumerate}
\item We shall again consider the whole life insurance in
\labelcref{it:wl-gross-loss-expr}. The gross loss at time \(t\) is
\[
L_t^g=(S+E)v^{K_{x+t}+1}-(G-e)\ax**{\angl{K_{x+t}+1}}.
\]
or
\[
L_t^g=(S+E)Z-(G-e)Y,
\]
where \(Z\) and \(Y\) are the p.v.r.v's for annual whole life insurance and
annual whole life annuity-due (with benefits being 1 [each]) issued to
\((x+t)\) respectively.

\item \label{it:wl-gross-pv-fmla}
With the expression of \(L_t^g\), we readily get the following formula
for \(\Vx[t]{}[g]\):
\[
\Vx[t]{}[g]=(S+E)\Ax{x+t}-(G-e)\ax**{x+t}.
\]

\item Likewise we can get the expression for net loss at time \(t\) by removing
all expenses (the annual net premiums are \(S\Px{x}\) each, determined by
equivalence principle):
\[
L_t^n=SZ-SP_xY=SZ-\frac{S\Ax{x}}{\ax**{x}}Y.
\]

\item
We can then obtain the following formulas for \(\Vx[t]{}[n]\):
\begin{proposition}
\label{prp:wl-net-pv-fmlas}
For the whole life insurance with setting described in
\labelcref{it:wl-gross-loss-expr}, we have
\[
\Vx[t]{}[n]=S\qty(1-\frac{\ax**{x+t}}{\ax**{x}})
=S\qty(\frac{\Ax{x+t}-\Ax{x}}{1-\Ax{x}}).
\]
\end{proposition}
\begin{pf}
Firstly, we have
\[
\Vx[t]{}[n]=S\Ax{x+t}-S\Px{x}\ax**{x+t}
=S\qty[1-d\ax**{x+t}-\qty(\frac{1}{\ax**{x}}-d)\ax**{x+t}]
=S\qty(1-\frac{\ax**{x+t}}{\ax**{x}}).
\]
Next, we have
\[
\Vx[t]{}[n]
=S\Ax{x+t}-S\Px{x}\ax**{x+t}
=S\qty[\frac{\Ax{x+t}(1-\Ax{x})}{1-\Ax{x}}-\frac{d\Ax{x}}{1-\Ax{x}}\cdot\frac{1-\Ax{x+t}}{d}]
=S\qty(\frac{\Ax{x+t}-\Ax{x}}{1-\Ax{x}}).
\]
\end{pf}

\item \label{it:loss-at-t-var-fmlas}
For the variance formulas, using similar arguments as the proofs for
\cref{prp:wl-gross-loss-var-fmla,prp:wl-net-loss-var-fmla}, we get:
\begin{itemize}
\item \(\displaystyle\vari{L_t^g}=\qty(S+E+\frac{G-e}{d})^{2}(\Ax[][2]{{\color{purple}x+t}}-\Ax{{\color{purple}x+t}}^2)\);

\item \(\displaystyle \vari{L_t^n}=\qty(\frac{S}{1-\Ax{x}})^{2}(\Ax[][2]{{\color{purple}x+t}}-\Ax{{\color{purple}x+t}}^2)\).
\end{itemize}


\item Using similar arguments, we can again adapt \cref{prp:wl-net-pv-fmlas} and
formulas in \labelcref{it:loss-at-t-var-fmlas} for
\begin{itemize}
\item endowment insurance: change 
\begin{itemize}
\item \(\Ax{x}\to\Ax{\endowxn}\) and \(\ax**{x}\to\ax**{\endowxn}\);

\item \(\Ax{x+t}\to\Ax{x+t:\angl{n-t}}\) and \(\ax**{x+t}\to\ax**{x+t:\angl{n-t}}\);

\item \(\Ax[][2]{x+t}\to \Ax[][2]{x+t:\angl{n-t}}\);
\end{itemize}

\item \(1/m\)thly case (for both insurance and premiums): change
\(\ax**{}\to\ax**{}[(m)]\), \(\Ax{}\to\Ax{}[(m)]\), and \(d\to d^{(m)}\);
\item fully continuous case: change
\(\ax**{}\to\ax*{}\), \(\Ax{}\to\Ax*{}\), and \(d\to \delta\).
\end{itemize}
\end{enumerate}
\subsection{Annual Recursion}
\label{subsect:pv-annual-recursion}
\begin{enumerate}
\item Again there are two main reasons for studying recursions of policy values here:
\begin{enumerate}
\item It provides insight on how the policy values change over time.
\item It allows quick computations based on limited amount of information.
\end{enumerate}

\item To intuitively understand recursions for policy values, it would be
helpful to regard policy value at time \(t\) as the amount of reserve
\faIcon{coins} held at time \(t\) (i.e., work in the ``simple'' framework for
reserving).

\item Here we shall focus on \emph{annual} recursion. For other kinds of
recursions and further topics about it, see STAT3909.

\item \label{it:pv-annual-recursion}
\ystar An ``intuition-based'' annual recursion formula for policy values
(insurance) is as follows (where \(t\in\N_0\)):
\begin{align*}
&(\Vx[t]{}+{\color{brown}\text{premium \faIcon{dollar-sign}}}
-{\color{violet}\text{initial/renewal expense \faIcon{hourglass-start}/\faIcon{redo}}})
({\color{blue}1+i})\\
&\quad=\Vx[t+1]{}\,({\color{ForestGreen}\text{survival prob.}})+({\color{orange}\text{death benefit \faIcon{money-bill-wave}}}
+\text{{\color{violet}termination expense \faIcon{hourglass-end}}})({\color{red}\text{death prob.}}).
\end{align*}
\begin{center}
\begin{tikzpicture}
\draw[] (0,0) rectangle (2,1);
\node[] () at (1, 1.4) {\(\Vx[t]{}\)};
\node[] () at (0.3,0.7) {\faIcon{coins}};
\node[] () at (0.7,0.7) {\faIcon{coins}};
\node[] () at (1.1,0.7) {\faIcon{coins}};
\node[] () at (1.5,0.7) {\faIcon{coins}};
\node[brown] () at (0.3,0.3) {\faIcon{dollar-sign}};
\node[brown] () at (0.7,0.3) {\faIcon{dollar-sign}};
\node[draw=red, cross out] () at (1.1,0.3) {\faIcon{coins}};
\draw[-Latex, violet] (1.1,0.2) to[bend right] (1.5,-0.5);
\node[violet, font=\small] () at (1.5,-0.7) {expense \faIcon{hourglass-start}/\faIcon{redo}};
\draw[-Latex] (2,0.5) -- (4,1.3)
node[midway, above=0.2cm, ForestGreen, font=\small] {survive};
\draw[-Latex] (2,0.5) -- (4,-0.3)
node[midway, below=0.2cm, red, font=\small] {die};
\draw[] (4,0.7) rectangle (6,1.7);
\draw[] (4,0.3) rectangle (6,-0.7);
\node[] () at (5, 2.2) {\(\Vx[t+1]{}\)};
\draw[red] (4.2,0.1) -- (5.8,-0.5);
\draw[red] (4.2,-0.5) -- (5.8,0.1);
\node[] () at (4.3,1.4) {\faIcon{coins}};
\node[] () at (4.7,1.4) {\faIcon{coins}};
\node[] () at (5.1,1.4) {\faIcon{coins}};
\node[] () at (5.5,1.4) {\faIcon{coins}};
\node[brown] () at (4.3,1) {\faIcon{coins}};
\node[brown] () at (4.7,1) {\faIcon{coins}};
\node[blue] () at (5.2,1) {\faIcon{money-bill}};
\draw[-Latex, red] (5,-0.4) to[bend right] (5,-1.2);
\node[orange] () at (4.3,-1.5) {\faIcon{coins}};
\node[orange] () at (4.7,-1.5) {\faIcon{coins}};
\node[orange] () at (5.1,-1.5) {\faIcon{coins}};
\node[orange] () at (5.5,-1.5) {\faIcon{coins}};
\node[orange] () at (5.9,-1.5) {\faIcon{coins}};
\node[violet] () at (4.3,-1.9) {\faIcon{coins}};
\node[violet] () at (4.8,-1.9) {\faIcon{money-bill}};
\draw[orange, ->, thick] (7.4,-1.5) -- (6.6,-1.5)
node[pos=-1.8] {death benefit \faIcon{money-bill-wave}};
\draw[violet, ->, thick] (7,-1.9) -- (6.2,-1.9)
node[pos=-1.2] {expense \faIcon{hourglass-end}};
\draw[-Latex, blue] (1,2.5) to[bend left] (4.5,2.5);
\node[blue] () at (2.75,2.4) {accumulate};
\end{tikzpicture}
\end{center}

\begin{pf}

\begin{tikzpicture}
\draw[-Latex] (0,0) -- (10,0) node[right]{Time};
\fill[] (0,0) circle [radius=0.05]
node[below=0.1cm] {0};
\fill[] (2,0) circle [radius=0.05]
node[below=0.1cm] {1};
\fill[] (4,0) circle [radius=0.05]
node[below=0.1cm] {\(t=2\)};
\fill[] (6,0) circle [radius=0.05]
node[below=0.1cm] {3};
\fill[] (8,0) circle [radius=0.05]
node[below=0.1cm] {4};
\draw[->, brown, line width=0.3mm, opacity=0.2] (-0.05,-1.2) -- (-0.05,-1.6)
node[pos=2] {\faIcon{dollar-sign}};
\draw[->, violet, line width=0.3mm, opacity=0.2] (0.05,-1.6) -- (0.05,-1.2)
node[pos=2] {\faIcon{hourglass-start}};
\draw[->, brown, line width=0.3mm, opacity=0.2] (1.95,-1.2) -- (1.95,-1.6)
node[pos=2] {\faIcon{dollar-sign}};
\draw[->, violet, line width=0.3mm, opacity=0.2] (2.05,-1.6) -- (2.05,-1.2)
node[pos=2] {\faIcon{redo}};
\draw[->, brown, line width=0.3mm] (3.95,-1.2) -- (3.95,-1.6)
node[pos=2] {\faIcon{dollar-sign}};
\draw[->, violet, line width=0.3mm] (4.05,-1.6) -- (4.05,-1.2)
node[pos=2] {\faIcon{redo}};
\draw[->, brown, line width=0.3mm] (5.95,-1.2) -- (5.95,-1.6)
node[pos=2] {\faIcon{dollar-sign}};
\draw[->, violet, line width=0.3mm] (6.05,-1.6) -- (6.05,-1.2)
node[pos=2] {\faIcon{redo}};
\draw[->, brown, line width=0.3mm] (7.95,-1.2) -- (7.95,-1.6)
node[pos=2] {\faIcon{dollar-sign}};
\draw[->, violet, line width=0.3mm] (8.05,-1.6) -- (8.05,-1.2)
node[pos=2] {\faIcon{redo}};


\draw[->, orange, line width=0.3mm, opacity=0.2] (-0.05,1.2) -- (-0.05,1.6)
node[pos=2] {\faIcon{money-bill-wave}};
\draw[->, violet, line width=0.3mm, opacity=0.2] (0.05,1.2) -- (0.05,1.6)
node[pos=2] {\faIcon{hourglass-end}};
\draw[->, orange, line width=0.3mm, opacity=0.2] (1.95,1.2) -- (1.95,1.6)
node[pos=2] {\faIcon{money-bill-wave}};
\draw[->, violet, line width=0.3mm, opacity=0.2] (2.05,1.2) -- (2.05,1.6)
node[pos=2] {\faIcon{hourglass-end}};
\draw[->, orange, line width=0.3mm, opacity=0.2] (3.95,1.2) -- (3.95,1.6)
node[pos=2] {\faIcon{money-bill-wave}};
\draw[->, violet, line width=0.3mm, opacity=0.2] (4.05,1.2) -- (4.05,1.6)
node[pos=2] {\faIcon{hourglass-end}};
\draw[->, orange, line width=0.3mm] (5.95,1.2) -- (5.95,1.6)
node[pos=2] {\faIcon{money-bill-wave}};
\draw[->, violet, line width=0.3mm] (6.05,1.2) -- (6.05,1.6)
node[pos=2] {\faIcon{hourglass-end}};
\draw[->, orange, line width=0.3mm] (7.95,1.2) -- (7.95,1.6)
node[pos=2] {\faIcon{money-bill-wave}};
\draw[->, violet, line width=0.3mm] (8.05,1.2) -- (8.05,1.6)
node[pos=2] {\faIcon{hourglass-end}};

\draw[dashed, ForestGreen] (8,2.5) -- (4.5,-2.5);
\draw[dashed, ForestGreen] (8,2.5) -- (10,2.5);
\draw[dashed, ForestGreen] (4.5,-2.5) -- (10,-2.5);
\draw[-Latex, ForestGreen] (6.8,0.3) to[bend right] (4,1);
\node[] () at (5,3) {(actuarial discounting)};
\node[ForestGreen] () at (2,0.9) {\(v\,(\text{survival prob.})\Vx[t+1]{}\;\)};
\draw[dashed, magenta] (3.7,-0.5) rectangle (4.3,-2.5)
node[below] {\(\text{\faIcon{redo}}-\text{\faIcon{dollar-sign}}\)};
\draw[dashed, teal] (5.7,1) rectangle (6.3,2.5);
\draw[-Latex, teal] (5.7,0.8) to[bend left] (4,0.5);
\node[teal, font=\small] () at (2,0.4) {\(v(\text{death prob.})(\text{\faIcon{money-bill-wave}}+\text{\faIcon{hourglass-end}})\)};
\end{tikzpicture}

\begin{note}
We treat cash outflow as positive and cash inflow as negative, since we are
dealing with \emph{losses} for policy values (so we consider net cash
\emph{out}flows).
\end{note}

The figure above illustrates the key idea in the proof (``splitting'' the cash
flows appropriately). Using this idea, we get:
\[
\Vx[t]{}={\color{magenta}\text{\faIcon{redo}}-\text{\faIcon{dollar-sign}}}
+{\color{teal}v(\text{death prob.})(\text{\faIcon{money-bill-wave}}+\text{\faIcon{hourglass-end}})}
+{\color{ForestGreen}v(\text{survival prob.})\,\Vx[t+1]{}}.
\]
When \(t=0\), ``\faIcon{redo}'' is replaced by ``\faIcon{hourglass-start}''. So
the result follows, by rearranging the terms.
\end{pf}
\end{enumerate}
\subsection{Retrospective Policy Value}
\begin{note}
This topic is not inside SOA exam FAM syllabus currently.
\end{note}
\begin{enumerate}
\item Recall that the policy value at time \(t\) is defined as the actuarial
present value of future loss at time \(t\). Sometimes this policy value is
called \defn{prospective policy value} since the definition is inherently
forward looking and prospective.

\item Now we are interested in studying another approach for determining policy
value: \emph{retrospective approach}. Loosely, it calculates the
\emph{retrospective policy value}, which is the ``actuarial accumulated value''
of past gain.

\begin{note}
The retrospective policy value is kind of a ``dual'' of the prospective policy value:
\begin{itemize}
\item actuarial accumulated value \faIcon{arrows-alt-h} actuarial present value;

\item past \faIcon{arrows-alt-h} future;

\item gain \faIcon{arrows-alt-h} loss.
\end{itemize}
\end{note}

\item Before proceeding further, we should define what ``actuarial accumulated
value'' is. Recall that actuarial present value can be obtained through
actuarial discounting (``multiply \(\Ex[n]{x}\)'' to actuarially discount back
\(n\) years).

Following the same spirit, actuarial accumulated value is defined such that an
analogous ``actuarial accumulation'' works. The \defn{actuarial accumulated
value} (AAV) of a cash flow \faIcon{money-bill-wave} (for \faIcon{user} aged
\(x\) at time 0), at time \(n\), is
\[
\text{APV of \faIcon{money-bill-wave} (at time 0)}\times\frac{1}{\Ex[n]{x}}.
\]
\begin{center}
\begin{tikzpicture}
\draw[-Latex] (0,0) -- (10,0) node[right]{Time};
\fill[] (0,0) circle [radius=0.05]
node[below] {0}
node[above] {\((x)\)};
\fill[] (3,0) circle [radius=0.05]
node[above=0.3cm] {\faIcon{money-bill-wave}};
\fill[] (5,0) circle [radius=0.05]
node[below] {\(n\)};
\draw[-Latex, ForestGreen] (2.8,0.8) to[bend right] (0,0.7);
\node[ForestGreen] () at (1.4,1.5) {actuarial discounting};
\draw[-Latex, ForestGreen] (0,-0.5) to[bend right] (5,-0.5);
\node[ForestGreen] () at (2.5,-1.7) {actuarial accumulation \(\qty(\times\frac{1}{\Ex[n]{x}})\)};
\end{tikzpicture}
\end{center}

\item It is natural to regard the pale cash flows below as ``past'' cash flows
(serving as ``dual'' for ``future'' cash flows):

\begin{tikzpicture}
\draw[-Latex] (0,0) -- (10,0) node[right]{Time};
\fill[] (0,0) circle [radius=0.05]
node[below=0.1cm] {0};
\fill[] (2,0) circle [radius=0.05]
node[below=0.1cm] {1};
\fill[] (4,0) circle [radius=0.05]
node[below=0.1cm] {\(t=2\)};
\fill[] (6,0) circle [radius=0.05]
node[below=0.1cm] {3};
\fill[] (8,0) circle [radius=0.05]
node[below=0.1cm] {4};
\draw[->, brown, line width=0.3mm, opacity=0.2] (-0.05,-1.2) -- (-0.05,-1.6)
node[pos=2] {\faIcon{dollar-sign}};
\draw[->, violet, line width=0.3mm, opacity=0.2] (0.05,-1.6) -- (0.05,-1.2)
node[pos=2] {\faIcon{hourglass-start}};
\draw[->, brown, line width=0.3mm, opacity=0.2] (1.95,-1.2) -- (1.95,-1.6)
node[pos=2] {\faIcon{dollar-sign}};
\draw[->, violet, line width=0.3mm, opacity=0.2] (2.05,-1.6) -- (2.05,-1.2)
node[pos=2] {\faIcon{redo}};
\draw[->, brown, line width=0.3mm] (3.95,-1.2) -- (3.95,-1.6)
node[pos=2] {\faIcon{dollar-sign}};
\draw[->, violet, line width=0.3mm] (4.05,-1.6) -- (4.05,-1.2)
node[pos=2] {\faIcon{redo}};
\draw[->, brown, line width=0.3mm] (5.95,-1.2) -- (5.95,-1.6)
node[pos=2] {\faIcon{dollar-sign}};
\draw[->, violet, line width=0.3mm] (6.05,-1.6) -- (6.05,-1.2)
node[pos=2] {\faIcon{redo}};
\draw[->, brown, line width=0.3mm] (7.95,-1.2) -- (7.95,-1.6)
node[pos=2] {\faIcon{dollar-sign}};
\draw[->, violet, line width=0.3mm] (8.05,-1.6) -- (8.05,-1.2)
node[pos=2] {\faIcon{redo}};


\draw[->, orange, line width=0.3mm, opacity=0.2] (-0.05,1.2) -- (-0.05,1.6)
node[pos=2] {\faIcon{money-bill-wave}};
\draw[->, violet, line width=0.3mm, opacity=0.2] (0.05,1.2) -- (0.05,1.6)
node[pos=2] {\faIcon{hourglass-end}};
\draw[->, orange, line width=0.3mm, opacity=0.2] (1.95,1.2) -- (1.95,1.6)
node[pos=2] {\faIcon{money-bill-wave}};
\draw[->, violet, line width=0.3mm, opacity=0.2] (2.05,1.2) -- (2.05,1.6)
node[pos=2] {\faIcon{hourglass-end}};
\draw[->, orange, line width=0.3mm, opacity=0.2] (3.95,1.2) -- (3.95,1.6)
node[pos=2] {\faIcon{money-bill-wave}};
\draw[->, violet, line width=0.3mm, opacity=0.2] (4.05,1.2) -- (4.05,1.6)
node[pos=2] {\faIcon{hourglass-end}};
\draw[->, orange, line width=0.3mm] (5.95,1.2) -- (5.95,1.6)
node[pos=2] {\faIcon{money-bill-wave}};
\draw[->, violet, line width=0.3mm] (6.05,1.2) -- (6.05,1.6)
node[pos=2] {\faIcon{hourglass-end}};
\draw[->, orange, line width=0.3mm] (7.95,1.2) -- (7.95,1.6)
node[pos=2] {\faIcon{money-bill-wave}};
\draw[->, violet, line width=0.3mm] (8.05,1.2) -- (8.05,1.6)
node[pos=2] {\faIcon{hourglass-end}};

\draw[dashed, ForestGreen] (-0.5,2.5) -- (-0.5,-2.5);
\draw[dashed, ForestGreen] (-0.5,-2.5) -- (2,-2.5);
\draw[dashed, ForestGreen] (-0.5,2.5) -- (5,2.5);
\draw[dashed, ForestGreen] (5,2.5) -- (2,-2.5);
\draw[-Latex, ForestGreen] (2,-3) -- (1,-0.5)
node[pos=-0.2]{``past'' cash flows};
\end{tikzpicture}

For ``past'' \emph{gain}, we should treat cash inflow as positive and cash
outflow as negative. It is negative of ``past'' loss.

\item Fix any time \(t\in\N_0\). Let \(L_{0,t}\) be the present value of the
``past'' loss (at time 0). Then the \defn{retrospective policy value} at time
\(t\), denoted by
\(\Vx[t]{}[R]\), is given by
\[
\Vx[t]{}[R]=\frac{-\expv{L_{0,t}}}{\Ex[t]{x}}
\]
(AAV of the ``past'' gain).

\item Naturally we would like to know when the prospective and retrospective
policy values at the same time point would coincide. (We are more interested in
studying retrospective policy values in such case.) The following result
suggests sufficient conditions for that:
\begin{proposition}
\label{prp:pro-retro-pv-equal}
Let \(\Vx[t]{}[P]\) and \(\Vx[t]{}[R]\) be the prospective and retrospective
policy values at time \(t\) respectively. Then, we have
\(\Vx[t]{}[P]=\Vx[t]{}[R]\) if
\begin{enumerate}
\item the premiums are determined by equivalence principle;

\item premium, prospective policy value, and retrospective policy value bases are the same.
\end{enumerate}
\end{proposition}
\begin{pf}
Assume the two conditions hold. Firstly, by equivalence principle we have
\(\Vx[0]{}[P]=0\). Now, as the bases are the same, we have
\[
\underbrace{\Vx[0]{}[P]}_{0}={\color{blue}\expv{L_{0,t}}}
+{\color{ForestGreen}\Ex[t]{x}}\,{\color{magenta}\Vx[t]{}[P]}
\]
by ``splitting'' the cash flows into ``past'' and ``future'' portions, and
``actuarial discounting''.

\begin{center}
\begin{tikzpicture}
\draw[-Latex] (0,0) -- (10,0) node[right]{Time};
\fill[] (0,0) circle [radius=0.05]
node[below=0.1cm] {0};
\fill[] (2,0) circle [radius=0.05]
node[below=0.1cm] {1};
\fill[] (4,0) circle [radius=0.05]
node[below=0.1cm] {\(t=2\)};
\fill[] (6,0) circle [radius=0.05]
node[below=0.1cm] {3};
\fill[] (8,0) circle [radius=0.05]
node[below=0.1cm] {4};
\draw[->, brown, line width=0.3mm, opacity=0.2] (-0.05,-1.2) -- (-0.05,-1.6)
node[pos=2] {\faIcon{dollar-sign}};
\draw[->, violet, line width=0.3mm, opacity=0.2] (0.05,-1.6) -- (0.05,-1.2)
node[pos=2] {\faIcon{hourglass-start}};
\draw[->, brown, line width=0.3mm, opacity=0.2] (1.95,-1.2) -- (1.95,-1.6)
node[pos=2] {\faIcon{dollar-sign}};
\draw[->, violet, line width=0.3mm, opacity=0.2] (2.05,-1.6) -- (2.05,-1.2)
node[pos=2] {\faIcon{redo}};
\draw[->, brown, line width=0.3mm] (3.95,-1.2) -- (3.95,-1.6)
node[pos=2] {\faIcon{dollar-sign}};
\draw[->, violet, line width=0.3mm] (4.05,-1.6) -- (4.05,-1.2)
node[pos=2] {\faIcon{redo}};
\draw[->, brown, line width=0.3mm] (5.95,-1.2) -- (5.95,-1.6)
node[pos=2] {\faIcon{dollar-sign}};
\draw[->, violet, line width=0.3mm] (6.05,-1.6) -- (6.05,-1.2)
node[pos=2] {\faIcon{redo}};
\draw[->, brown, line width=0.3mm] (7.95,-1.2) -- (7.95,-1.6)
node[pos=2] {\faIcon{dollar-sign}};
\draw[->, violet, line width=0.3mm] (8.05,-1.6) -- (8.05,-1.2)
node[pos=2] {\faIcon{redo}};


\draw[->, orange, line width=0.3mm, opacity=0.2] (-0.05,1.2) -- (-0.05,1.6)
node[pos=2] {\faIcon{money-bill-wave}};
\draw[->, violet, line width=0.3mm, opacity=0.2] (0.05,1.2) -- (0.05,1.6)
node[pos=2] {\faIcon{hourglass-end}};
\draw[->, orange, line width=0.3mm, opacity=0.2] (1.95,1.2) -- (1.95,1.6)
node[pos=2] {\faIcon{money-bill-wave}};
\draw[->, violet, line width=0.3mm, opacity=0.2] (2.05,1.2) -- (2.05,1.6)
node[pos=2] {\faIcon{hourglass-end}};
\draw[->, orange, line width=0.3mm, opacity=0.2] (3.95,1.2) -- (3.95,1.6)
node[pos=2] {\faIcon{money-bill-wave}};
\draw[->, violet, line width=0.3mm, opacity=0.2] (4.05,1.2) -- (4.05,1.6)
node[pos=2] {\faIcon{hourglass-end}};
\draw[->, orange, line width=0.3mm] (5.95,1.2) -- (5.95,1.6)
node[pos=2] {\faIcon{money-bill-wave}};
\draw[->, violet, line width=0.3mm] (6.05,1.2) -- (6.05,1.6)
node[pos=2] {\faIcon{hourglass-end}};
\draw[->, orange, line width=0.3mm] (7.95,1.2) -- (7.95,1.6)
node[pos=2] {\faIcon{money-bill-wave}};
\draw[->, violet, line width=0.3mm] (8.05,1.2) -- (8.05,1.6)
node[pos=2] {\faIcon{hourglass-end}};

\draw[dashed, blue] (-0.5,2.5) -- (-0.5,-2.5);
\draw[dashed, blue] (-0.5,-2.5) -- (2,-2.5);
\draw[dashed, blue] (-0.5,2.5) -- (5,2.5);
\draw[dashed, blue] (5,2.5) -- (2,-2.5);
\draw[-Latex, blue] (0,-3) -- (1,-0.5)
node[pos=-0.2]{APV @ time 0: \(\expv{L_{0,t}}\)};

\draw[dashed, magenta] (5.5,2.5) -- (2.5,-2.5);
\draw[dashed, magenta] (5.5,2.5) -- (10,2.5);
\draw[dashed, magenta] (2.5,-2.5) -- (10,-2.5);
\draw[-Latex, magenta] (6,-3) -- (5,-0.5)
node[pos=-0.2]{APV @ time \(t\): \(\Vx[t]{}[P]\)};
\draw[-Latex, ForestGreen] (4.8,0.3) to[bend right] (0,0.3);
\node[ForestGreen] () at (2.4,0.5) {\(\times\Ex[t]{x}\)};
\end{tikzpicture}
\end{center}
The result then follows by rearranging the terms.
\end{pf}
\end{enumerate}

