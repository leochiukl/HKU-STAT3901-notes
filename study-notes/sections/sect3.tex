\section{Life Insurance}
\label{sect:life-insurance}
\begin{enumerate}
\item The decision of whether a policy \faIcon{file-alt} should be sold the
policyholder \faIcon{user} or not depends critically on the \emph{present
value} for the cash flows involved in \faIcon{file-alt} (including both
benefits and premiums).
\item To improve the tractability, we shall assume that the annual interest
rate \(i\) is a fixed constant, so that the ``randomness'' of the present value
(random variable) would only source from the modelled survival distribution for
\faIcon{user} --- the present value is \emph{life contingent}.
\item Often the insurance business \faIcon{building} is of very
large scale and many identical policies may be sold to lives modelled by the
same survival distribution (the ``ordinary'' one, say).
\item In such case, the \emph{average} present value of those policies is a
key metric for the profitability per policy sold. By virtue of \emph{weak
law of large number} (assuming it is applicable), the average would converge to
the \emph{expected present value} as the number of such policies sold
goes to infinity. Hence, the study of life contingencies focuses a lot on the
expected present value (also known as \defn{actuarial present value} (APV)).
\item \Cref{sect:life-insurance,sect:life-annuity} mainly focus on determining
the APV of the benefit payments of the contracts mentioned in
\cref{subsect:traditional-contracts}.
\item The contracts can be classified into two big categories, by the payment
``timing'':
\begin{enumerate}
\item Discrete time: insurance \(\rightarrow\) payable at the end of period;
life annuity \(\rightarrow\) payable at the beginning of period. (\(K_x\) is
involved mainly.)
\item Continuous time: insurance \(\rightarrow\) payable ``immediately''; life
annuity \(\rightarrow\) payable ``continuously'' (\emph{continuous annuity} in
STAT2902). (\(T_x\) is involved mainly.)
\end{enumerate}
It turns out that working in continuous time (working with \(T_x\)) is
mathematically more ``convenient''. However, practical contracts are always
discrete in nature. So, sometimes quantities for ``discrete'' contracts in
practice are approximated through working in continuous time instead, for
mathematical convenience. (The approximation works better if the measurement
period for the contract is shorter.)

\item
\label{it:gen-apv-fmla}
\ystar An ``intuition-based'' general APV calculation formula for
insurance/life annuity:
\[
\underset{\text{all possible payment times}}{\sum\text{ or }\int}\text{benefit amount}\times\text{discount
factor}\times\text{prob.\ of triggering payment}.
\]
\begin{note}
For continuous case with ``\(\int\)'', the terms above can be in
``infinitesimal'' sense (e.g., involving ``\(dt\)''), loosely.
\end{note}


\begin{intuition}
It ``sums up'' the ``expected present value of benefit payment'':
\[
\underbrace{\text{benefit amount}\times\text{discount factor}}_{\text{PV of payment}}
\times\text{prob.\ of triggering payment}
\]
in all time intervals.
\end{intuition}
\end{enumerate}

\subsection{Whole Life Insurance}
\label{subsect:whole-life}
\begin{enumerate}
\subsubsection*{Continuous Case}
\item Consider a \emph{continuous} whole life insurance with sum insured 1
and life insured (\(x\)). The sum insured 1 is paid \emph{exactly at} time of
death of (\(x\)), i.e., time \(T_x\) (when the life is aged \(x\), it is time
0).  \item The present value random variable (p.v.r.v.) (present value of the
benefit payment) is
\[
Z=v^{T_x}=e^{-\delta T_x}.
\]
where \(\delta\) is the annual force of interest equivalent to the annual rate
\(i\). The APV is:
\[
\expv{Z}=\expv{e^{-\delta T_x}}=\int_{0}^{\infty}e^{-\delta t}\underbrace{\px[t]{x}\mu_{x+t}}_{f_x(t)}\,dt
\]
\begin{intuition}
RHS ``sums up'' the ``expected present value of death benefit'':
\[\underbrace{e^{-\delta t}}_{\text{PV of payment}}\times \underbrace{\px[t]{x}\mu_{x+t}\,dt}_{\text{death prob. in }[t,t+dt]}\]
(\(e^{-\delta(t+dt)}\) and \(e^{-\delta t}\) are ``the same'' in
``infinitesimal'' sense) in all ``infinitesimal'' time intervals.
\end{intuition}
\item Actuarial notation for the APV:

\begin{tikzpicture}
\node[font=\huge] (Ax) at (0,0) {\defn{\(\Ax*{x}\)}};
\draw[-Latex] (-1,1) -- (0,0.5);
\node[] () at (-1.5,1.3)  {``bar'' indicates ``continuous''};
\draw[-Latex] (1.5,1) -- (0.25,-0.1);
\node[] () at (1.8,1.3) {for (\(x\))};
\draw[-Latex] (-1.5,-1) -- (-0.35,-0.35);
\node[text width=3.5cm] () at (-1.8,-2.2) {\underline{A}ssurance (another term for
``insurance'', which is more commonly used in the UK)};
\end{tikzpicture}

\item Sometimes we are also interested in the \emph{variability} of the
p.v.r.v.\ \(Z\). To measure this, we calculate the variance \(\vari{Z}\).
\item Consider first the second moment of \(Z\), which is given by
\[
\expv{Z^2}=\expv{e^{-{\color{purple} 2\delta} T_x}}=\Ax*{x}@\;{\color{purple} 2\delta},
\]
i.e., the APV evaulated at force of interest \(2\delta\) (Actuarial notation:
\defn{\(\Ax*[][2]{x}\)}). Then the variance is
\[
\vari{Z}=\Ax*[][2]{x}-(\Ax*{x})^2.
\]
\item When the sum insured for the insurance becomes \( S\), the p.v.r.v.\
becomes \(S\cdot Z\). Then, the APV becomes \(\expv{SZ}=S\Ax*{x}\), the second
moment becomes \(\expv{(SZ)^2}=S^{\color{purple}2}\cdot\Ax*[][2]{x}\), and the
variance is \(\vari{SZ}=S^{\color{purple}2}\vari{Z}\).
\begin{warning}
Do not miss the square for the second moment and variance!
\end{warning}
\item To compute \emph{probability} related to the p.v.r.v. in the form
of \(\prob{Z\le \text{\faIcon{question-circle}}}\), we try to find an
expression of the form
``\(T_x\ge\text{\faIcon[regular]{question-circle}}\)''\footnote{The direction
of inequality is given by ``\(\ge\)'' since, loosely, \(T_x\uparrow\iff
Z\downarrow\), for a fixed insurance contract.}  that is \emph{equivalent} to
``\(Z\le \text{\faIcon{question-circle}}\)''. Then, we instead compute the
probability \(\prob{T_x\ge\text{\faIcon[regular]{question-circle}}}\) (which
equals \(\prob{Z\le \text{\faIcon{question-circle}}}\)). We have similar
approaches for other ``forms'' of the probability.
\begin{note}
This works generally for all kinds of insurance introduced in
\cref{sect:life-insurance}.
\end{note}
\item
\label{it:cts-whole-life-fmlas}
 Summary:

\begin{tabular}{ccccc}
\toprule
&p.v.r.v.&APV&2nd moment&variance\\
\midrule
expression&\(e^{-\delta T_x}\)&\(\displaystyle\int_{0}^{\infty}e^{-\delta t}\px[t]{x}\mu_{x+t}\,dt\)
&\(\Ax*{x}@\;2\delta\)&\(\Ax*[][2]{x}-(\Ax*{x})^2\)\\
notation&\(Z\)&\(\Ax*{x}\)&\(\Ax*[][2]{x}\)&\(\vari{Z}\)\\
\bottomrule
\end{tabular}
\subsubsection*{Annual Case}
\item Consider a \emph{discrete} whole life insurance with sum insured
\faIcon{money-bill-wave} 1 and life insured (\(x\)). The sum insured 1 is
paid \emph{at the end of (policy) year} of death of (\(x\)), i.e., time
\(K_x+1\).

\begin{tikzpicture}
\draw[-Latex] (0,0) -- (10,0) node[right]{Time};
\fill[] (0,0) circle [radius=0.05]
node[below] {0}
node[above] {\((x)\)};
\fill[] (2,0) circle [radius=0.05]
node[below] {1};
\fill[] (4,0) circle [radius=0.05]
node[below] {\(K_x\)};
\fill[] (4.8,0) circle [radius=0.05]
node[red, above] {\faIcon{skull}}
node[below] {\(T_x\)};
\fill[] (6,0) circle [radius=0.05]
node[ForestGreen, above] {\faIcon{money-bill-wave}}
node[ForestGreen, above=0.4cm] {1}
node[below] {\(K_x+1\)};
\node[] () at (11,-1) {Policy year};
\draw[very thick, decorate,decoration={calligraphic brace, amplitude=5pt, raise=15pt, mirror}] (0,0) -- (2,0)
node[midway, below=0.7cm]{1};
\draw[very thick, decorate,decoration={calligraphic brace, amplitude=5pt, raise=15pt, mirror}] (4,0) -- (6,0)
node[midway, below=0.7cm]{\(K_x\)};
\end{tikzpicture}
\item 
\label{it:annual-whole-life-fmlas}
\begin{tabular}{ccccc}
\toprule
&p.v.r.v.&APV&2nd moment&variance\\
\midrule
expression&\(v^{K_x+1}\)&\(\displaystyle\sum_{k=0}^{\infty}v^{k+1}\px[k]{x}\qx{x+k}\)
&\(\Ax{x}@\;2\delta\)&\(\Ax[][2]{x}-(\Ax{x})^2\)\\
notation&\(Z\)&\defn{\(\Ax{x}\)}&\defn{\(\Ax[][2]{x}\)}&\(\vari{Z}\)\\
\bottomrule
\end{tabular}

\begin{rmk}
\item The actuarial notation \(\Ax{x}\) has no ``bar'' as the insurance is no longer
continuous.
\item We can change ``\(v\)'' to ``\(e^{-\delta}\)'' above, where \(\delta\) is the
equivalent annual force of interest.
\end{rmk}

\begin{intuition}
For the APV formula, it sums up the expected present value of death benefit \faIcon{money-bill-wave}:
\[\underbrace{v^{k+1}}_{\text{PV of payment}}\times \underbrace{\px[k]{x}\qx{x+k}}_{\text{death prob. in }[k,k+1)}\]
in all ``unit'' time intervals.
\end{intuition}
\subsubsection*{\(1/m\)thly Case}
\item Consider a \emph{discrete} whole life insurance with sum insured
\faIcon{money-bill-wave} 1 and life insured (\(x\)). The sum insured 1 is
paid \emph{at the end of \(1/m\)th of a (policy) year} of death of (\(x\)).
\item Since ``\(1/m\)th of a (policy) year'' is involved, it would be more
convenient to develop a similar concept as the curtate future lifetime random
variable \(K_x\) but with time unit being \(1/m\)th of a year.
\item The \defn{\(1/m\)thly curtate future lifetime random variable} (denoted
by \defn{\(K_x^{(m)}\)}) is \(T_x\) rounded down to 1/\(m\)th of a year, i.e.,
\[
K_x^{(m)}=\frac{1}{m}\lfloor mT_x\rfloor.
\]
\begin{tikzpicture}
\draw[-Latex] (0,0) -- (10,0) node[right]{Time};
\fill[] (0,0) circle [radius=0.05]
node[above] {(\(x\))}
node[below] {0};
\fill[] (1,0) circle [radius=0.05];
\fill[] (2,0) circle [radius=0.05]
node[below] {1};
\fill[] (3,0) circle [radius=0.05];
\fill[] (4,0) circle [radius=0.05]
node[below] {2};
\fill[] (5,0) circle [radius=0.05];
\fill[] (6,0) circle [radius=0.05]
node[below] {3};
\fill[] (7,0) circle [radius=0.05];
\fill[] (8,0) circle [radius=0.05]
node[below] {4};
\draw[thick, decorate,decoration={calligraphic brace, amplitude=5pt, raise=15pt, mirror}] (1,0) -- (1.99,0)
node[midway, below=0.7cm, font=\small] {\(K_x^{(1/2)}=0.5\)};
\draw[thick, decorate,decoration={calligraphic brace, amplitude=5pt, raise=15pt, mirror}] (4,0) -- (4.99,0)
node[midway, below=0.7cm, font=\small] {\(K_x^{(1/2)}=2\)};
\draw[thick, decorate,decoration={calligraphic brace, amplitude=5pt, raise=15pt, mirror}] (7,0) -- (7.99,0)
node[midway, below=0.7cm, font=\small] {\(K_x^{(1/2)}=3.5\)};
\end{tikzpicture}
\item The random variable \(K_x^{(m)}\) is still discrete, and its pmf is:
\[
\prob{K_x^{(m)}=\frac{k}{m}}=\qx[\left.\frac{k}{m}\right|\frac{1}{m}]{x}
=\px[\frac{k}{m}]{x}\,\qx[\frac{1}{m}]{x+\frac{k}{m}},
\]
for any \(k\in\N_0\).
\item 
\label{it:1m-whole-life-fmlas}
\begin{tabular}{ccccc}
\toprule
&p.v.r.v.&APV&2nd moment&variance\\
\midrule
expression&\(v^{\color{brown}K_x^{(m)}+1/m}\)
&\(\displaystyle\sum_{k=0}^{\infty}v^{\color{brown}\frac{k+1}{m}}\px[\color{brown}\frac{k}{m}]{x}\,\qx[\color{brown}\frac{1}{m}]{x+{\color{brown}\frac{k}{m}}}\)
&\(\Ax{x}[(m)]@\;2\delta\)&\(\Ax[][2]{x}[(m)]-\qty(\Ax{x}[(m)])^2\)\\
notation&\(Z\)&\defn{\(\Ax{x}[(m)]\)}&\defn{\(\Ax[][2]{x}[(m)]\)}&\(\vari{Z}\)\\
\bottomrule
\end{tabular}
\end{enumerate}
\subsection{(Pure) Endowment and Term Life Insurances}
\label{subsect:pe-tl}
\begin{enumerate}
\item The following shows the relationship between different kinds of insurance
(in one approach of ``construction''):

\begin{tikzpicture}
\node[draw, brown] (wl) at (0,0) {whole life};
\node[draw, brown] (pe) at (5,0) {pure endowment};
\node[draw, purple] (dl) at (3,-4.2) {deferred contract};
\node[draw, violet] (tl) at (1,-2) {term life};
\node[draw, violet] (en) at (4,-2) {endowment};
\draw[pen colour=orange, thick, decorate,decoration={calligraphic brace, amplitude=10pt, raise=15pt, mirror}] (-1,-2) -- (7,-2)
node[purple, midway, below=1cm, draw] (vb) {variable benefits (different kinds)};
\draw[-Latex] (wl) -- (tl);
\draw[-Latex] (pe) -- (tl);
\draw[-Latex] (wl) -- (en);
\draw[-Latex] (pe) -- (en);
\node[brown] () at (9,0) {basic building blocks};
\node[violet] () at (9,-2) {``constructed'' contracts};
\node[purple] () at (9,-3.7) {variants};
\end{tikzpicture}
\subsubsection*{Pure Endowment}
\item Consider an \(n\)-year pure endowment contract with survival benefit 1,
i.e., 1 is payable at time \(n\) if the life insured is alive at that time.

\begin{note}
\(n\) is usually an integer, but not necessarily. This applies to all
\(n\)-year contracts.
\end{note}

\item \label{it:pure-endow-fmlas}
\begin{tabular}{ccccc}
\toprule
&p.v.r.v.&APV&2nd moment&variance\\
\midrule
expression&\(v^{n}\indicset{T_x>n}\)&\(v^n\px[n]{x}\)
&\(\Ex[n]{x}@\;2\delta\)&\(\Ex[n][2]{x}-(\Ex[n]{x})^2\)\\
notation&\(Z\)&\defn{\(\Ex[n]{x}\)} or \defn{\(\Ax{\pureendowxn}\)}
&\defn{\(\Ex[n][2]{x}\)} or \defn{\(\Ax[][2]{\pureendowxn}\)}
&\(\vari{Z}\)\\
\bottomrule
\end{tabular}

\begin{rmk}
\item The ``1'' on top of \(\angl{n}\) suggests that the \(n\)-year term is the
\emph{1st} thing to be ``gone'' (earllier than \((x)\) being ``gone''/dead) for
the benefit to be triggered\footnote{In other words, the life insured must be alive at
time \(n\) to trigger the benefit.}. More details about this kind of notation
will be discussed in STAT3909.
\item There are not concepts like ``continuous'' and ``discrete'' for pure
endowment --- the benefit is always paid (potentially) at time \(n\) for an
\(n\)-year pure endowment. Conventionally we do \emph{not} write
``\(\Ax*{x:\itop{\angl{n}}}\)''.
\end{rmk}
\item The following provides some ``shortcuts'' for the formulas in
\labelcref{it:pure-endow-fmlas}:
\begin{proposition}
\label{prp:pure-endow-shortcut}
For any age \(x\) and any \(n\ge 0\),
\begin{enumerate}
\item \(\Ex[n][2]{x}=v^n\Ex[n]{x}\);
\item \(\vari{Z}=v^{2n}\px[n]{x}\,\qx[n]{x}\).
\end{enumerate}
\end{proposition}
\begin{pf}
Firstly, \(\Ex[n][2]{x}=v^{2n}\px[n]{x}=v^n\Ex[n]{x}\). Next,
\[\vari{Z}=\Ex[n][2]{x}-(\Ex[n]{x})^2=v^{2n}\px[n]{x}(1-\px[n]{x})
=v^{2n}\px[n]{x}\,\qx[n]{x}.\]
\end{pf}
\subsubsection*{Term Life Insurance}
\item Consider an \(n\)-year term life insurance with death benefit \(1\),
i.e., 1 is only payable if the life insured dies within \(n\) years.
\item \label{it:cts-term-life-fmlas}
Continuous case:

\begin{tabular}{ccccc}
\toprule
&p.v.r.v.&APV&2nd moment&variance\\
\midrule
expression&\(e^{-\delta T_x}\indicset{T_x\le n}\)&\(\Ax*{x}-\Ex[n]{x}\Ax*{x+n}\)
or \(\displaystyle\int_{0}^{n}e^{-\delta t}\px[t]{x}\mu_{x+t}\,dt\)
&\(\Ax*{\termxn}@\;2\delta\)&\(\Ax*[][2]{\termxn}-\qty(\Ax*{\termxn})^2\)\\
notation&\(Z\)&\defn{\(\Ax*{\termxn}\)}
&\defn{\(\Ax*[][2]{\termxn}\)}
&\(\vari{Z}\)\\
\bottomrule
\end{tabular}

\begin{pf}
To get the first APV formula, note that \(Z=\underbrace{e^{-\delta T_x}}_{\text{whole life}}-v^ne^{-\delta
(T_x-n)}\indicset{T_x>n}\), so
\[
\expv{Z}=\Ax*{x}+v^n\prob{T_x>n}\expv{\left. e^{-\delta({\color{brown}T_x-n})}\right|{\color{brown}T_x>n}}
=\Ax*{x}+v^n\px[n]{x}\underbrace{\expv{e^{-\delta({\color{brown}T_{x+n}})}}}_{\Ax*{x+n}}.
\]
For the integral APV formula, it follows directly from the expression for p.v.r.v.
\end{pf}

\begin{intuition}
\(\Ex[n]{x}\) is like an \defn{``actuarial discount factor''}:

\begin{tikzpicture}
\draw[-Latex] (0,0) -- (10,0) node[right]{Time};
\fill[] (0,0) circle [radius=0.05]
node[below] {0}
node[above] {\((x)\)};
\fill[] (3,0) circle [radius=0.05]
node[below] {\(n\)};
\draw[very thick, decorate,decoration={mirror, calligraphic brace, amplitude=5pt, raise=15pt}] (0,0) -- (3,0)
node[midway, below=0.7cm]{\(\Ax*{\termxn}\)};
\draw[pen colour=red, very thick, decorate,decoration={calligraphic brace, amplitude=5pt, raise=25pt}] (3,0) -- (10,0)
node[red, midway, above=1cm]{\(\Ax*{x+n}\)};
\draw[pen colour=violet, very thick, decorate,decoration={calligraphic brace, amplitude=5pt, raise=15pt}] (0,0) -- (10,0)
node[violet, pos=0.2, above=0.7cm]{\(\Ax*{x}\)};
\draw[thick, decorate,decoration={brace, amplitude=5pt, raise=5pt}] (0,1.5) -- (0,2)
node[midway, left=0.5cm] {take away};
\draw[-Latex, color=ForestGreen] (3,1) to[bend right] (0,1.6);
\draw[-Latex, color=ForestGreen] (10,1) to[bend right] (0,1.9);
\node[] () at (2.5,2) {\(\times\Ex[n]{x}\)};
\end{tikzpicture}
\end{intuition}

\begin{note}
As we shall see in \cref{subsect:defer-var-insurance}, \(\Ex[n]{x}\Ax*{x+n}\)
is the APV of a continuous \(n\)-year deferred whole life insurance. So this
formula for finding APV of term life insurance also suggests another formula
for finding APV of \emph{deferred} whole life insurance, by rearranging the
terms.
\end{note}


\item \label{it:annual-term-life-fmlas}
Annual case:

\begin{tabular}{ccccc}
\toprule
&p.v.r.v.&APV&2nd moment&variance\\
\midrule
expression&\(v^{K_x+1}\indicset{K_x\le n-1}\)&\(\Ax{x}-\Ex[n]{x}\Ax{x+n}\)
or \(\displaystyle\sum_{k=0}^{n-1}v^{k+1}\px[k]{x}\qx{x+k}\)
&\(\Ax{\termxn}@\;2\delta\)&\(\Ax[][2]{\termxn}-\qty(\Ax{\termxn})^2\)\\
notation&\(Z\)&\defn{\(\Ax{\termxn}\)}
&\defn{\(\Ax[][2]{\termxn}\)}
&\(\vari{Z}\)\\
\bottomrule
\end{tabular}

\begin{pf}
Similar to the proof for \labelcref{it:cts-term-life-fmlas}.
\end{pf}

\item \label{it:1m-term-life-fmlas}
\(1/m\)thly case:

\begin{tabular}{ccccc}
\toprule
&p.v.r.v.&APV&2nd moment&variance\\
\midrule
expression&\(v^{K_x^{(m)}+\frac{1}{m}}\indicset{K_x^{(m)}\le n-\frac{1}{m}}\)
&\makecell{\(\Ax{x}[(m)]-\Ex[n]{x}\Ax{x+n}[(m)]\)\\
or \(\displaystyle\sum_{k=0}^{mn-1}v^{\frac{k+1}{m}}\px[\frac{k}{m}]{x}\,\,\qx[\frac{1}{m}]{x+\frac{k}{m}}\)}
&\(\Ax{\termxn}@\;2\delta\)&\(\Ax[][2]{}[(m)]{}_{\termxn}-\qty(\Ax{}[(m)]{}_{\termxn})^2\)\\
notation&\(Z\)&\defn{\(\Ax{}[(m)]{}_{\termxn}\)}
&\defn{\(\Ax[][2]{}[(m)]{}_{\termxn}\)}
&\(\vari{Z}\)\\
\bottomrule
\end{tabular}

\begin{pf}
Similar to the proof for \labelcref{it:cts-term-life-fmlas}.
\end{pf}
\subsubsection*{Endowment Insurance}
\item \label{it:endow-term-plus-pure} A key observation is that \(n\)-year
endowment insurance is just \(n\)-year term life insurance + \(n\)-year pure
endowment.
\item As a result, the p.v.r.v.'s in different cases can be written as:

\begin{tabular}{cc}
\toprule
case&p.v.r.v.\\
\midrule
continuous&\(e^{-\delta T_x}\indicset{T_x\le n}+e^{-\delta n}\indicset{T_x>n}\)\\
annual&\(v^{K_x+1}\indicset{K_x\le n-1}+v^n\indicset{K_x\ge n}\)\\
\(1/m\)thly&\(v^{K_x^{(m)}+\frac{1}{m}}\indicset{K_x^{(m)}\le n-\frac{1}{m}}+v^n\indicset{K_x^{(m)}\ge n}\)\\
\bottomrule
\end{tabular}

\begin{note}
Both events \(\{K_x\ge n\}\) and \(\qty{K_x^{(m)}\ge n}\) have the same probability as
the event \(\{T_x> n\}\).
\end{note}
\item But actually they can be written more compactly as follows:

\begin{tabular}{cc}
\toprule
case&p.v.r.v.\\
\midrule
continuous&\(e^{-\delta(T_x\wedge n)}\)\\
annual&\(v^{(K_x+1)\wedge n}\)\\
\(1/m\)thly&\(v^{\qty(K_x^{(m)}+\frac{1}{m})\wedge n}\)\\
\bottomrule
\end{tabular}
\item \label{it:cts-endow-fmlas}
Continuous case:

\begin{tabular}{ccccc}
\toprule
&p.v.r.v.&APV&2nd moment&variance\\
\midrule
expression&\(e^{-\delta(T_x\wedge n)}\)&\(\Ax*{\termxn}+\Ex[n]{x}\)
&\(\Ax*{\endowxn}@\;2\delta\)&\(\Ax*[][2]{\endowxn}-\qty(\Ax*{\endowxn})^2\)\\
notation&\(Z\)&\defn{\(\Ax*{\endowxn}\)}
&\defn{\(\Ax*[][2]{\endowxn}\)}
&\(\vari{Z}\)\\
\bottomrule
\end{tabular}

\begin{note}
When there is not ``1'' on top of either \(x\) or \(\angl{n}\), it indicates
that a benefit payment is triggered when \emph{any} one of them is ``gone''.
Again more details will be discussed in STAT3909.
\end{note}

\item \label{it:annual-endow-fmlas}
Annual case:

\begin{tabular}{ccccc}
\toprule
&p.v.r.v.&APV&2nd moment&variance\\
\midrule
expression&\(v^{(K_x+1)\wedge n}\)&\(\Ax{\termxn}+\Ex[n]{x}\)
&\(\Ax{\endowxn}@\;2\delta\)&\(\Ax[][2]{\endowxn}-\qty(\Ax{\endowxn})^2\)\\
notation&\(Z\)&\defn{\(\Ax{\endowxn}\)}
&\defn{\(\Ax[][2]{\endowxn}\)}
&\(\vari{Z}\)\\
\bottomrule
\end{tabular}

\item \label{it:1m-endow-fmlas}
\(1/m\)thly case:

\begin{tabular}{ccccc}
\toprule
&p.v.r.v.&APV&2nd moment&variance\\
\midrule
expression&\(v^{\qty(K_x^{(m)}+\frac{1}{m})\wedge n}\)&\(\Ax{}[(m)]{}_{\termxn}+\Ex[n]{x}\)
&\(\Ax{\endowxn}[(m)]@\;2\delta\)&\(\Ax[][2]{\endowxn}[(m)]-\qty(\Ax{\endowxn}[(m)])^2\)\\
notation&\(Z\)&\defn{\(\Ax{\endowxn}[(m)]\)}
&\defn{\(\Ax[][2]{\endowxn}[(m)]\)}
&\(\vari{Z}\)\\
\bottomrule
\end{tabular}

\item Due to the property in \labelcref{it:endow-term-plus-pure}, we have an
alternative method for calculating the \emph{variance} for p.v.r.v. for
endowment insurance, as follows:
\begin{proposition}
\label{prp:endow-var-alt-fmla}
Write \(Z=Z_1+Z_2\) where \(Z,Z_1,Z_2\) are the p.v.r.v.'s of the \(n\)-year
endowment, term life, and pure endowment insurances respectively. Then,
\begin{itemize}
\item (continuous case) \(\vari{Z}=\vari{Z_1}+\vari{Z_2}-2\Ax*{\termxn}\,\Ex[n]{x}\);
\item (annual case) \(\vari{Z}=\vari{Z_1}+\vari{Z_2}-2\Ax{\termxn}\,\Ex[n]{x}\);
\item (\(1/m\)thly case) \(\vari{Z}=\vari{Z_1}+\vari{Z_2}-2\Ax{}[(m)]{}_{\termxn}\,\Ex[n]{x}\).
\end{itemize}
\end{proposition}
\begin{pf}
First note that in any situation, one of \(Z_1\) and \(Z_2\) would be 0 (a life
either dies within or survive for \(n\) years). So, the product \(Z_1Z_2\) is
always 0, and hence the covariance term
\[\cov{Z_1,Z_2}=\expv{Z_1Z_2}-\expv{Z_1}\expv{Z_2}=-\expv{Z_1}\expv{Z_2}.\]
Now the result follows from the formula
\[
\vari{Z}=\vari{Z_1}+\vari{Z_2}+2\cov{Z_1,Z_2}.
\]
\end{pf}
\end{enumerate}
\subsection{Deferred and Variable Benefits Insurances}
\label{subsect:defer-var-insurance}
\begin{enumerate}
\item Here we discuss two kinds of variants for insurance contracts discussed
in \cref{subsect:whole-life,subsect:pe-tl}: deferred and variable
benefits contracts.
\subsubsection*{Deferred Insurances}
\item There are 3 main types of deferred insurances of interest here:
\begin{itemize}
\item deferred whole life
\item deferred term life
\item deferred endowment
\end{itemize}
\item In general, the APV formulas here can be developed using the ``actuarial
discount factor'' intuition, as we can simply ``actuarially discount'' the
coverage for a life aged \(x+u\) back now to get ``deferred coverage'' for the
same life but aged \(x\).

\begin{tikzpicture}
\draw[-Latex] (0,0) -- (10,0) node[right]{Time};
\fill[] (0,0) circle [radius=0.05]
node[below] {0}
node[above] {\((x)\)};
\fill[] (3,0) circle [radius=0.05]
node[below=0.05cm, font=\large] {\(u\)}
node[above] {\((x+u)\)};
\draw[very thick, decorate,decoration={calligraphic brace, amplitude=5pt, raise=15pt}] (3,0) -- (10,0)
node[midway, above=0.7cm]{\(\Ax*{x+u}\)};
\draw[-Latex, color=ForestGreen] (3,0.6) to[bend right] (0,0.6);
\draw[-Latex, color=ForestGreen] (10,0.6) to[bend right] (0,0.9);
\node[] () at (2.5,1.3) {\(\times\Ex[u]{x}\)};
\draw[-Latex] (3,-0.7) -- (6,0.3);
\node[] () at (3,-1) {deferred coverage};
\end{tikzpicture}

But of course, one can also use the general APV formula in \labelcref{it:gen-apv-fmla} to develop them.
\item For a \(u\)-year deferred whole life insurance, we have:
\begin{enumerate}
\item continuous case:
\label{it:deferred-cts-whole-life-fmlas}

\begin{tabular}{ccccc}
\toprule
&p.v.r.v.&APV&2nd moment&variance\\
\midrule
expression&\(e^{-\delta T_x}\indicset{T_x>u}\)
&\(\Ex[u]{x}\Ax*{x+u}\) or \(\displaystyle\int_{u}^{\infty}e^{-\delta t}\px[t]{x}\mu_{x+t}\,dt\)
&\(\Ax*[u|]{x}@\;2\delta\)&\(\Ax*[u|][2]{x}-(\Ax*[u|]{x})^2\)\\
notation&\(Z\)&\defn{\(\Ax*[u|]{x}\)}&\defn{\(\Ax*[u|][2]{x}\)}&\(\vari{Z}\)\\
\bottomrule
\end{tabular}
\item annual case:
\label{it:deferred-annual-whole-life-fmlas}

\begin{tabular}{ccccc}
\toprule
&p.v.r.v.&APV&2nd moment&variance\\
\midrule
expression&\(v^{K_x+1}\indicset{K_x\ge u}\)
&\(\Ex[u]{x}\Ax{x+u}\) or \(\displaystyle\sum_{k=u}^{\infty}v^{k+1}\px[k]{x}\qx{x+k}\)
&\(\Ax[u|]{x}@\;2\delta\)&\(\Ax[u|][2]{x}-(\Ax[u|]{x})^2\)\\
notation&\(Z\)&\defn{\(\Ax[u|]{x}\)}&\defn{\(\Ax[u|][2]{x}\)}&\(\vari{Z}\)\\
\bottomrule
\end{tabular}
\item \(1/m\)thly case:
\label{it:deferred-1m-whole-life-fmlas}

\begin{tabular}{ccccc}
\toprule
&p.v.r.v.&APV&2nd moment&variance\\
\midrule
expression&\(v^{K_x^{(m)}+\frac{1}{m}}\indicset{K_x^{(m)}\ge u}\)
&\(\Ex[u]{x}\Ax{x+u}[(m)]\) or \(\displaystyle\sum_{k=mu}^{\infty}v^{\frac{k+1}{m}}\px[\frac{k}{m}]{x}\,\qx[\frac{1}{m}]{x+\frac{k}{m}}\)
&\(\Ax[u|]{x}[(m)]@\;2\delta\)&\(\Ax[u|][2]{x}[(m)]-\qty(\Ax[u|]{x}[(m)])^2\)\\
notation&\(Z\)&\defn{\(\Ax[u|]{x}[(m)]\)}&\defn{\(\Ax[u|][2]{x}[(m)]\)}&\(\vari{Z}\)\\
\bottomrule
\end{tabular}
\end{enumerate}

\item For a \(u\)-year deferred \(n\)-year term life insurance, we have:
\begin{enumerate}
\item continuous case:
\label{it:deferred-cts-term-life-fmlas}

\begin{tabular}{ccccc}
\toprule
&p.v.r.v.&APV&2nd moment&variance\\
\midrule
expression&\(e^{-\delta T_x}\indicset{u<T_x\le u+n}\)
&\makecell{\(\Ex[u]{x}\Ax*{\itop{(x+u)}:\angl{n}}\)\\ or \(\displaystyle\int_{u}^{u+n}e^{-\delta t}\px[t]{x}\mu_{x+t}\,dt\)}
&\(\Ax*[u|]{\termxn}@\;2\delta\)&\(\Ax*[u|][2]{\termxn}-(\Ax*[u|]{\termxn})^2\)\\
notation&\(Z\)&\defn{\(\Ax*[u|]{\termxn}\)}&\defn{\(\Ax*[u|][2]{\termxn}\)}&\(\vari{Z}\)\\
\bottomrule
\end{tabular}
\item annual case:
\label{it:deferred-annual-term-life-fmlas}

\begin{tabular}{ccccc}
\toprule
&p.v.r.v.&APV&2nd moment&variance\\
\midrule
expression&\(v^{K_x+1}\indicset{u\le K_x\le u+n-1}\)
&\makecell{\(\Ex[u]{x}\Ax{\itop{(x+u)}:\angl{n}}\)\\ or \(\displaystyle\sum_{k=u}^{u+n-1}v^{k+1}\px[k]{x}\qx{x+k}\)}
&\(\Ax[u|]{\termxn}@\;2\delta\)&\(\Ax[u|][2]{\termxn}-(\Ax[u|]{\termxn})^2\)\\
notation&\(Z\)&\defn{\(\Ax[u|]{\termxn}\)}&\defn{\(\Ax[u|][2]{\termxn}\)}&\(\vari{Z}\)\\
\bottomrule
\end{tabular}
\item \(1/m\)thly case:
\label{it:deferred-1m-term-life-fmlas}

\begin{tabular}{ccc}
\toprule
&p.v.r.v.&APV\\
\midrule
expression&\(v^{K_x^{(m)}+\frac{1}{m}}\indicset{u\le K_x^{(m)}\le u+n+\frac{1}{m}}\)
&\(\Ex[u]{x}\Ax{}[(m)] {}_{\itop{(x+u)}:\angl{n}}\) or \(\displaystyle\sum_{k=mu}^{mu+mn-1}v^{\frac{k+1}{m}}\px[\frac{k}{m}]{x}\,\qx[\frac{1}{m}]{x+\frac{k}{m}}\)\\
notation&\(Z\)&\defn{\(\Ax[u|]{}[(m)]{}_{\termxn}\)}\\
\bottomrule
\toprule
&2nd moment&variance\\
\midrule
expression&
\(\Ax[u|]{}[(m)]{}_{\termxn}@\;2\delta\)&\(\Ax[u|][2]{}[(m)]{}_{\termxn}-\qty(\Ax[u|]{}[(m)]{}_{\termxn})^2\)\\
notation&\defn{\(\Ax[u|][2]{}[(m)]{}_{\termxn}\)}&\(\vari{Z}\)\\
\bottomrule
\end{tabular}
\end{enumerate}

\item For a \(u\)-year deferred \(n\)-year endowment insurance, we have:
\begin{enumerate}
\item continuous case:
\label{it:deferred-cts-endowment-fmlas}

\begin{tabular}{ccccc}
\toprule
&p.v.r.v.&APV&2nd moment&variance\\
\midrule
expression&
\(e^{-\delta T_x}\indicset{u<T_x\le u+n}+e^{-\delta(u+n)}\indicset{T_x>u+n}\)
&\(\Ax*[u|]{\termxn}+\Ex[u+n]{x}\)
&\(\Ax*[u|]{\endowxn}@\;2\delta\)&\(\Ax*[u|][2]{\endowxn}-(\Ax*[u|]{\endowxn})^2\)\\
notation&\(Z\)&\defn{\(\Ax*[u|]{\endowxn}\)}&\defn{\(\Ax*[u|][2]{\endowxn}\)}&\(\vari{Z}\)\\
\bottomrule
\end{tabular}
\item annual case:
\label{it:deferred-annual-endowment-fmlas}

\begin{tabular}{ccccc}
\toprule
&p.v.r.v.&APV&2nd moment&variance\\
\midrule
expression&
\(v^{K_x+1}\indicset{u\le K_x\le u+n-1}+v^{u+n}\indicset{K_x\ge u+n}\)
&\(\Ax[u|]{\termxn}+\Ex[u+n]{x}\)
&\(\Ax[u|]{\endowxn}@\;2\delta\)&\(\Ax[u|][2]{\endowxn}-(\Ax[u|]{\endowxn})^2\)\\
notation&\(Z\)&\defn{\(\Ax[u|]{\endowxn}\)}&\defn{\(\Ax[u|][2]{\endowxn}\)}&\(\vari{Z}\)\\
\bottomrule
\end{tabular}

\item \(1/m\)thly case:

\label{it:deferred-1m-endowment-fmlas}
\begin{tabular}{ccccc}
\toprule
&p.v.r.v.&APV&2nd moment&variance\\
\midrule
expression&
\makecell{
\(
\begin{aligned}
&v^{K_x^{(m)}+\frac{1}{m}}\indicset{u\le K_x^{(m)}\le u+n-\frac{1}{m}}\\
&\quad+v^{u+n}\indicset{K_x^{(m)}\ge u+n}
\end{aligned}
\)}
&\(\Ax[u|]{}[(m)]{}_{\termxn}+\Ex[u+n]{x}\)
&\(\Ax[u|]{\endowxn}[(m)]@\;2\delta\)&\(\Ax[u|][2]{\endowxn}[(m)]-\qty(\Ax[u|]{\endowxn}[(m)])^2\)\\
notation&\(Z\)&\defn{\(\Ax[u|]{\endowxn}[(m)]\)}&\defn{\(\Ax[u|][2]{\endowxn}[(m)]\)}&\(\vari{Z}\)\\
\bottomrule
\end{tabular}
\end{enumerate}
\subsubsection*{Variable Benefits Insurance}

\item Generally, for variable benefits insurances, we use the general APV
formula in \labelcref{it:gen-apv-fmla}. Here we shall discuss two special cases:
\begin{itemize}
\item arithmetically increasing/decreasing insurances
\item geometrically increasing/decreasing insurances
\end{itemize}
\item For arithmetically increasing/decreasing insurances, there are some actuarial
notations designed for them. To understand them, let us first explore different
\emph{kinds} of arithmetically increasing/decreasing insurances:
\begin{enumerate}
\item increasing annually: death benefits in policy years \(1,2,\dotsc\) are
\(1, 2,\dotsc\) respectively;
\item increasing \(1/m\)thly: death benefits in time intervals \([0,1/m),
[1/m,2/m), \dotsc\) are \(1/m, 2/m,\dotsc\) respectively;
\item increasing continuously: death benefit to be paid at any time \(t\) is \(t\);
\item decreasing annually (for \(n\)-year insurance): death benefits in policy
years \(1,2,\dotsc,n\) are \\\(n, n-1,\dotsc,1\) respectively;
\item decreasing \(1/m\)thly (for \(n\)-year insurance): death benefits in
time intervals \\ \([0,1/m),[1/m,2/m),\dotsc,[(mn-1)/m,m]\) are \(n, n-(1/m),\dotsc,1/m\) respectively;
\item decreasing continuously (for \(n\)-year insurance): death benefit to be paid at time \(t\) is \(n-t\), for any \(t\in[0,n]\).
\end{enumerate}
\begin{rmk}
\item These are understood to be only applicable for the period covered by the
insurance. There is not death benefit for the period outside insurance
coverage.
\item The continuous cases can be seen as limits of the \(1/m\)thly cases as \(m\to\infty\).
\end{rmk}
\item The designed actuarial notations are as follows:

\begin{tabular}{c|ccc}
\toprule
\diagbox{kind}{insurance}&continuous&\(1/m\)thly&annual\\
\midrule
increasing continuously&\defn{\((\bar{I}\Ax*{})\)}&\diagbox[dir=NE]{}{}&\diagbox[dir=NE]{}{}\\
increasing \(1/m\)thly&\defn{\((I^{(m)}\Ax*{})\)}&\defn{\((I^{(m)}\Ax{}[(m)])\)}&\diagbox[dir=NE]{}{}\\
increasing annually&\defn{\((I\Ax*{})\)}&\defn{\((I\Ax{}[(m)])\)}&\defn{\((IA)\)}\\
decreasing continuously&\defn{\((\bar{D}\Ax*{})\)}&\diagbox[dir=NE]{}{}&\diagbox[dir=NE]{}{}\\
decreasing \(1/m\)thly&\defn{\((D^{(m)}\Ax*{})\)}&\defn{\((D^{(m)}\Ax{}[(m)])\)}&\diagbox[dir=NE]{}{}\\
decreasing annually&\defn{\((D\Ax*{})\)}&\defn{\((D\Ax{}[(m)])\)}&\defn{\((DA)\)}\\
\bottomrule
\end{tabular}

\begin{rmk}
\item These are only ``cores'' for the notations, and subscripts/superscripts can be
added on them to represent a variety of insurances (but of course the terms
must be finite for decreasing insurances).
\item Here we do not include the case where the benefit varying frequency is
\emph{higher} than the insurance itself since in such case it is not clear how
changes in death benefits (strictly) \emph{between} the dying time and the
death benefit payment
time (which may exist) should be handled.
\item The kinds of insurance usually encountered here are the ones where the
benefit varying and insurance ``frequencies'' are the same
in the ``main diagonal'' for each of increasing \& decreasing case) --- the APV
formulas for them are ``nicer'' also.
\end{rmk}
\item
\label{it:as-term-life-fmlas}
Example: The following gives the APV formulas for arithmetically
increasing/decreasing \(n\)-year term life insurances, developed by the general
APV formula in \labelcref{it:gen-apv-fmla}:

\begin{tabular}{cc}
\toprule
notation&APV\\
\midrule
\(\IA*_{\termxn}\)&
\(\displaystyle\int_{0}^{1}\px[t]{x}\mu_{x+t}\,dt
+\int_{1}^{2}2\px[t]{x}\mu_{x+t}\,dt
+\dotsb+\int_{n-1}^{n}n\,\px[t]{x}\mu_{x+t}\,dt\)\\
\(\ImA*_{\termxn}\)&
\(\displaystyle\int_{0}^{1/m}\frac{1}{m}\px[t]{x}\mu_{x+t}\,dt
+\int_{1/m}^{2/m}\frac{2}{m}\px[t]{x}\mu_{x+t}\,dt
+\dotsb+\int_{(mn-1)/m}^{n}n\,\px[t]{x}\mu_{x+t}\,dt\)\\
\ystar\(\IbA*_{\termxn}\)&
\(\displaystyle\int_{0}^{n}t\,\px[t]{x}\mu_{x+t}\,dt\)\\
\(\DA*_{\termxn}\)&
\(\displaystyle\int_{0}^{1}n\,\px[t]{x}\mu_{x+t}\,dt
+\int_{1}^{2}(n-1)\px[t]{x}\mu_{x+t}\,dt
+\dotsb+\int_{n-1}^{n}\px[t]{x}\mu_{x+t}\,dt\)\\
\(\DmA*_{\termxn}\)&
\(\displaystyle\int_{0}^{1/m}n\,\px[t]{x}\mu_{x+t}\,dt
+\int_{1/m}^{2/m}\qty(n-\frac{1}{m})\px[t]{x}\mu_{x+t}\,dt
+\dotsb+\int_{(mn-1)/m}^{n}\frac{1}{m}\px[t]{x}\mu_{x+t}\,dt\)\\
\ystar\(\DbA*_{\termxn}\)&
\(\displaystyle\int_{0}^{n}(n-t)\,\px[t]{x}\mu_{x+t}\,dt\)\\
\midrule
\((IA^{(m)})_{\termxn}\)&
\(\displaystyle\sum_{k=0}^{m-1}v^{\frac{k+1}{m}}\px[\frac{k}{m}]{x}\qx[\frac{1}{m}]{x+\frac{k}{m}}
+\dotsb+\sum_{k=m(n-1)}^{mn-1}nv^{\frac{k+1}{m}}\px[\frac{k}{m}]{x}\qx[\frac{1}{m}]{x+\frac{k}{m}}\)\\
\ystar\((I^{(m)}A^{(m)})_{\termxn}\)&
\(\displaystyle
\sum_{k=0}^{mn-1}\frac{k+1}{m}v^{\frac{k+1}{m}}\px[\frac{k}{m}]{x}\qx[\frac{1}{m}]{x+\frac{k}{m}}\)\\
\((DA^{(m)})_{\termxn}\)&
\(\displaystyle\sum_{k=0}^{m-1}nv^{\frac{k+1}{m}}\px[\frac{k}{m}]{x}\qx[\frac{1}{m}]{x+\frac{k}{m}}
+\dotsb+\sum_{k=m(n-1)}^{mn-1}v^{\frac{k+1}{m}}\px[\frac{k}{m}]{x}\qx[\frac{1}{m}]{x+\frac{k}{m}}\)\\
\ystar\((D^{(m)}A^{(m)})_{\termxn}\)&
\(\displaystyle
\sum_{k=0}^{mn-1}\qty(n-\frac{\color{brown}{k}}{m})v^{\frac{k+1}{m}}\px[\frac{k}{m}]{x}\qx[\frac{1}{m}]{x+\frac{k}{m}}\)\\
\midrule
\ystar\((IA)_{\termxn}\)&
\(\displaystyle\sum_{k=0}^{n-1}(k+1)v^{k+1}\px[k]{x}\qx{x+k}\)\\
\ystar\((DA)_{\termxn}\)&
\(\displaystyle\sum_{k=0}^{n-1}(n-{\color{brown}k})v^{k+1}\px[k]{x}\qx{x+k}\)\\
\bottomrule
\end{tabular}
\begin{note}
\ystar{} = benefit varying and insurance ``frequencies'' are the same.
\end{note}

\item For arithmetically \emph{decreasing} insurances in particular, we have an
``interesting'' result:
\begin{proposition}
\label{prp:arith-decreasing-fmla}
We have
\begin{enumerate}
\item \(\displaystyle\DbA*_{\termxn}=\int_{0}^{n}\Ax*{\itop{x}:\angl{n-t}}\,dt\);\\
\item \(\displaystyle(D^{(m)}A^{(m)})_{\termxn}=\sum_{k=0}^{mn-1}\Ax{\itop{x}:\angl{n-\frac{k}{m}}}[(m)]\);\\
\item \(\displaystyle(DA)_{\termxn}=\sum_{k=0}^{n-1}\Ax{\itop{x}:\angl{n-k}}\).
\end{enumerate}
\end{proposition}
\begin{intuition}

\begin{tikzpicture}
\node[] () at (5.5, 4) {LHS: ``slicing vertically''};
\node[] () at (1.5,3) {\(\DbA*_{\termxn}\)};
\draw[pattern={Lines[angle=90, distance=1pt]}] (0,0)  -- (3,0)  -- (0,3) -- cycle;
\node[] () at (5.5,3) {\((D^{(m)}A^{(m)})_{\termxn}\)};
\draw[pattern=vertical lines] (4,0)  -- (7,0)  -- (4,3) -- cycle;
\node[] () at (9.5,3) {\((DA)_{\termxn}\)};
\draw[pattern={Lines[angle=90, distance=7.5pt]}] (8,0)  -- (11,0)  -- (8,3) -- cycle;
\end{tikzpicture}

\begin{tikzpicture}
\node[] () at (5.5, 4) {RHS: ``slicing horizontally''};
\node[] () at (1.5,3) {\(\displaystyle\int_{0}^{n}\Ax*{\itop{x}:\angl{n-t}}\,dt\)};
\draw[pattern={Lines[distance=1pt]}] (0,0)  -- (3,0)  -- (0,3) -- cycle;
\node[] () at (5.5,3) {\(\displaystyle\sum_{k=0}^{mn-1}\Ax{\itop{x}:\angl{n-\frac{k}{m}}}\)};
\draw[pattern=horizontal lines] (4,0)  -- (7,0)  -- (4,3) -- cycle;
\node[] () at (9.5,3) {\(\displaystyle\sum_{k=0}^{n-1}\Ax{\itop{x}:\angl{n-k}}\)};
\draw[pattern={Lines[distance=7.5pt]}] (8,0)  -- (11,0)  -- (8,3) -- cycle;
\end{tikzpicture}
\end{intuition}

\begin{pf}
For the continuous case we have
\[
\DbA*_{\termxn}=\int_{0}^{n}(n-s)\,\px[s]{x}\mu_{x+s}\,ds
=\int_{0}^{n}\qty(\int_{0}^{n-s}1\,dt)\,\px[s]{x}\mu_{x+s}\,ds
=\int_{0}^{n}\underbrace{\qty(\int_{0}^{n-t}\,\px[s]{x}\mu_{x+s}\,ds)}_{\Ax*{\itop{x}:\angl{n-t}}}\,dt.
\]
Next, for the \(1/m\)thly case we have
\begin{align*}
(D^{(m)}A^{(m)})_{\termxn}
&=\sum_{j=0}^{mn-1}\qty(n-\frac{j}{m})
v^{\frac{j+1}{m}}\px[\frac{j}{m}]{x}\qx[\frac{1}{m}]{x+\frac{j}{m}}\\
&=\sum_{j=0}^{mn-1}\qty(\sum_{k=0}^{n-\frac{j}{m}}1)v^{\frac{j+1}{m}}\px[\frac{j}{m}]{x}\qx[\frac{1}{m}]{x+\frac{j}{m}}\\
&=\sum_{k=0}^{mn-1}\underbrace{\sum_{j=0}^{n-\frac{k}{m}}v^{\frac{j+1}{m}}\px[\frac{j}{m}]{x}\qx[\frac{1}{m}]{x+\frac{j}{m}}}_{\Ax{\itop{x}:\angl{n-\frac{k}{m}}}}.
\end{align*}
The proof for the annual case is similar.
\end{pf}
\item For geometrically increasing/decreasing insurances, we focus on the
annual case (where the benefit varying and insurance ``frequencies'' are the
same) here: death benefits in policy years \(1,2,3,\dotsc\) are
\(1,(1+j),(1+j)^2,\dotsc\) respectively, where \(-1<j<1\).

\begin{note}
For the \(1/m\)thly case where the benefit varying and insurance
``frequencies'' are still the same, the results here can be easily translated.
\end{note}
\item Generally, the APV can be computed by considering the general APV formula
in \labelcref{it:gen-apv-fmla} and the formula for geometric series. But the
following gives a shortcut formula:
\begin{proposition}
\label{prp:gs-term-life-fmla}
The APV of an \(n\)-year term life insurance with such geometric
sequence in the death benefit amounts is given by:
\[
\frac{1}{1+j}\Ax{\itop{x}:\angl{n}i^*}.
\]
where \(i^*=(i-j)/(1+j)\).
\end{proposition}
\begin{pf}
Note that the APV is
\[
\sum_{k=0}^{n-1}(1+j)^kv^{k+1}\px[k]{x}\qx{x+k}
=\frac{1}{1+j}\sum_{k=0}^{n-1}\qty(\frac{1+i}{1+j})^{-(k+1)}\px[k]{x}\qx{x+k},
\]
and that
\[
1+i^*=\frac{1+j+i-j}{1+j}=\frac{1+i}{1+j}.
\]
\end{pf}

\begin{note}
For the special case where \(i=j\), we have \(i^*=0\) and hence the APV can be
simplified to \(\qx[n]{x}/(1+j)\) as the p.v.r.v.\ for an \(n\)-year term life
insurance issued to \((x)\) with zero interest rate is simply
\(\indicset{T_x\le n}\).
\end{note}
\end{enumerate}

\subsection{Recursions for APVs}
\begin{enumerate}
\item There are two main reasons for considering recursions of APVs here:
\begin{enumerate}
\item It provides insight on the relationships of APVs at different ages.
\item It allows quick computations based on limited amount of information.
\end{enumerate}
\item Most recursion formulas can be intuitively understood through ``actuarial
discounting''.
\item Recursions for whole life insurance:
\begin{proposition}
\label{prp:wl-recursions}
For any age \(x\) and any \(n\in\N\),
\begin{enumerate}
\item \(\Ax*{x}=\Ax*{\termxn}+\Ex[n]{x}\Ax*{x+n}\);
\item \(\Ax{x}=\Ax{\termxn}+\Ex[n]{x}\Ax{x+n}\);
\item \(\Ax{x}[(m)]=\Ax{}[(m)]{}_{\itop{x}:\angl{n/m}}+\Ex[n]{x}\Ax{x+n/m}[(m)]\).
\end{enumerate}
\end{proposition}
\begin{intuition}

\begin{tikzpicture}
\draw[-Latex] (0,0) -- (10,0) node[right]{Time};
\fill[] (0,0) circle [radius=0.05]
node[below] {0}
node[above] {\((x)\)};
\fill[] (3,0) circle [radius=0.05]
node[below] {\(n\)}; \draw[pen colour=brown, very thick, decorate,decoration={calligraphic brace, amplitude=5pt, raise=15pt}] (0,0) -- (3,0) node[brown, midway, above=0.7cm]{\(\Ax*{\termxn}\)};
\draw[pen colour=brown, very thick, decorate,decoration={calligraphic brace, amplitude=5pt, raise=25pt}] (3,0) -- (10,0)
node[brown, midway, above=1cm]{\(\Ax*{x+n}\)};
\draw[pen colour=violet, very thick, decorate,decoration={mirror, calligraphic brace, amplitude=5pt, raise=15pt}] (0,0) -- (10,0)
node[violet, midway, below=0.7cm]{\(\Ax*{x}\)};
\draw[-Latex, color=ForestGreen] (3,1) to[bend right] (0,1.6);
\draw[-Latex, color=ForestGreen] (10,1) to[bend right] (0,1.9);
\node[] () at (2.5,2) {\(\times\Ex[n]{x}\)};
\end{tikzpicture}
\end{intuition}

\begin{pf}
The result follows from the proofs in
\labelcref{it:cts-term-life-fmlas,it:annual-term-life-fmlas,it:1m-term-life-fmlas}.
\end{pf}

\begin{note}
As a special case, when \(n=1\), we have \[\Ax{x}=v\qx{x}+v\px{x}\Ax{x+1}\]
and
\[
\Ax{x}=v^{\frac{1}{m}}\qx[\frac{1}{m}]{x}+v^{\frac{1}{m}}\px[\frac{1}{m}]{x}\Ax{x+\frac{1}{m}}.
\]
\end{note}
\item Recursions for term life insurance:
\begin{proposition}
\label{prp:tl-recursions}
For any age \(x\) and any \(n\in\N\),
\begin{enumerate}
\item \(\Ax*{\termxn}=\Ax*{\itop{x}:\angl{u}}+\Ex[u]{x}\Ax*{\itop{x+u}:\angl{n-u}}\) (for any \(u\in\N\) with \(u\le n\));
\item \(\Ax{\termxn}=\Ax{\itop{x}:\angl{u}}+\Ex[u]{x}\Ax{\itop{x+u}:\angl{n-u}}\) (for any \(u\in\N\) with \(u\le n\));
\item \(\Ax{}[(m)]{}_{\termxn}=\Ax{}[(m)]{}_{\itop{x}:\angl{u/m}}+\Ex[u/m]{x}\Ax{}[(m)]{}_{\itop{(x+u/m)}:\angl{n-u/m}}\) (for any \(u\in\N\) with \(u\le mn\)).

\end{enumerate}
\end{proposition}

\begin{intuition}

\begin{tikzpicture}
\draw[-Latex] (0,0) -- (10,0) node[right]{Time};
\fill[] (0,0) circle [radius=0.05]
node[below] {0}
node[above] {\((x)\)};
\fill[] (3,0) circle [radius=0.05]
node[below] {\(u\)};
\fill[] (8,0) circle [radius=0.05]
node[below] {\(n\)};
\draw[pen colour=brown, very thick, decorate,decoration={calligraphic brace, amplitude=5pt, raise=15pt}] (0,0) -- (3,0)
node[brown, midway, above=0.7cm]{\(\Ax*{\itop{x}:\angl{u}}\)};
\draw[pen colour=brown, very thick, decorate,decoration={calligraphic brace, amplitude=5pt, raise=25pt}] (3,0) -- (8,0)
node[brown, midway, above=1cm]{\(\Ax*{\itop{x+u}:\angl{n-u}}\)};
\draw[pen colour=violet, very thick, decorate,decoration={mirror, calligraphic brace, amplitude=5pt, raise=15pt}] (0,0) -- (8,0)
node[violet, midway, below=0.7cm]{\(\Ax*{\termxn}\)};
\draw[-Latex, color=ForestGreen] (3,1) to[bend right] (0,1.6);
\draw[-Latex, color=ForestGreen] (8,1) to[bend right] (0,1.9);
\node[] () at (2.5,2) {\(\times\Ex[u]{x}\)};
\end{tikzpicture}
\end{intuition}

\begin{pf}
Similar to the proof for \cref{prp:wl-recursions}.
\end{pf}

\item Recursions for endowment insurance:
\begin{proposition}
\label{prp:endow-recursions}
For any age \(x\) and any \(n\in\N\),
\begin{enumerate}
\item \(\Ax*{\endowxn}=\Ax*{\itop{x}:\angl{u}}+\Ex[u]{x}\Ax*{x+u:\angl{n-u}}\) (for any \(u\in\N\) such that \(u\le n\));
\item \(\Ax{\endowxn}=\Ax{\itop{x}:\angl{u}}+\Ex[u]{x}\Ax{x+u:\angl{n-u}}\) (for any \(u\in\N\) such that \(u\le n\));
\item \(\Ax{}[(m)]{}_{\endowxn}=\Ax{}[(m)]{}_{\itop{x}:\angl{u/m}}+\Ex[u/m]{x}\Ax{(x+u/m):\angl{n-u/m}}[(m)]\) (for any \(u\in\N\) such that \(u\le mn\)).
\end{enumerate}
\end{proposition}

\begin{intuition}

\begin{tikzpicture}
\draw[-Latex] (0,0) -- (10,0) node[right]{Time};
\fill[] (0,0) circle [radius=0.05]
node[below] {0}
node[above] {\((x)\)};
\fill[] (3,0) circle [radius=0.05]
node[below] {\(u\)};
\fill[] (8,0) circle [radius=0.05]
node[below] {\(n\)};
\draw[pen colour=brown, very thick, decorate,decoration={calligraphic brace, amplitude=5pt, raise=15pt}] (0,0) -- (3,0)
node[brown, midway, above=0.7cm]{\(\Ax*{\itop{x}:\angl{u}}\)};
\draw[pen colour=brown, very thick, decorate,decoration={calligraphic brace, amplitude=5pt, raise=25pt}] (3,0) -- (8,0)
node[brown, midway, above=1cm]{\(\Ax*{x+u:\angl{n-u}}\)};
\draw[pen colour=violet, very thick, decorate,decoration={mirror, calligraphic brace, amplitude=5pt, raise=15pt}] (0,0) -- (8,0)
node[violet, midway, below=0.7cm]{\(\Ax*{\endowxn}\)};
\draw[-Latex, color=ForestGreen] (3,1) to[bend right] (0,1.6);
\draw[-Latex, color=ForestGreen] (8,1) to[bend right] (0,1.9);
\node[] () at (2.5,2) {\(\times\Ex[u]{x}\)};
\end{tikzpicture}
\end{intuition}

\begin{pf}
Similar to the proof for \cref{prp:wl-recursions}.
\end{pf}
\item Recursions for arithmetically increasing/decreasing insurances \(\IA\)/\(\DA\):
\begin{proposition}
\label{prp:annual-arith-recursions}
We have
\begin{enumerate}
\item\label{it:incr-split-first} \(\IA_{\termxn}=v\qx{x}+v\px{x}\qty[\IA_{\itop{x+1}:\angl{n-1}}+\Ax{\itop{x+1}:\angl{n-1}}]\) and
\(\IA_{x}=v\qx{x}+v\px{x}\qty[\IA_{x+1}+\Ax{x+1}]\);
\item\label{it:decr-split-first} \(\DA_{\termxn}=nv\qx{x}+v\px{x}\DA_{\itop{x+1}:\angl{n-1}}\);
\item\label{it:incr-split-horizon} \(\IA_{\termxn}=\Ax{\termxn}+v\px{x}\IA_{\itop{x+1}:\angl{n-1}}\) and
\(\IA_{x}=\Ax{x}+v\px{x}\IA_{x+1}\).
\end{enumerate}
\end{proposition}

\begin{intuition}

\labelcref{it:incr-split-first}:

\begin{tikzpicture}
\draw[-Latex] (0,0) -- (10,0) node[right]{Time};
\fill[] (0,0) circle [radius=0.05]
node[below] {0};
\fill[] (2,0) circle [radius=0.05]
node[below] {1}
node[above, violet] {\faIcon{money-bill-wave}};
\fill[] (4,0) circle [radius=0.05]
node[below] {2}
node[above, magenta] {\faIcon{money-bill-wave}}
node[above=0.3cm, brown] {\faIcon{money-bill-wave}};
\fill[] (6,0) circle [radius=0.05]
node[below] {3}
node[above, magenta] {\faIcon{money-bill-wave}}
node[above=0.3cm, brown] {\faIcon{money-bill-wave}}
node[above=0.6cm, brown] {\faIcon{money-bill-wave}};
\fill[] (8,0) circle [radius=0.05]
node[below] {4}
node[above, magenta] {\faIcon{money-bill-wave}}
node[above=0.3cm, brown] {\faIcon{money-bill-wave}}
node[above=0.6cm, brown] {\faIcon{money-bill-wave}}
node[above=0.9cm, brown] {\faIcon{money-bill-wave}};
\node[thick, brown] () at (8.7,1) {\(\cdots\)};
\node[thick, magenta] () at (8.7,0.3) {\(\cdots\)};
\draw[-Latex, violet] (3,-0.7) -- (2.3,0)
node[pos=-0.3] {\(v\qx{x}\)};
\draw[-Latex, magenta] (11,0.3) -- (9,0.3)
node[pos=-0.45] {\(\Ax{\itop{x+1}:\angl{n-1}}\)};
\draw[brown, thick, decorate,decoration={mirror, brace, amplitude=5pt, raise=15pt}] (8.5,0.5) -- (8.5,1.5)
node[midway, right=1cm] {\(\IA_{\itop{x+1}:\angl{n-1}}\)};
\draw[-Latex, ForestGreen] (3.7,0.2) to[bend right] (0,0.1);
\draw[-Latex, ForestGreen] (8.2,1.4) to[bend right] (0,0.3);
\node[] () at (3.5,1.4) {\(\times v\px{x}\)};
\end{tikzpicture}



\labelcref{it:decr-split-first}:

\begin{tikzpicture}
\draw[-Latex] (0,0) -- (10,0) node[right]{Time};
\fill[] (0,0) circle [radius=0.05]
node[below] {0};
\fill[] (2,0) circle [radius=0.05]
node[below] {1}
node[above, violet] {\faIcon{money-bill-wave}}
node[above=0.4cm, violet] {\(\vdots\)}
node[above=0.8cm, violet] {\faIcon{money-bill-wave}}
node[above=1.2cm, violet] {\faIcon{money-bill-wave}}
node[above=1.6cm, violet] {\faIcon{money-bill-wave}}
node[above=2cm, violet] {\faIcon{money-bill-wave}}
;
\fill[] (4,0) circle [radius=0.05]
node[below] {2}
node[above, brown] {\faIcon{money-bill-wave}}
node[above=0.4cm, brown] {\(\vdots\)}
node[above=0.8cm, brown] {\faIcon{money-bill-wave}}
node[above=1.2cm, brown] {\faIcon{money-bill-wave}}
node[above=1.6cm, brown] {\faIcon{money-bill-wave}}
;
\fill[] (6,0) circle [radius=0.05]
node[below] {3}
node[above, brown] {\faIcon{money-bill-wave}}
node[above=0.4cm, brown] {\(\vdots\)}
node[above=0.8cm, brown] {\faIcon{money-bill-wave}}
node[above=1.2cm, brown] {\faIcon{money-bill-wave}};

\fill[] (8,0) circle [radius=0.05]
node[below] {4}
node[above, brown] {\faIcon{money-bill-wave}}
node[above=0.4cm, brown] {\(\vdots\)}
node[above=0.8cm, brown] {\faIcon{money-bill-wave}};
\node[thick, brown] () at (8.7,1) {\(\cdots\)};
\draw[-Latex, violet] (3,-0.7) -- (2.3,0)
node[pos=-0.3] {\(nv\qx{x}\)};
\draw[brown, thick, decorate,decoration={mirror, brace, amplitude=5pt, raise=15pt}] (8.5,0.2) -- (8.5,2)
node[midway, right=1cm] {\(\DA_{\itop{x+1}:\angl{n-1}}\)};
\draw[-Latex, ForestGreen] (3.7,0.2) to[bend right] (0,0.1);
\draw[-Latex, ForestGreen] (8.2,1.4) to[bend right] (0,0.3);
\node[brown] () at (3,1.4) {\(\times v\px{x}\)};
\end{tikzpicture}

\labelcref{it:incr-split-horizon}:

\begin{tikzpicture}
\draw[-Latex] (0,0) -- (10,0) node[right]{Time};
\fill[] (0,0) circle [radius=0.05]
node[below] {0};
\fill[] (2,0) circle [radius=0.05]
node[below] {1}
node[above, violet] {\faIcon{money-bill-wave}};
\fill[] (4,0) circle [radius=0.05]
node[below] {2}
node[above, violet] {\faIcon{money-bill-wave}}
node[above=0.3cm, brown] {\faIcon{money-bill-wave}};
\fill[] (6,0) circle [radius=0.05]
node[below] {3}
node[above, violet] {\faIcon{money-bill-wave}}
node[above=0.3cm, brown] {\faIcon{money-bill-wave}}
node[above=0.6cm, brown] {\faIcon{money-bill-wave}};
\fill[] (8,0) circle [radius=0.05]
node[below] {4}
node[above, violet] {\faIcon{money-bill-wave}}
node[above=0.3cm, brown] {\faIcon{money-bill-wave}}
node[above=0.6cm, brown] {\faIcon{money-bill-wave}}
node[above=0.9cm, brown] {\faIcon{money-bill-wave}};
\node[thick, brown] () at (8.7,1) {\(\cdots\)};
\node[thick, violet] () at (8.7,0.3) {\(\cdots\)};
\draw[-Latex, violet] (11,0.3) -- (9,0.3)
node[pos=-0.3] {\(\Ax{\termxn}\)};
\draw[brown, thick, decorate,decoration={mirror, brace, amplitude=5pt, raise=15pt}] (8.5,0.5) -- (8.5,1.5)
node[midway, right=1cm] {\(\IA_{\itop{x+1}:\angl{n-1}}\)};
\draw[-Latex, ForestGreen] (3.7,0.5) to[bend right] (0,0.1);
\draw[-Latex, ForestGreen] (8.2,1.4) to[bend right] (0,0.3);
\node[brown] () at (3.5,1.4) {\(\times v\px{x}\)};
\end{tikzpicture}
\end{intuition}

\begin{pf}
Exercise. (The intuition already illustrates the key idea in the proof ---
one just needs to ``split'' the terms appropriately.)
\end{pf}
\end{enumerate}
\subsection{Relating \(\Ax*{}\), \(\Ax{}\) and \(\Ax{}[(m)]\)}
\begin{enumerate}
\item In a life table, we often only have the values for ``\(\Ax{}\)'' but not for
``\(\Ax*{}\)'' and ``\(\Ax{}[(m)]\)''. So, we are interested in the
relationship between them to see how we can get ``\(\Ax*{}\)'' and ``\(\Ax{}[(m)]\)''
from ``\(\Ax{}\)''.
\item It turns out that in order to obtain some ``nice'' relationships between them, we
need to impose the UDD assumption:
\begin{proposition}
\label{prp:udd-relate-axs}
Under UDD assumption,
\begin{enumerate}
\item \(\displaystyle\Ax*{*}=\frac{i}{\delta}\Ax{*}\)
\item \(\displaystyle\Ax{*}[(m)]=\frac{i}{i^{(m)}}\Ax{*}\)
\end{enumerate}
where \(\Ax{*}\) denotes any of \(\Ax{x},\Ax{\termxn}, \Ax[u|]{x}, \Ax[u|]{\termxn}, \IA_x,
\IA_{\termxn}, \DA_{\termxn}\) (i.e., any whole life/term life with possibly
deferral or ``arithmetically increasing/decreasing (annual)'' variant).
\end{proposition}
\begin{warning}
\(\Ax{*}\) does \underline{not} include \(\Ax{\endowxn}\). Indeed, under UDD
assumption we have
\[
\Ax*{\endowxn}=\frac{i}{\delta}\Ax{\termxn}+\Ex[n]{x}
\]
which can be seen by writing \(\Ax*{\endowxn}=\Ax*{\termxn}+\Ex[n]{x}\)
(similar for \(\Ax{\endowxn}[(m)]\)).
\end{warning}

\begin{pf}
We shall only give some key idea. See \textcite[Section~4.4]{bowers1997actuarial} for more details.

For the continuous case, consider \cref{lma:udd-equiv-def} --- and note that
\(v^{T_x}=v^{K_x+1}(1+i)^{1-U_x}\) where \[\expv{(1+i)^{1-U_x}}=i/\delta.\]

For the \(1/m\)thly case, consider e.g.,
\[
\sum_{j=0}^{\infty}v^{\frac{j+1}{m}}\px[\frac{j}{m}]{x}\qx[\frac{1}{m}]{x+\frac{j}{m}}
=\sum_{k=0}^{\infty}\sum_{u=0}^{m-1}v^{k+\frac{u+1}{m}}\px[k+\frac{u}{m}]{x}\qty(\frac{1}{m}\qx{x+k+\frac{u}{m}})
\]
and note that
\[
\sum_{u=0}^{m-1}\frac{1}{m}v^{(u+1)/m}=\frac{i}{i^{(m)}}.
\]
\end{pf}
\end{enumerate}
\subsection{Incorporating Selection}
\begin{enumerate}
\item All previous developments also apply to ``select'' lives (just change
``\(x\)'' to ``\([x]\)'' in the notations).
\end{enumerate}
