\section{Life Insurance}
\label{sect:life-insurance}
\begin{enumerate}
\item The decision of whether a policy \faIcon{file-alt} should be sold the
policyholder \faIcon{user} or not depends critically on the \emph{present
value} for the cash flows involved in \faIcon{file-alt} (including both
benefits and premiums).
\item To improve the tractability, we shall assume that the annual interest
rate \(i\) is a fixed constant, so that the ``randomness'' of the present value
(random variable) would only source from the modelled survival distribution for
\faIcon{user} --- the present value is \emph{life contingent}.
\item Often the insurance business \faIcon{building} is of very
large scale and many identical policies may be sold to lives modelled by the
same survival distribution (the ``ordinary'' one, say).
\item In such case, the \emph{average} present value of those policies is a
key metric for the profitability per policy sold. By virtue of \emph{weak
law of large number} (assuming it is applicable), the average would converge to
the \emph{expected present value} as the number of such policies sold
goes to infinity. Hence, the study of life contingencies focuses a lot on the
expected present value (also known as \defn{actuarial present value} (APV)).
\item \Cref{sect:life-insurance,sect:life-annuity} mainly focus on determining
the APV of the benefit payments of the contracts mentioned in
\cref{subsect:traditional-contracts}.
\item The contracts can be classified into two big categories, by the payment
``timing'':
\begin{enumerate}
\item Discrete time: insurance \(\rightarrow\) payable at the end of period;
life annuity \(\rightarrow\) payable at the beginning of period. (\(K_x\) is
involved mainly.)
\item Continuous time: insurance \(\rightarrow\) payable ``immediately''; life
annuity \(\rightarrow\) payable ``continuously'' (\emph{continuous annuity} in
interest theory). (\(T_x\) is involved mainly.)
\end{enumerate}
It turns out that working in continuous time (working with \(T_x\)) is
mathematically more ``convenient''. However, practical contracts are always
discrete in nature. So, sometimes quantities for ``discrete'' contracts in
practice are approximated through working in continuous time instead, for
mathematical convenience. (The approximation works better if the measurement
period for the contract is shorter.)

\item
\label{it:gen-APV-fmla}
\ystar An ``intuition-based'' general APV calculation formula for
insurance/life annuity:
\[
\underset{\text{all possible payment times}}{\sum\text{ or }\int}\text{benefit amount}\times\text{discount
factor}\times\text{prob.\ of triggering payment}.
\]
\begin{note}
For continuous case with ``\(\int\)'', the terms above can be in
``infinitesimal'' sense (e.g., involving ``\(dt\)''), loosely.
\end{note}


\begin{intuition}
It ``sums up'' the ``expected present value of benefit payment'':
\[
\underbrace{\text{benefit amount}\times\text{discount factor}}_{\text{PV of payment}}
\times\text{prob.\ of triggering payment}
\]
in all time intervals.
\end{intuition}
\end{enumerate}

\subsection{Valuating Whole Life Insurance}
\begin{enumerate}
\subsubsection*{Continuous Case}
\item Consider a \emph{continuous} whole life insurance with sum insured \$1
and life insured (\(x\)). The sum insured \$1 is paid \emph{exactly at} time of
death of (\(x\)), i.e., time \(T_x\) (when the life is aged \(x\), it is time
0).  \item The present value random variable (p.v.r.v.) (present value of the
benefit payment) is
\[
Z=v^{T_x}=e^{-\delta T_x}.
\]
where \(\delta\) is the annual force of interest equivalent to the annual rate
\(i\). The APV is:
\[
\expv{Z}=\expv{e^{-\delta T_x}}=\int_{0}^{\infty}e^{-\delta t}\underbrace{\px[t]{x}\mu_{x+t}}_{f_x(t)}\,dt
\]
\begin{intuition}
RHS ``sums up'' the ``expected present value of death benefit'':
\[\underbrace{e^{-\delta t}}_{\text{PV of payment}}\times \underbrace{\px[t]{x}\mu_{x+t}\,dt}_{\text{death prob. in }[t,t+dt]}\]
(\(e^{-\delta(t+dt)}\) and \(e^{-\delta t}\) are ``the same'' in
``infinitesimal'' sense) in all ``infinitesimal'' time intervals.
\end{intuition}
\item Actuarial notation for the APV:

\begin{tikzpicture}
\node[font=\huge] (Ax) at (0,0) {\defn{\(\Ax*{x}\)}};
\draw[-Latex] (-1,1) -- (0,0.5);
\node[] () at (-1.5,1.3)  {``bar'' indicates ``continuous''};
\draw[-Latex] (1.5,1) -- (0.25,-0.1);
\node[] () at (1.8,1.3) {for (\(x\))};
\draw[-Latex] (-1.5,-1) -- (-0.35,-0.35);
\node[text width=3.5cm] () at (-1.8,-2.2) {\underline{A}ssurance (another term for
``insurance'', which is more commonly used in the UK)};
\end{tikzpicture}

\item Sometimes we are also interested in the \emph{variability} of the
p.v.r.v.\ \(Z\). To measure this, we calculate the variance \(\vari{Z}\).
\item Consider first the second moment of \(Z\), which is given by
\[
\expv{Z^2}=\expv{e^{-{\color{purple} 2\delta} T_x}}=\Ax*{x}@\;{\color{purple} 2\delta},
\]
i.e., the APV evaulated at force of interest \(2\delta\) (Actuarial notation:
\defn{\(\Ax*[][2]{x}\)}). Then the variance is
\[
\vari{Z}=\Ax*[][2]{x}-(\Ax*{x})^2.
\]
\item When the sum insured for the insurance becomes \(\$ S\), the p.v.r.v.\
becomes \(S\cdot Z\). Then, the APV becomes \(\expv{SZ}=S\Ax*{x}\), the second
moment becomes \(\expv{(SZ)^2}=S^{\color{purple}2}\cdot\Ax*[][2]{x}\), and the
variance is \(\vari{SZ}=S^{\color{purple}2}\vari{Z}\).
\begin{warning}
Do not miss the square for the second moment and variance!
\end{warning}
\item To compute \emph{probability} related to the p.v.r.v. in the form
of \(\prob{Z\le \text{\faIcon{question-circle}}}\), we try to find an
expression of the form
``\(T_x\ge\text{\faIcon[regular]{question-circle}}\)''\footnote{The direction
of inequality is given by ``\(\ge\)'' since, loosely, \(T_x\uparrow\iff
Z\downarrow\), for a fixed insurance contract.}  that is \emph{equivalent} to
``\(Z\le \text{\faIcon{question-circle}}\)''. Then, we instead compute the
probability \(\prob{T_x\ge\text{\faIcon[regular]{question-circle}}}\) (which
equals \(\prob{Z\le \text{\faIcon{question-circle}}}\)). We have similar
approaches for other ``forms'' of the probability.
\begin{note}
This works generally for all kinds of insurance introduced in
\cref{sect:life-insurance}.
\end{note}
\item
\label{it:cts-whole-life-fmlas}
 Summary:

\begin{tabular}{ccccc}
\toprule
&p.v.r.v.&APV&2nd moment&variance\\
\midrule
expression&\(e^{-\delta T_x}\)&\(\displaystyle\int_{0}^{\infty}e^{-\delta t}\px[t]{x}\mu_{x+t}\,dt\)
&\(\Ax*{x}@\;2\delta\)&\(\Ax*[][2]{x}-(\Ax*{x})^2\)\\
notation&\(Z\)&\(\Ax*{x}\)&\(\Ax*[][2]{x}\)&\(\vari{Z}\)\\
\bottomrule
\end{tabular}
\subsubsection*{Annual Case}
\item Consider a \emph{discrete} whole life insurance with sum insured
\faIcon{money-bill-wave} \$1 and life insured (\(x\)). The sum insured \$1 is
paid \emph{at the end of (policy) year} of death of (\(x\)), i.e., time
\(K_x+1\).

\begin{tikzpicture}
\draw[-Latex] (0,0) -- (10,0) node[right]{Time};
\fill[] (0,0) circle [radius=0.05]
node[below] {0}
node[above] {\((x)\)};
\fill[] (2,0) circle [radius=0.05]
node[below] {1};
\fill[] (4,0) circle [radius=0.05]
node[below] {\(K_x\)};
\fill[] (4.8,0) circle [radius=0.05]
node[red, above] {\faIcon{skull}}
node[below] {\(T_x\)};
\fill[] (6,0) circle [radius=0.05]
node[ForestGreen, above] {\faIcon{money-bill-wave}}
node[ForestGreen, above=0.4cm] {\$1}
node[below] {\(K_x+1\)};
\node[] () at (11,-1) {Policy year};
\draw[very thick, decorate,decoration={calligraphic brace, amplitude=5pt, raise=15pt, mirror}] (0,0) -- (2,0)
node[midway, below=0.7cm]{1};
\draw[very thick, decorate,decoration={calligraphic brace, amplitude=5pt, raise=15pt, mirror}] (4,0) -- (6,0)
node[midway, below=0.7cm]{\(K_x\)};
\end{tikzpicture}
\item 
\label{it:annual-whole-life-fmlas}
\begin{tabular}{ccccc}
\toprule
&p.v.r.v.&APV&2nd moment&variance\\
\midrule
expression&\(v^{K_x+1}\)&\(\displaystyle\sum_{k=0}^{\infty}v^{k+1}\px[k]{x}\qx{x+k}\)
&\(\Ax{x}@\;2\delta\)&\(\Ax[][2]{x}-(\Ax{x})^2\)\\
notation&\(Z\)&\defn{\(\Ax{x}\)}&\defn{\(\Ax[][2]{x}\)}&\(\vari{Z}\)\\
\bottomrule
\end{tabular}

\begin{rmk}
\item The actuarial notation \(\Ax{x}\) has no ``bar'' as the insurance is no longer
continuous.
\item We can change ``\(v\)'' to ``\(e^{-\delta}\)'' above, where \(\delta\) is the
equivalent annual force of interest.
\end{rmk}

\begin{intuition}
For the APV formula, it sums up the expected present value of death benefit \faIcon{money-bill-wave}:
\[\underbrace{v^{k+1}}_{\text{PV of payment}}\times \underbrace{\px[k]{x}\qx{x+k}}_{\text{death prob. in }[k,k+1)}\]
in all ``unit'' time intervals.
\end{intuition}
\subsubsection*{\(1/m\)thly Case}
\item Consider a \emph{discrete} whole life insurance with sum insured
\faIcon{money-bill-wave} \$1 and life insured (\(x\)). The sum insured \$1 is
paid \emph{at the end of \(1/m\)th of a (policy) year} of death of (\(x\)).
\item Since ``\(1/m\)th of a (policy) year'' is involved, it would be more
convenient to develop a similar concept as the curtate future lifetime random
variable \(K_x\) but with time unit being \(1/m\)th of a year.
\item The \defn{\(1/m\)thly curtate future lifetime random variable} (denoted
by \defn{\(K_x^{(m)}\)}) is \(T_x\) rounded down to 1/\(m\)th of a year, i.e.,
\[
K_x^{(m)}=\frac{1}{m}\lfloor mT_x\rfloor.
\]
\begin{tikzpicture}
\draw[-Latex] (0,0) -- (10,0) node[right]{Time};
\fill[] (0,0) circle [radius=0.05]
node[above] {(\(x\))}
node[below] {0};
\fill[] (1,0) circle [radius=0.05];
\fill[] (2,0) circle [radius=0.05]
node[below] {1};
\fill[] (3,0) circle [radius=0.05];
\fill[] (4,0) circle [radius=0.05]
node[below] {2};
\fill[] (5,0) circle [radius=0.05];
\fill[] (6,0) circle [radius=0.05]
node[below] {3};
\fill[] (7,0) circle [radius=0.05];
\fill[] (8,0) circle [radius=0.05]
node[below] {4};
\draw[thick, decorate,decoration={calligraphic brace, amplitude=5pt, raise=15pt, mirror}] (1,0) -- (1.99,0)
node[midway, below=0.7cm, font=\small] {\(K_x^{(1/2)}=0.5\)};
\draw[thick, decorate,decoration={calligraphic brace, amplitude=5pt, raise=15pt, mirror}] (4,0) -- (4.99,0)
node[midway, below=0.7cm, font=\small] {\(K_x^{(1/2)}=2\)};
\draw[thick, decorate,decoration={calligraphic brace, amplitude=5pt, raise=15pt, mirror}] (7,0) -- (7.99,0)
node[midway, below=0.7cm, font=\small] {\(K_x^{(1/2)}=3.5\)};
\end{tikzpicture}
\item The random variable \(K_x^{(m)}\) is still discrete, and its pmf is:
\[
\prob{K_x^{(m)}=\frac{k}{m}}=\qx[\left.\frac{k}{m}\right|\frac{1}{m}]{x}
=\px[\frac{k}{m}]{x}\,\qx[\frac{1}{m}]{x+\frac{k}{m}},
\]
for any \(k\in\N_0\).
\item 
\label{it:1m-whole-life-fmlas}
\begin{tabular}{ccccc}
\toprule
&p.v.r.v.&APV&2nd moment&variance\\
\midrule
expression&\(v^{\color{brown}K_x^{(m)}+1/m}\)
&\(\displaystyle\sum_{k=0}^{\infty}v^{\color{brown}\frac{k+1}{m}}\px[\color{brown}\frac{k}{m}]{x}\,\qx[\color{brown}\frac{1}{m}]{x+{\color{brown}\frac{k}{m}}}\)
&\(\Ax{x}[(m)]@\;2\delta\)&\(\Ax[][2]{x}[(m)]-\qty(\Ax{x}[(m)])^2\)\\
notation&\(Z\)&\defn{\(\Ax{x}[(m)]\)}&\defn{\(\Ax[][2]{x}[(m)]\)}&\(\vari{Z}\)\\
\bottomrule
\end{tabular}
\end{enumerate}
\subsection{Valuating (Pure) Endowment, Term Life, Deferred, Variable Benefits
Insurances}
\begin{enumerate}
\item The following shows the relationship between different kinds of insurance
(in one approach of ``construction''):

\begin{tikzpicture}
\node[draw, brown] (wl) at (0,0) {whole life};
\node[draw, brown] (pe) at (5,0) {pure endowment};
\node[draw, purple] (dl) at (3,-4.2) {deferred contract};
\node[draw, violet] (tl) at (1,-2) {term life};
\node[draw, violet] (en) at (4,-2) {endowment};
\draw[pen colour=orange, thick, decorate,decoration={calligraphic brace, amplitude=10pt, raise=15pt, mirror}] (-1,-2) -- (7,-2)
node[purple, midway, below=1cm, draw] (vb) {variable benefits (different kinds)};
\draw[-Latex] (wl) -- (tl);
\draw[-Latex] (pe) -- (tl);
\draw[-Latex] (wl) -- (en);
\draw[-Latex] (pe) -- (en);
\node[brown] () at (9,0) {basic building blocks};
\node[violet] () at (9,-2) {``constructed'' contracts};
\node[purple] () at (9,-3.7) {variants};
\end{tikzpicture}
\subsubsection*{Pure Endowment}
\item Consider an \(n\)-year pure endowment contract with survival benefit \$1,
i.e., \$1 is payable at time \(n\) if the life insured is alive at that time.

\begin{note}
\(n\) is usually an integer, but not necessarily. This applies to all
\(n\)-year contracts.
\end{note}

\item \label{it:pure-endow-fmlas}
\begin{tabular}{ccccc}
\toprule
&p.v.r.v.&APV&2nd moment&variance\\
\midrule
expression&\(v^{n}\indicset{T_x>n}\)&\(v^n\px[n]{x}\)
&\(\Ex[n]{x}@\;2\delta\)&\(\Ex[n][2]{x}-(\Ex[n]{x})^2\)\\
notation&\(Z\)&\defn{\(\Ex[n]{x}\)} or \defn{\(\Ax{\pureendowxn}\)}
&\defn{\(\Ex[n][2]{x}\)} or \defn{\(\Ax[][2]{\pureendowxn}\)}
&\(\vari{Z}\)\\
\bottomrule
\end{tabular}

\begin{rmk}
\item The ``1'' on top of \(\angl{n}\) suggests that the \(n\)-year term is the
\emph{1st} thing to be ``gone'' (earllier than \(x\) being ``gone'', i.e.,
dead) for the benefit to be triggered. More details about this kind of notation
will be discussed in STAT3909.
\item There are not concepts like ``continuous'' and ``discrete'' for pure
endowment --- the benefit is always paid (potentially) at time \(n\) for an
\(n\)-year pure endowment. Conventionally we do \emph{not} write
``\(\Ax*{x:\itop{\angl{n}}}\)''.
\end{rmk}
\item The following provides some ``shortcuts'' for the formulas in
\labelcref{it:pure-endow-fmlas}:
\begin{proposition}
\label{prp:pure-endow-shortcut}
For any age \(x\) and any \(n\ge 0\),
\begin{enumerate}
\item \(\Ex[n][2]{x}=v^n\Ex[n]{x}\);
\item \(\vari{Z}=v^{2n}\px[n]{x}\,\qx[n]{x}\).
\end{enumerate}
\end{proposition}
\begin{pf}
Firstly, \(\Ex[n][2]{x}=v^{2n}\px[n]{x}=v^n\Ex[n]{x}\). Next,
\[\vari{Z}=\Ex[n][2]{x}-(\Ex[n]{x})^2=v^{2n}\px[n]{x}(1-\px[n]{x})
=v^{2n}\px[n]{x}\,\qx[n]{x}.\]
\end{pf}
\subsubsection*{Term Life Insurance}
\item Consider an \(n\)-year term life insurance with death benefit \(\$1\),
i.e., \$1 is only payable if the life insured dies within \(n\) years.
\item \label{it:cont-term-life-fmlas}
Continuous case:

\begin{tabular}{ccccc}
\toprule
&p.v.r.v.&APV&2nd moment&variance\\
\midrule
expression&\(e^{-\delta T_x}\indicset{T_x\le n}\)&\(\Ax*{x}-\Ex[n]{x}\Ax*{x+n}\)
or \(\displaystyle\int_{0}^{n}e^{-\delta t}\px[t]{x}\mu_{x+t}\,dt\)
&\(\Ax*{\termxn}@\;2\delta\)&\(\Ax*[][2]{\termxn}-\qty(\Ax*{\termxn})^2\)\\
notation&\(Z\)&\defn{\(\Ax*{\termxn}\)}
&\defn{\(\Ax*[][2]{\termxn}\)}
&\(\vari{Z}\)\\
\bottomrule
\end{tabular}

\begin{pf}
To get the first APV formula, note that \(Z=\underbrace{e^{-\delta T_x}}_{\text{whole life}}-v^ne^{-\delta
(T_x-n)}\indicset{T_x>n}\), so
\[
\expv{Z}=\Ax*{x}+v^n\prob{T_x>n}\expv{\left. e^{-\delta({\color{brown}T_x-n})}\right|{\color{brown}T_x>n}}
=\Ax*{x}+v^n\px[n]{x}\underbrace{\expv{e^{-\delta({\color{brown}T_{x+n}})}}}_{\Ax*{x+n}}.
\]
For the integral APV formula, it follows directly from the expression for p.v.r.v.
\end{pf}

\item \label{it:annual-term-life-fmlas}
Annual case:

\begin{tabular}{ccccc}
\toprule
&p.v.r.v.&APV&2nd moment&variance\\
\midrule
expression&\(v^{K_x+1}\indicset{K_x\le n-1}\)&\(\Ax{x}-\Ex[n]{x}\Ax{x+n}\)
or \(\displaystyle\sum_{k=0}^{n-1}v^{k+1}\px[k]{x}\qx{x+k}\)
&\(\Ax{\termxn}@\;2\delta\)&\(\Ax[][2]{\termxn}-\qty(\Ax{\termxn})^2\)\\
notation&\(Z\)&\defn{\(\Ax{\termxn}\)}
&\defn{\(\Ax[][2]{\termxn}\)}
&\(\vari{Z}\)\\
\bottomrule
\end{tabular}

\begin{pf}
Similar to the proof for \labelcref{it:cont-term-life-fmlas}.
\end{pf}

\item \label{it:1m-term-life-fmlas}
\(1/m\)thly case:

\begin{tabular}{ccccc}
\toprule
&p.v.r.v.&APV&2nd moment&variance\\
\midrule
expression&\(v^{K_x^{(m)}+\frac{1}{m}}\indicset{K_x^{(m)}\le n-\frac{1}{m}}\)
&\makecell{\(\Ax{x}[(m)]-\Ex[n]{x}\Ax{x+n}[(m)]\)\\
or \(\displaystyle\sum_{k=0}^{mn-1}v^{\frac{k+1}{m}}\px[\frac{k}{m}]{x}\,\qx{x+\frac{k}{m}}\)}
&\(\Ax{\termxn}@\;2\delta\)&\(\Ax[][2]{}[(m)]_\termxn-\qty(\Ax{}[(m)]_\termxn)^2\)\\
notation&\(Z\)&\defn{\(\Ax{}[(m)]_\termxn\)}
&\defn{\(\Ax[][2]{}[(m)]_\termxn\)}
&\(\vari{Z}\)\\
\bottomrule
\end{tabular}

\begin{pf}
Similar to the proof for \labelcref{it:cont-term-life-fmlas}.
\end{pf}
\subsubsection*{Endowment Insurance}
\item \label{it:endow-term-plus-pure} A key observation is that \(n\)-year
endowment insurance is just \(n\)-year term life insurance + \(n\)-year pure
endowment.
\item As a result, the p.v.r.v.'s in different cases can be written as:

\begin{tabular}{cc}
\toprule
case&p.v.r.v.\\
\midrule
continuous&\(e^{-\delta T_x}\indicset{T_x\le n}+e^{-\delta n}\indicset{T_x>n}\)\\
annual&\(v^{K_x+1}\indicset{K_x\le n-1}+v^n\indicset{K_x\ge n}\)\\
\(1/m\)thly&\(v^{K_x^{(m)}+\frac{1}{m}}\indicset{K_x^{(m)}\le n-\frac{1}{m}}+v^n\indicset{K_x^{(m)}\ge n}\)\\
\bottomrule
\end{tabular}

\begin{note}
Both events \(\{K_x\ge n\}\) and \(\qty{K_x^{(m)}\ge n}\) have the same probability as
the event \(\{T_x> n\}\).
\end{note}
\item But actually they can be written more compactly as follows:

\begin{tabular}{cc}
\toprule
case&p.v.r.v.\\
\midrule
continuous&\(e^{-\delta(T_x\wedge n)}\)\\
annual&\(v^{(K_x+1)\wedge n}\)\\
\(1/m\)thly&\(v^{\qty(K_x^{(m)}+\frac{1}{m})\wedge n}\)\\
\bottomrule
\end{tabular}
\item \label{it:cont-endow-fmlas}
Continuous case:

\begin{tabular}{ccccc}
\toprule
&p.v.r.v.&APV&2nd moment&variance\\
\midrule
expression&\(e^{-\delta(T_x\wedge n)}\)&\(\Ax*{\termxn}+\Ex[n]{x}\)
&\(\Ax*{\endowxn}@\;2\delta\)&\(\Ax*[][2]{\endowxn}-\qty(\Ax*{\endowxn})^2\)\\
notation&\(Z\)&\defn{\(\Ax*{\endowxn}\)}
&\defn{\(\Ax*[][2]{\endowxn}\)}
&\(\vari{Z}\)\\
\bottomrule
\end{tabular}

\begin{note}
When there is not ``1'' on top of either \(x\) or \(\angl{n}\), it indicates
that a benefit payment is triggered when \emph{any} one of them is ``gone''.
Again more details will be discussed in STAT3909.
\end{note}

\item \label{it:annual-endow-fmlas}
Annual case:

\begin{tabular}{ccccc}
\toprule
&p.v.r.v.&APV&2nd moment&variance\\
\midrule
expression&\(v^{(K_x+1)\wedge n}\)&\(\Ax{\termxn}+\Ex[n]{x}\)
&\(\Ax{\endowxn}@\;2\delta\)&\(\Ax[][2]{\endowxn}-\qty(\Ax{\endowxn})^2\)\\
notation&\(Z\)&\defn{\(\Ax{\endowxn}\)}
&\defn{\(\Ax[][2]{\endowxn}\)}
&\(\vari{Z}\)\\
\bottomrule
\end{tabular}

\item \label{it:1m-endow-fmlas}
\(1/m\)thly case:

\begin{tabular}{ccccc}
\toprule
&p.v.r.v.&APV&2nd moment&variance\\
\midrule
expression&\(v^{\qty(K_x^{(m)}+\frac{1}{m})\wedge n}\)&\(\Ax{}[(m)]_{\termxn}+\Ex[n]{x}\)
&\(\Ax{\endowxn}[(m)]@\;2\delta\)&\(\Ax[][2]{\endowxn}[(m)]-\qty(\Ax{\endowxn}[(m)])^2\)\\
notation&\(Z\)&\defn{\(\Ax{\endowxn}[(m)]\)}
&\defn{\(\Ax[][2]{\endowxn}[(m)]\)}
&\(\vari{Z}\)\\
\bottomrule
\end{tabular}

\item Due to the property in \labelcref{it:endow-term-plus-pure}, we have an
alternative method for calculating the \emph{variance} for p.v.r.v. for
endowment insurance, as follows:
\begin{proposition}
\label{prp:endow-var-alt-fmla}
Write \(Z=Z_1+Z_2\) where \(Z,Z_1,Z_2\) are the p.v.r.v.'s of the \(n\)-year
endowment, term life, and pure endowment insurances respectively. Then,
\begin{itemize}
\item (continuous case) \(\vari{Z}=\vari{Z_1}+\vari{Z_2}-2\Ax*{\termxn}\,\Ex[n]{x}\);
\item (annual case) \(\vari{Z}=\vari{Z_1}+\vari{Z_2}-2\Ax{\termxn}\,\Ex[n]{x}\);
\item (\(1/m\)thly case) \(\vari{Z}=\vari{Z_1}+\vari{Z_2}-2\Ax{}[(m)]_{\termxn}\,\Ex[n]{x}\).
\end{itemize}
\end{proposition}
\begin{pf}
First note that in any situation, one of \(Z_1\) and \(Z_2\) would be 0 (a life
either dies within or survive for \(n\) years). So, the product \(Z_1Z_2\) is
always 0, and hence the covariance term
\[\cov{Z_1,Z_2}=\expv{Z_1Z_2}-\expv{Z_1}\expv{Z_2}=-\expv{Z_1}\expv{Z_2}.\]
Now the result follows from the formula
\[
\vari{Z}=\vari{Z_1}+\vari{Z_2}+2\cov{Z_1,Z_2}.
\]
\end{pf}
\end{enumerate}
\subsection{Recursions for APVs}
\begin{enumerate}
\item There are two main reasons for considering recursions of APVs here:
\begin{enumerate}
\item It provides insight on the relationships of APVs at different ages.
\item It allows quick computations based on limited amount of information.
\end{enumerate}
\end{enumerate}
